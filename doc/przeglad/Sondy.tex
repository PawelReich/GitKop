\section{Sondy}
\todo[inline]{dokładniej opisać teorię}
Sonda to fundamentalna część wykrywacza metalu odpowiedzialna za interakcję z otoczeniem.

Podstawowy wzór na indukcyjność cewki soleidalnej:

\begin{equation}
    L = \frac{\mu N^2 A}{l}
    \label{eq:indukcyjnosc_cewki}
\end{equation}
gdzie $\mu$ oznacza przenikalność materiału użytego do wykonania cewki, N oznacza ilość wykonanych zwojów, A oznacza pole przekroju poprzecznego, l długość


\subsection{Przegląd wykorzystywanych typów cewek w wykrywaczach metali}

\todo[inline]{tutaj podeprzeć źródłami}

\subsubsection{Cewki koncentryczne}
\begin{figure}[H]
    \begin{center}
        \includegraphics[width=0.6\textwidth]{img/minelab_concentric_coil.png}
    \end{center}
    \caption{Schemat działania cewki koncentrycznej}\label{fig:minelab_concentric_coil}
    \hfill\small źródło: Minelab Electronics \cite{minelab_coils} 
\end{figure}
Cewki koncentryczne złożone są z dwóch uzwojeń: zewnętrznego, pełniącego funkcję nadawczą, oraz wewnętrzengo, odbierającą sygnał powrotny.
Dzięki wąskiemu stożkowemu kształtowi generowanego pola elektromagnetycznego, cewka koncentryczna dobrze spełnia się podczas próby dokłądnego określenia położenia szukanego obiektu.

\subsubsection{Cewki typu DD}
\begin{figure}[H]
    \begin{center}
        \includegraphics[width=0.6\textwidth]{img/minelab_double_d_coil.png}
    \end{center}
    \caption{Schemat działania cewki typu DD}\label{fig:minelab_double_d_coil}
    \hfill\small źródło: Minelab Electronics \cite{minelab_coils} 
\end{figure}
Cewki typu DD (Podwójne-D, ang. \textit{Double-D}). Składa się z dwóch częściowo nachodzących na siebie uzwojeń, jeden będący uzwojeniem nadawczym,
drugi odbiorczym. Dzięki częściowemu nałożeniu na siebie, problem balansu gruntowego jest częściowo zniwelowany poprzez przesunięcie fazowe powstałe
na skutek fizycznego przesunięcia uzwojeń. \todo[inline]{zweryfikować, podeprzeć}

\subsubsection{Cewka typu \textit{monoloop}}
\begin{figure}[H]
    \begin{center}
        \includegraphics[width=0.6\textwidth]{img/minelab_monoloop_coil.png}
    \end{center}
    \caption{Schemat działania cewki \textit{monoloop}}\label{fig:minelab_monoloop_coil}
    \hfill\small źródło: Minelab Electronics \cite{minelab_coils} 
\end{figure}

Cewki \textit{monoloop} najczęściej stosowane są w urządzeniach wykorzystujących technikę wykrywania impulsowego.
Składają się z jednego uzwojenia, pełniącego jednocześnie rolę nadawczą oraz odbiorczą.
Produkuje pole elektromagnetyczne w kształcie stożka.

\subsection{Analiza cewki typu \textit{monoloop}}

% Jak wspomniano wcześniej, cewka typu \textit{monoloop} stosowana jest w wykrywaczach impulsowych.
% W związku z budową prototypu wykrywacza metalu typu PI w niniejszym podroździale przybliżono teorię stojącą za tym rodzajem

\begin{figure}[H]
    \begin{center}
        \includegraphics[width=0.95\textwidth]{./img/multiturn_cewka.png}
    \end{center}
    \caption{Przekrój poprzeczny cewki monoloop}\label{fig:poprz}
    \hfill\small źródło: C. Moreland \cite{moreland_bfo_theory} 
\end{figure}

Wzór na przybliżoną indukcyjność cewki:
\begin{equation}
    L = \frac{0,8 (N A^2)}{6a + 9b + 10c}
    \label{eq:indukcyjnosc_monoloop}
\end{equation}

