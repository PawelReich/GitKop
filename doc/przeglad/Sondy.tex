\section{Przegląd wykorzystywanych typów cewek w wykrywaczach metali}

\subsection{Cewki koncentryczne}
\begin{figure}[H]
    \begin{center}
        \includegraphics[width=0.6\textwidth]{img/minelab_concentric_coil.png}
    \end{center}
    \caption{Schemat działania cewki koncentrycznej}\label{fig:minelab_concentric_coil}
    \hfill\small źródło: Minelab Electronics \cite{minelab_coils} 
\end{figure}

Najczęściej stosowany rodzaj sondy w wykrywaczach metalu. Cewki koncentryczne złożone są z dwóch uzwojeń: zewnętrznego, pełniącego funkcję nadawczą, oraz wewnętrznego, odbierającą sygnał powrotny.
Dzięki wąskiemu stożkowemu kształtowi generowanego pola elektromagnetycznego, cewka koncentryczna dobrze spełnia się podczas próby dokładnego określenia położenia szukanego obiektu \cite{moreland_bfo_theory}.

\subsection{Cewki typu DD}
\begin{figure}[H]
    \begin{center}
        \includegraphics[width=0.6\textwidth]{img/minelab_double_d_coil.png}
    \end{center}
    \caption{Schemat działania cewki typu DD}\label{fig:minelab_double_d_coil}
    \hfill\small źródło: Minelab Electronics \cite{minelab_coils} 
\end{figure}
Cewki typu DD (Podwójne-D, ang. \textit{Double-D}). Składa się z dwóch częściowo nachodzących na siebie uzwojeń, jeden będący uzwojeniem nadawczym,
drugi odbiorczym. Dzięki częściowemu nałożeniu na siebie, problem balansu gruntowego jest częściowo zniwelowany poprzez przesunięcie fazowe powstałe
na skutek fizycznego przesunięcia uzwojeń.

\subsection{Cewka typu \textit{monoloop}}
\begin{figure}[H]
    \begin{center}
        \includegraphics[width=0.6\textwidth]{img/minelab_monoloop_coil.png}
    \end{center}
    \caption{Schemat działania cewki \textit{monoloop}}\label{fig:minelab_monoloop_coil}
    \hfill\small źródło: Minelab Electronics \cite{minelab_coils} 
\end{figure}

Cewki \textit{monoloop} najczęściej stosowane są w urządzeniach wykorzystujących technikę wykrywania impulsowego.
Składają się z jednego uzwojenia, pełniącego jednocześnie rolę nadawczą oraz odbiorczą.
Produkuje pole elektromagnetyczne w kształcie stożka.

\section{Analiza cewki typu \textit{monoloop}}

\begin{figure}[H]
    \begin{center}
        \includegraphics[width=0.95\textwidth]{./img/multiturn_cewka.png}
    \end{center}
    \caption{Przekrój poprzeczny cewki monoloop}\label{fig:poprz}
    \hfill\small źródło: C. Moreland \cite{moreland_bfo_theory} 
\end{figure}

Wzór na przybliżoną indukcyjność cewki (formuła Wheelera, wymiary w calach) \cite{moreland2024inside}:
\begin{equation}
    L = \frac{0,8 (N A^2)}{6a + 9b + 10c}
    \label{eq:wheeler}
\end{equation}

