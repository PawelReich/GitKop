\section{Sondy}

Podstawowy wzór na indukcyjność cewki soleidalnej:

\begin{equation}
    L = \frac{\mu N^2 A}{l}
    \label{eq:indukcyjnosc_cewki}
\end{equation}
gdzie $\mu$ oznacza przenikalność materiału użytego do wykonania cewki, N oznacza ilość wykonanych zwojów, A oznacza pole przekroju poprzecznego, l długość

Wykrywacze metalu z reguły nie korzystają z tego rodzaju sondy.

\begin{figure}[H]
    \begin{center}
        \includegraphics[width=0.95\textwidth]{./img/multiturn_cewka.png}
    \end{center}
    \caption{Przekrój poprzeczny cewki}\label{fig:poprz}
\end{figure}


\begin{equation}
    L = \frac{0,8 (N A^2)}{6a + 9b + 10c}
    \label{eq:indukcyjnosc_dobra}
\end{equation}
