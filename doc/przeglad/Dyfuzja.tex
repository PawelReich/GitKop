\section{Dyfuzja magnetyczna}

Wykrywacze metalu z wbudowaną funkcjonalnością dyskryminacji metalu pozwalają na selektywne ofiltrowanie sygnałów od niepożądanych obiektów.
Działają na podstawie właściwości materiału badanego:
\begin{itemize}
    \item przenikalność magnetyczna
    \item przewodność właściwa
\end{itemize}
Umożliwia to użytkownikowi urządzenia natychmiastowe i nieinwazyjne ustalenie rodzaju wykrytego metalu.
Zwiększa to efektywność poszukiwań, pozwalając na wczesne odrzucenie sygnału w przypadku podejrzenia wykrycia niepożądanego obiektu.

\subsection{Wyprowadzenie wzoru}

Wyprowadzenie wzoru na dyfuzję magnetyczną przy założeniach określonych na wstępie do rozdziału można rozpocząć od wyznaczenia wzoru na gęstość prądu z równania Ampère'a-Maxwella:
\begin{equation}
    \label{eq:ampere}
    \nabla \times \mathbf{B} = \mu \mathbf{J}
\end{equation}
\begin{equation}
    \label{eq:gestosc_pradu}
    \mathbf{J} = \frac{1}{\mu} (\nabla \times \mathbf{B})
\end{equation}
gdzie $\mathbf{B}$ jest polem magnetycznym, $\mu$ jest przenikalnością magnetyczną materiału, $\mathbf{J}$ jest gęstością prądu elektrycznego.
\\Wzór na gęstość prądu wynikający z prawa Ohma:
\begin{equation}    
    \label{eq:ohma}
    \mathbf{J} = \sigma \mathbf{E}
\end{equation}
gdzie $\sigma$ oznacza konduktywność prądu, $\mathbf{E}$ oznacza pole elektryczne

Łącząc wzór \ref{eq:gestosc_pradu} ze wzorem na gęstość prądu wynikającą z prawa Ohma (\ref {eq:ohma}, otrzymano wzór na pole elektryczne wyrażone przez pole magnetyczne:
\begin{equation}
     \mathbf{E} = \frac{1}{\mu \sigma} (\nabla \times \mathbf{B})
     \label{eq:elektryczne_przez_magnetyczne}
\end{equation}
Wiedząc, że gradient pola elektycznego jest równy ujemnej zmianie pola magnetycznego (równanie Maxwella-Faradaya, \ref{eq:prawo_faradaya})
\begin{equation}
\nabla \times \mathbf{E} = - \frac{\partial \mathbf{B}}{\partial t}
      \tag{\ref{eq:prawo_faradaya}}
\end{equation}
podstawiono $\mathbf{E}$ ze wzoru \ref{eq:elektryczne_przez_magnetyczne}:
\begin{equation}
    \nabla \times \left( \frac{1}{\mu \sigma} (\nabla \times \mathbf{B}) \right) = - \frac{\partial \mathbf{B}}{\partial t}
    \label{eq:test}
\end{equation}
Korzystając z tożsamości wektorowej

$$\nabla \times (\nabla \times \mathbf{B}) = \nabla(\nabla \cdot \mathbf{B}) - \nabla^2 \mathbf{B}$$
otrzymano:
\begin{equation}
    - \frac{1}{\mu \sigma} \nabla^2 \mathbf{B} = - \frac{\partial \mathbf{B}}{\partial t}
    \label{eq:diffused_dirty}
\end{equation}
Po uporządkowaniu otrzymano równanie dyfuzji:
\begin{equation}
    \label{eq:diffused}
    \nabla^2 \mathbf{B} = \mu \sigma \frac{\partial \mathbf{B}}{\partial t}
\end{equation}

gdzie $\mu$ to absolutna przenikalność magnetyczna materiału, a $\sigma$ to jego konduktywność (przewodność właściwa). Równanie to wskazuje, że interakcja pola z metalem nie jest natychmiastowa, lecz podlega procesom relaksacyjnym zależnym od właściwości materiałowych obiektu oraz częstotliwości fali pobudzającej. Prowadzi to do zjawiska naskórkowości (ang. skin effect), które ogranicza penetrację prądów wirowych do warstw powierzchniowych przy wyższych częstotliwościach, co ma fundamentalne znaczenie dla identyfikacji obiektów.

\subsection{Zjawisko naskórkowości}

\begin{table}[H]
    \centering
    \caption{Rezystywność wybranych materiałów dla 1 atm w 20$^\circ$C.}
    \begin{tabular}{llll}
        \toprule
        \textbf{Materiał} & \textbf{Rezystywność ($\Omega \cdot m$)} & \textbf{Materiał} & \textbf{Rezystywność ($\Omega \cdot m$)} \\
    \midrule
        \textit{Przewodniki:} & & \textit{Materiały słabo przewodzące:} & \\
        Srebro & $1,59 \times 10^{-8}$ & Woda morska & $0,2$ \\
        Miedź & $1,68 \times 10^{-8}$ & German & $0,46$ \\
        Złoto & $2,21 \times 10^{-8}$ & Diament & $2,7$ \\
        Aluminium & $2,65 \times 10^{-8}$ & Krzem & $2500$ \\
        Żelazo & $9,61 \times 10^{-8}$ & \textit{Izolatory:} & \\
        Rtęć & $9,61 \times 10^{-7}$ & Woda (czysta) & $8,3 \times 10^3$ \\
        Nichrom & $1,08 \times 10^{-6}$ & Szkło & $10^9 - 10^{14}$ \\
        Mangan & $1,44 \times 10^{-6}$ & Guma & $10^{13} - 10^{15}$ \\
        Grafit & $1,6 \times 10^{-5}$ & Teflon & $10^{22} - 10^{24}$ \\
    \bottomrule
    \end{tabular}
    \vspace{0.2cm} \\
    \hfill\small źródło: Handbook of Chemistry and Physics \cite{haynes2010crc} 
\end{table}
\begin{table}[H]
    \centering
    \caption{Maksymalna przenikalność magnetyczna wybranych materiałów dla niskich częstotliwości}
    \begin{tabular}{l l l}
        \toprule
        \textbf{Substancja} & \textbf{Typ grupy} & \textbf{Względna przenikalność } $\mu_r$ \\
        \midrule
        Bizmut & Diamagnetyk & 0,99983 \\
        Srebro & Diamagnetyk & 0,99998 \\
        Ołów & Diamagnetyk & 0,99983 \\
        Miedź & Diamagnetyk & 0,99991 \\
        Woda & Diamagnetyk & 0,99991 \\
        \addlinespace
        Próżnia & Niemagnetyczny & 1 \\
        \addlinespace
        Powietrze & Paramagnetyk & 1,0000004 \\
        Aluminium & Paramagnetyk & 1,00002 \\
        Pallad & Paramagnetyk & 1,0008 \\
        \addlinespace
        Kobalt & Ferromagnetyk & 250 \\
        Nikiel & Ferromagnetyk & 600 \\
        Ferroxcube 3 (proszek ferrytowy Mn-Zn) & Ferromagnetyk & 1 500 \\
        Stal miękka (0,2\% C) & Ferromagnetyk & 2 000 \\
        Żelazo (0,2\% zanieczyszczeń) & Ferromagnetyk & 5 000 \\
        \addlinespace
        Żelazo krzemowe (4\% Si) & Ferromagnetyk & 7 000 \\
        Permalloy 78 (78,5\% Ni) & Ferromagnetyk & 100 000 \\
        Mumetal (75\% Ni, 5\% Cu, 2\% Cr) & Ferromagnetyk & 100 000 \\
        Oczyszczone żelazo (0,05\% zanieczyszczeń) & Ferromagnetyk & 200 000 \\
        Supermaloj (5\% Mo, 79\% Ni) & Ferromagnetyk & 1 000 000 \\
        \bottomrule
    \end{tabular}

    \vspace{0.2cm}
    \hfill\small źródło: Electromagnetics Explained, \cite{schmitt2002electromagnetics}, Electromagnetics with Applications \cite[p.~201]{kraus1999electromagnetics} 
\end{table}

