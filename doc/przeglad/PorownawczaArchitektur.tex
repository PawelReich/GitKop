\section{Analiza porównawcza architektur detektorów}

\subsection{Problemy różnych rodzajów wykrywaczy}
\todo[inline]{napisać}
\begin{itemize}
    \item Balans gruntowy
    \item Głębokość penetracji
    \item Szerokość penetracji
    \item Odporność na mineralizację gruntu
    \item Umiejętność dokładnego wskazywania położenia poszukiwanego obiektu
    \item Umiejętność dyskryminacji metali
\end{itemize}

\subsection{BFO - Beat Frequency Oscillator}


Technologia wykrywaczy metali oparta o generator zdudnieniowy (ang. \textit{Beat Frequency Oscillator}).
Podstawowy system BFO opiera się o dwa oscylatory, będące dostrojone do jak najbliższej wspólnej częstotliwości.
Do generowania żądanej częstotliwości wykorzystywane są przeważnie obwody rezonansowe LC.
Pierwszy z oscylatorów, referencyjny, korzysta z cewki o zadanej indukcyjności.
Oscylator szukający, używa sondy wykrywacza metali jako swoje źródło indukcyjności. \cite{moreland_bfo_theory}

\begin{figure}[H]
    \begin{center}
        \includegraphics[width=0.7\textwidth]{./img/bfo_uklad.png}
    \end{center}
    \caption{Schemat podstawowego systemu BFO}\label{fig:bfo}
\end{figure}
\todo[inline]{przerysowac i przetlumaczyc na polski}

Generowane przez układy rezonansowe sygnały wprowadzane są do mieszacza częstotliwości, który wyznacza sumę oraz różnicę obu częstotliwości.

Przy założeniu dwóch sygnałów sinusoidalnych:
\begin{equation}
    V_s(t) = A_s \cos(2\pi f_{search} t)
    \label{eq:bfo_v_s_t}
\end{equation}
gdzie $V_s(t)$ jest sygnałem sterującym cewką szukającą, $A_s$ amplidudą napięcia, $f_{search}$ częstotliwością sygnału.

oraz

\begin{equation}
    V_r(t) = A_r \cos(2\pi f_{ref} t)
    \label{eq:bfo_v_r_t}
\end{equation}
gdzie $V_r(t)$ jest sygnałem referencyjnym, $A_r$ amplidudą napięcia, $f_{ref}$ częstotliwością sygnału referencyjnego.

Wykonywane jest mieszanie sygnałów \ref{eq:bfo_v_s_t} oraz \ref{eq:bfo_v_r_t}, co idealnie oznacza ich mnożenie opisywanym przez tożsamość trygonometryczną:
\begin{equation}
    \cos(\alpha)\cos(\beta) = \frac{1}{2} [\cos(\alpha - \beta) + \cos(\alpha + \beta)]
    \label{eq:mnozenie_cosinusow}
\end{equation}
Produkt sygnałów opisuje równanie:
\begin{equation}
    V_{mix}(t) \propto \frac{A_s A_r}{2} [\cos(2\pi(f_{search} - f_{ref})t) + \cos(2\pi(f_{search} + f_{ref})t)]
    \label{eq:produkt_bfo}
\end{equation}
gdzie $V_{mix}(t)$ to sygnał wychodzący z mieszacza.

Wymieszany sygnał zawiera dwie składowe:

\begin{itemize}
    \item sumę częstotliwości, którą należy odfiltrować (około kilkuset kHz)
    \item różnicę częstotliwości, będącą słyszalnym dudnieniem (około kilku kHz)
\end{itemize}
Wymieszany sygnał jest następnie traktowany filtrem dolnoprzepustowym, usuwającym wysoką składową częstotliwości.
Ze względu na prostotę 


\subsection{IB - Induction Balance}

Oryginalnie przedstawiona przed doktora  \cite{doi:10.1098/rspl.1879.0012}.

\subsection{PI - Pulse Induction}
\cite{moreland2024inside}   

