\section{Analiza porównawcza architektur detektorów}

\subsection{BFO - Beat Frequency Oscillator}


Technologia wykrywaczy metali oparta o generator zdudnieniowy (ang. \textit{Beat Frequency Oscillator}).
Podstawowy system BFO opiera się o dwa oscylatory, będące dostrojone do jak najbliższej wspólnej częstotliwości.
Do generowania żądanej częstotliwości wykorzystywane są przeważnie obwody rezonansowe LC.
Pierwszy z oscylatorów, referencyjny, korzysta z cewki o zadanej indukcyjności.
Oscylator szukający, używa sondy wykrywacza metali jako swoje źródło indukcyjności. \cite{moreland_bfo_theory}

\begin{figure}[H]
    \begin{center}
        \includegraphics[width=0.7\textwidth]{./img/bfo_uklad.png}
    \end{center}
    \caption{Schemat podstawowego systemu BFO}\label{fig:bfo}
\end{figure}
\todo[inline]{przerysowac i przetlumaczyc na polski}

Generowane przez układy rezonansowe sygnały wprowadzane są do mieszacza częstotliwości, który wyznacza sumę oraz różnicę obu częstotliwości.
Wymieszany sygnał jest następnie traktowany filtrem dolnoprzepustowym, usuwającym wysokie składowe częstotliwości.
Wyjście filtru

\subsection{IB}

Przedstawiona przed doktora  \cite{doi:10.1098/rspl.1879.0012}.

\subsection{PI - Pulse Induction}
\cite{moreland2024inside}   

