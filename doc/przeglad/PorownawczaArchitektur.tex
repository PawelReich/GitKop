\section{Analiza porównawcza architektur detektorów}

\subsection{Problemy różnych rodzajów wykrywaczy}
Różne technologie wykrywaczy metali posiadają swoje mocne, jak i słabe strony.
W niniejszym rozdziale zostanie przeprowadzona analiza porównawcza popularnych architektur detektorów oraz omówione ich własności takie jak:

\begin{itemize}
    \item Balans gruntowy
    \item Głębokość penetracji
    \item Szerokość penetracji
    \item Odporność na mineralizację gruntu
    \item Umiejętność dokładnego wskazywania położenia poszukiwanego obiektu
    \item Umiejętność dyskryminacji metali
\end{itemize}

\subsection{BFO - Beat Frequency Oscillator}

Technologia wykrywaczy metali oparta o generator zdudnieniowy (ang. \textit{Beat Frequency Oscillator}).
Podstawowy system BFO opiera się o dwa oscylatory, będące dostrojone do jak najbliższej wspólnej częstotliwości.
Do generowania żądanej częstotliwości wykorzystywane są przeważnie obwody rezonansowe LC.
Pierwszy z oscylatorów, referencyjny, korzysta z cewki o zadanej indukcyjności.
Oscylator szukający, używa sondy wykrywacza metali jako swoje źródło indukcyjności. \cite{moreland_bfo_theory}

\begin{figure}[H]
    \begin{center}
\begin{tikzpicture}[
    mixer/.style={
        draw, 
        circle, 
        minimum size=1.5cm, 
        align=center, 
        font=\sffamily
    },
    label_text/.style={
        font=\large\itshape
    }
]

    \node[blok] (search_osc) at (0, 2.5) {Oscylator\\szukający};
    \node[blok] (ref_osc) at (0, -2.5) {Oscylator\\referencyjny};
    \node[mixer] (mixer) at (4, 0) {Mieszacz};
    \node[blok] (lpf) at (8, 0) {Filtr\\dolnoprzepustowy};

    \coordinate (output) at (11, 0);


    \draw[thick] ($(search_osc.west) + (-1.2, 0)$) coordinate (coil_center) circle (0.5cm);
    \draw[thick] (coil_center) circle (0.35cm);
    \draw[thick] ($(coil_center) + (0.35, 0.1)$) -- ($(search_osc.west) + (0, 0.1)$);
    \draw[thick] ($(coil_center) + (0.35, -0.1)$) -- ($(search_osc.west) + (0, -0.1)$);
    \draw[thick, decoration={coil, aspect=0.4, segment length=2mm, amplitude=2mm, pre length=1mm, post length=1mm}, decorate] 
        ($(ref_osc.west) + (-1.5, 0.4)$) -- ($(ref_osc.west) + (-1.5, -0.4)$);
    \draw[thick] ($(ref_osc.west) + (-1.5, 0.4)$) -- ($(ref_osc.west) + (0, 0.4)$);
    \draw[thick] ($(ref_osc.west) + (-1.5, -0.4)$) -- ($(ref_osc.west) + (0, -0.4)$);


    \draw[arrow] (search_osc.east) -| (mixer.north) 
        node[midway, left, yshift=10pt, label_text] {$f_{search}$};
    \draw[arrow] (ref_osc.east) -| (mixer.south) 
        node[midway, left, yshift=-10pt, label_text] {$f_{ref}$};
    \draw[arrow] (mixer.east) -- (lpf.west) 
        node[midway, above, label_text] {$f_s+f_r$}
        node[midway, below, label_text] {$f_s-f_r$};
    \draw[arrow] (lpf.east) -- (output) 
        node[right, label_text] {$f_s-f_r$};

\end{tikzpicture}
    \end{center}
    \caption{Schemat podstawowego systemu BFO}\label{fig:bfo}
\end{figure}

Kluczowym procesem zachodzącym w wykrywaczach wykorzystujących BFO jest mieszanie sygnałów. 
Generowane przez układy rezonansowe sygnały wprowadzane są do mieszacza częstotliwości, który wyznacza sumę oraz różnicę obu częstotliwości.

Przy założeniu częstotliwości generowanych przez oscylatory opisanych wzorami Thomsona:
\begin{equation}
    f_{ref} = \frac{1}{2\pi\sqrt{LC}}  
    \label{eq:bfo_f_ref}
\end{equation}
gdzie $L$ oraz $C$ to parametry odpowiednio indukcyjności oraz pojemności oscylatorów

\begin{equation}
    f_{search} = \frac{1}{2\pi\sqrt{(L_0 \pm \Delta L)C}}
    \label{eq:bfo_f_search}
\end{equation}
gdzie $L_0$ to indukcyjność własna cewki (idealnie $L$), $\Delta L$ to jej zmiana wywołana przez obiekt

Przy założeniu dwóch sygnałów sinusoidalnych:
\begin{equation}
    V_s(t) = A_s \cos(2\pi f_{search} t)
    \label{eq:bfo_v_s_t}
\end{equation}
gdzie $V_s(t)$ jest sygnałem sterującym cewką szukającą, $A_s$ amplidudą napięcia.

\begin{equation}
    V_r(t) = A_r \cos(2\pi f_{ref} t)
    \label{eq:bfo_v_r_t}
\end{equation}
gdzie $V_r(t)$ jest sygnałem referencyjnym, $A_r$ amplidudą napięcia.

Wykonywane jest mieszanie sygnałów \ref{eq:bfo_v_s_t} oraz \ref{eq:bfo_v_r_t}, co idealnie oznacza ich mnożenie opisywanym przez tożsamość trygonometryczną:
\begin{equation}
    \cos(\alpha)\cos(\beta) = \frac{1}{2} [\cos(\alpha - \beta) + \cos(\alpha + \beta)]
    \label{eq:mnozenie_cosinusow}
\end{equation}
Produkt sygnałów opisuje równanie:
\begin{equation}
    V_{mix}(t) \propto \frac{A_s A_r}{2} [\cos(2\pi(f_{search} - f_{ref})t) + \cos(2\pi(f_{search} + f_{ref})t)]
    \label{eq:produkt_bfo}
\end{equation}
gdzie $V_{mix}(t)$ to sygnał wychodzący z mieszacza.

Wymieszany sygnał zawiera dwie składowe:

\begin{itemize}
    \item sumę częstotliwości, którą należy odfiltrować (około kilkuset kHz)
    \item różnicę częstotliwości, będącą słyszalnym dudnieniem (około kilku kHz)
\end{itemize}
Wymieszany sygnał jest następnie traktowany filtrem dolnoprzepustowym, usuwającym wysoką składową częstotliwości.


Ze względu na prostotę wykrywacze stosujące technologię BFO rzadko są stosowane w warunkach profesjonalnych.
Urządzenia oparte o generator zdudnieniowy wymagają częstej rekalibracji, szczególnie w wysokozmineralizowanych glebach \cite{moreland_bfo_theory}.


\subsection{IB - Induction Balance}

\begin{figure}[H]
    \begin{center}
    \begin{tikzpicture}
        \node[blok] (nadawczy) at (0,0) {Układ\\nadawczy};
        \node[blok] (odbiorczy) at (7,0) {Układ\\odbiorczy};

        \coordinate (middle) at ($(nadawczy.east)!0.5!(odbiorczy.west)$);
        \draw[connector] ($(middle)+(-0.6,0)$) circle (0.8cm);
        \draw[connector] ($(middle)+(0.6,0)$) circle (0.8cm);

        \draw[connector] (nadawczy.east) -- ($(middle)+(-0.6-0.8, 0)$);
        \draw[connector] (odbiorczy.west) -- ($(middle)+(0.6+0.8, 0)$);
    \end{tikzpicture}
    \end{center}
    \caption{Schemat działania wykrywacza balansu indukcyjnego}\label{fig:ib_schemat}
\end{figure}


Oryginalnie przedstawiona przed doktora Davida Edwarda Hughesa \cite{doi:10.1098/rspl.1879.0012} w 1879 technologia z powodzeniem używana współcześnie.
Sonda zbudowana jest z dwóch obwodów - nadawczego oraz odbiorczego. Cewka nadawcza sterowana jest stabilnym sygnałem sinusoidalnym o niskiej częstotliwości (typowo 5-20kHz) \cite{Mazurek_Wdowiak_2015}.
Z tego powodu technologia ta jest znana również pod nazwą VLF (ang. Very Low Frequency).

Wytwarzane przez cewkę nadawczą zmienne pole elektromagnetyczne odbierane jest przez cewkę odbiorczą.
W stanie neutralnym (brak przewodników w pobliżu) kostrukcja sondy powoduje zminimalizowanie sygnału powrotnego.
Umieszczenie w zasięgu działania sondy obiektu mającego wpływ na generowane pole elektromagnetyczne powoduje zakłócenie równowagi.
Rodzaj przewodnika, a także wielkość oraz odległość od obiektu ma wpływ na amplitudę oraz fazę sygnału odbieranego przez cewkę odbiorczą, co pozwala na przeprowadzenie dyskryminacji rodzaju metalu, z jakiego jest stworzony badany obiekt.

\subsection{PI - Pulse Induction}

W przeciwieństwie do wcześniej opisywanych, technologia indukcji impulsowej (ang. \textit{Pulse Induction}) operuje w dziedzinie czasu. 
Sonda najczęściej zbudowana z pojedynczej cewki typu monoloop pełniąca rolę nadawczej oraz odbiorczej.
Wykrywacz generuje impuls prądu wytwarzający pole magnetyczne penetrujące przestrzeń. Po gwałtownym odcięciu przepływu prądu, pole zanika, co indukuje prądy wirowe w badanym obiekcie \cite{pi_performance}.
Układ odbiorczy mierzy oraz porównuje zmianę prędkości zaniku sygnału powracającego oraz jego kształt \cite{Mazurek_Wdowiak_2015} \cite{moreland2024inside}.

Pasożytnicza pojemność cewki $C_p$ (wraz z pojemnością kabla) tworzy z indukcyjnością $L$ obwód rezonansowy, który ma tendencję do oscylacji. Oscylacje te maskują słabe sygnały pochodzące z badanych obiektów w fazie odbioru. Aby wygasić oscylacje w minimalnym czasie bez przeregulowania, konieczne jest zastosowanie rezystora tłumiącego $R_d$ włączonego równolegle do cewki.

Wartość dla tłumienia krytycznego wynosi:
\begin{equation}
    R_{crit} = \frac{1}{2} \sqrt{\frac{L}{C_p}}
    \label{eq:pi_crit}
\end{equation}
