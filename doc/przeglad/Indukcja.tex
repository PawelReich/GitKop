\section{Zasada indukcji i konstrukcja cewki}

Sonda to część wykrywacza metalu odpowiedzialna za interakcję z otoczeniem.
Stanowi interfejs fizyczny pomiędzy układem elektronicznym wykrywacza a badanym ośrodkiem.
Jej rolą jest generowanie strumienia magnetycznego penetrującego grunt, ale również odbiór sygnałów wtórnych,
w których zaszyte informacje na temat obiektów są często wiele rzędów wielkości słabsze od wszechobecnego tła.

Podstawą działania sondy wykrywacza metali jest zjawisko indukcji elektromagnetycznej, polegające na sprzęganiu dwóch obwodów poprzez zmienne pole magnetyczne. Aby wzmocnić pole magnetyczne, przewód formowany jest w cewkę o wielu zwojach, co pozwala na "skupienie" strumienia magnetycznego w jej centrum. Natężenie generowanego pola magnetycznego zależy od natężenia prądu, wymiarów cewki oraz liczby zwojów \cite{moreland_induction}.


Dla cewki solenoidalnej maksymalne natężenie pola magnetycznego $\mathbf{B}$ w jej centrum można wyrazić wzorem:

\begin{equation}
    \mathbf{B}=\frac{\mu NI}{2R}
    \label{eq:coil_field}
\end{equation}
gdzie $\mu$ - przenikalność magnetyczna rdzenia (dla cewek powietrznych $\mu \approx 4\pi\times10^{-7}$ H/m), $N$ - liczba zwojów cewki, $I$ - natężenie prądu płynącego przez cewkę, $R$ - promień cewki.

Podstawowy wzór na indukcyjność cewki soleidalnej:

\begin{equation}
    L = \frac{\mu N^2 A}{l}
    \label{eq:indukcyjnosc_cewki}
\end{equation}
gdzie $A$ oznacza pole przekroju poprzecznego, $l$ długość cewki

\subsection{Sprzężenie zwrotne i przesunięcia fazowe}

Zgodnie z przetoczonym we wcześniejszej sekcji prawem Faradaya (\ref{eq:prawo_faradaya}), zmienne pole magnetyczne cewki nadawczej indukuje prądy wirowe w przewodzącym obiekcie znajdującym się w zasięgu pola. Prądy te krążą w materiale, wytwarzając własne, wtórne pole elektromagnetyczne, skierowane przeciwnie do pola pierwotnego.

Istotnym aspektem detekcji jest analiza przesunięcia fazowego. W procesie indukcji występuje przesunięcie o 90 stopni między polem elektromagnetycznym a indukowanym napięciem. W rezultacie, pole wtórne wytworzone przez prądy wirowe jest przesunięte w fazie względem pola nadawczego.

Wykrywacze dyskryminujące wykorzystują ten fakt, analizując powstałe przesunięcie fazowe w celu odróżnienia wykrytego rodzaju przewodnika. Metale o wysokiej przewodności (np. srebro) wykazują przesunięcie fazowe zbliżone do $180^{\circ}$ względem sygnału nadawczego \cite{moreland_induction}.


\subsection{Balans gruntowy}

\begin{figure}[H]
    \begin{center}
        \includegraphics[width=0.3\textwidth]{img/efekt_gruntu.png}
    \end{center}
    \caption{Wpływ gruntu na tworzone pole magnetyczne}\label{fig:efekt_gruntu}
    \hfill\small źródło: C. Moreland \cite{moreland_coils} 
\end{figure}

Ziemia nie jest ośrodkiem neutralnym elektromagnetycznie. Zawiera tlenki żelaza (magnetyt, maghemit) o właściwościach ferromagnetycznych oraz sole przewodzące prąd (woda morska) \cite{soil_properties}.
Elementy te zaburzają pole magnetyczne, co wpływa na sygnał powracający do wykrywacza oraz głębokość penetracji fali w kierunku gruntu \cite{moreland_coils}.
Wykrywacze metalu posiadają mechanizm niwelowania wpływu gruntu na końcową informację o wykrytym obiekcie przedstawianą użytkownikowi.
W prostszych urządzeniach jest on realizowany za pomocą pokrętła umożliwającego wprowadzenie oraz konfigurację balansu gruntowego usuwającego efekt tła (ang. \textit{ground balance}).


