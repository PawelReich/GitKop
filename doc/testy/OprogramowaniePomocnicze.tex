\section{Oprogramowanie pomocnicze}

W celu zwizualizowania oraz wykreślenia danych testowych generowanych w czasie rzeczywistym przez prototyp wykrywacza metalu powstał program pomocniczy.
Napisany w języku Python, łączy się z urządzeniem za pomocą peryferium \textit{USART1} (Uniwersalny Synchroniczny Asynchroniczny Nadajnik-Odbiornik, ng. \textit{Universal Synchronous Asynchronous Receiver Transmitter}) wbudowanego w mikrokontroler.
Połączenie z komputerem następuje dzięki urządzeniu ST-Link oferującemu, oprócz interfejsu SWD (Szeregowy interfejs debugowania, ang. \textit{Serial Wire Debug}), konwerter komunikacji szeregowej TTL (logika tranzystor-tranzystor, ang. \textit{Transistor-Transistor Logic}). Dzięki zastosowaniu języka Python, program pozwala na szybką iterację, przetwarzanie oraz prezentację danych generowanych przez wykrywacz. Wykrywacz wysyła pakiet danych z częstotliwością 10 Hz, w których zawarta jest aktualna zawartość bufora nachylenia oraz wewnętrzne stany wykrywacza. Na wykresach prezentowanych w tym rozdziale ilustrowane jest 100 ostatnich próbek danych, reprezentujących 10 sekund pomiarów.
\begin{figure}[H]
    \centering
    \includegraphics[width=0.7\textwidth]{img/testy/test_ss.png}
    \caption{Zrzut ekranu oprogramowania pomocniczego}\label{fig:test_program}
\end{figure}
