\section{Test działania prototypu}

Podczas pracy wykrywacza pobierany prąd wynosił stałe $60mA$, przy przyjętym wcześniej napięciu zasilania $10V$, daje to stałe obciążenie $0.6W$, czyniąc prototyp energooszczędnym.

\begin{figure}[H]
    \centering
    \includegraphics[width=0.9\textwidth]{img/test_bufor.png}
    \caption{Prototyp wykrywacza metali podłączony do aparatury testowej}
    \label{fig:wykrywacz}
\end{figure}

Program umożliwia wykres bufora próbek zebranych przez konwerter analogowo-cyfrowy. Jak widać na rysunku \ref{fig:bufor_wyglad}, prototyp poprawnie agreguje informacje zwrotne pochodzące z sondy.

\begin{figure}[H]
    \centering
    \includegraphics[width=0.9\textwidth]{img/test_bufor.png}
    \caption{Wykres widoku bufora}\label{fig:bufor_wyglad}
\end{figure}

Urządzenie poprawnie wyznacza czas trwania odpowiedzi sondy, wraz z filtracją szumów. Na rysunku \ref{fig:bufor_czas} wykreślono surowe, pomierzone dane próbek wraz ze stanami filtrów EMA oraz końcowym, wynikowym sygnałem
będący podstawą do podjęcia decyzji o stwierdzonej anomalii sygnału.


\begin{figure}[h]
    \centering
    \includegraphics[width=0.9\textwidth]{img/test_bufor.png}
    \caption{Wykres wewnętrznych stanów wykrywacza}\label{fig:bufor_czas}
\end{figure}
