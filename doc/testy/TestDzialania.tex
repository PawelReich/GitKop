\section{Test działania prototypu}

Podczas pracy wykrywacza pobierany prąd zawarty był w zakresie $(50mA - 60mA)$, przy przyjętym wcześniej napięciu zasilania $10V$ daje to stałe obciążenie około $0,5W$, czyniąc omawiany prototyp energooszczędnym.

\begin{figure}
    \centering
    \includegraphics[width=0.9\textwidth, angle=-90]{img/testy/prototyp.jpg}
    \caption{Prototyp wykrywacza metali podłączony do aparatury testowej}
    \label{fig:wykrywacz}
\end{figure}

Program umożliwia wykres bufora próbek zebranych przez konwerter analogowo-cyfrowy. Jak widać na rysunku \ref{fig:bufor_wyglad}, prototyp poprawnie agreguje informacje zwrotne pochodzące z sondy.

\begin{figure}
    \centering
    \includesvg[width=0.9\textwidth]{img/testy/bufor.svg}
    \caption{Wykres widoku bufora}\label{fig:bufor_wyglad}
\end{figure}

Urządzenie poprawnie wyznacza czas trwania odpowiedzi sondy, wraz z filtracją szumów. Na rysunku \ref{fig:filtry_wewnetrzne} wykreślono surowe, pomierzone dane próbek wraz ze stanami filtrów EMA oraz końcowym, wynikowym sygnałem
będący podstawą do podjęcia decyzji o stwierdzonej anomalii sygnału. Na wykresie \ref{fig:filtry_wykryte} zauważyć można zmianę wartości spowodowaną obecnością obiektu badanego w pobliżu.


\begin{figure}
    \centering
    \includesvg[width=\textwidth]{img/testy/filtry.svg}
    \caption{Wykres wewnętrznych stanów wykrywacza}\label{fig:filtry_wewnetrzne}
\end{figure}

\begin{figure}
    \centering
    \includesvg[width=\textwidth]{img/testy/filtry_wykryte.svg}
    \caption{Wykres wewnętrznych stanów wykrywacza z obiektem w pobliżu}\label{fig:filtry_wykryte}
\end{figure}


