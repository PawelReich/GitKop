\section{Analiza skuteczności działania prototypu}

Analiza została przeprowadzona z wykorzystaniem metalowego obiektu badanego (widocznego na rysunku \ref{fig:obiekt_badany}) oddalonego od sondy wykrywacza na 15 cm.


\begin{figure}
    \centering
    \includegraphics[width=0.7\textwidth]{img/testy/obiekt.jpg}
    \caption{Obiekt zastosowany podczas analizy skuteczności prototypu wykrywacza metali}
    \label{fig:obiekt_badany}
\end{figure}

Jak widać na wykresie \ref{fig:test_obiekt} prototyp poprawnie reaguje na obecność obiektu.
Fluktuacja stanu przekracza próg anomalii generując sygnał akustyczny (zaznaczony na wykresie).

\begin{figure}
    \centering
    \includesvg[width=\textwidth]{img/testy/surowe_wykryte.svg}
    \caption{Wykres stanu wykrywacza w czasie, z zakolorowanymi segmentami ze stwierdzoną anomalią}
    \label{fig:test_obiekt}
\end{figure}

Analiza nachylenia końcowej części odpowiedzi sondy nie wykazała znaczących zmian mogących być podstawą do dalszej analizy.

\begin{figure}
    \centering
    \includesvg[width=\textwidth]{img/testy/nachylenie.svg}
    \caption{Wykres zarejestrowanych nachyleń odpowiedzi sondy}
    \label{fig:test_nachylenie}
\end{figure}
