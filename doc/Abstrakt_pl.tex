\chapter*{Streszczenie}

Tematem niniejszej pracy dyplomowej jest projekt oraz realizacja prototypu wykrywacza metali działającego w technologii impulsowej (ang. \textit{Pulse Induction}).
Część teoretyczna zawiera rys historyczny rozwoju metod detekcji, począwszy od odkryć w XIX wieku, kończąc na współczesnych rozwiązaniach.
Dokonano analizy zjawisk fizycznych, takich jak prawo indukcji Faradaya, czy prądy wirowe, które stanowią fundament działania urządzenia.
Przeprowadzono także przegląd istniejących popularnych architektur wykrywaczy metali oraz typów sond, uzasadniając wybór technologii.
W części praktycznej szczegółowo opisano proces budowy prototypu opartego na mikrokontrolerze \textit{STM32H523CE} z rdzeniem \textit{ARM Cortex M-33}.
Wytłumaczono przebieg pracy utworzonego oprogramowania realizującego akwizycję danych w czasie rzeczywistym oraz detekcję anomalii sygnałowych poprzez analizę czasu odpowiedzi oraz nachylenia spadku.
Projekt zakłada wykonanie funkcjonalnego urządzenia mobilnego, wraz z wyprodukowaną płytką drukowaną oraz sondą zamknięte w zaprojektowanej obudowanie wykonanej w technologii druku 3D.

\bigskip

\noindent\textbf{Słowa kluczowe:} wykrywacz metali, indukcja impulsowa, cyfrowe przetwarzanie sygnałów, systemy wbudowane

\bigskip

\noindent\textbf{Dziedzina nauki i techniki zgodna z OECD} Nauki inżynieryjne i techniczne, Elektrotechnika, elektronika i inżynieria informatyczna, Robotyka i Automatyka
