\chapter*{Abstract}

The subject of this thesis is the design and implementation of a metal detector prototype operating in Pulse Induction technology. The theoretical part contains a historical outline of the development of detection methods, starting from discoveries in the 19th century and ending with modern solutions. An analysis of physical phenomena, such as Faraday's law of induction or eddy currents, which constitute the foundation of the device's operation, was performed. A review of existing popular metal detector architectures and coil types was also conducted, justifying the choice of technology. In the practical part, the process of building a prototype based on the STM32H523CE microcontroller with an ARM Cortex M-33 core is described in detail. The operation of the created software implementing real-time data acquisition and signal anomaly detection through the analysis of response time and decay slope is explained. The project assumes the creation of a functional mobile device, along with a manufactured printed circuit board and a probe enclosed in a designed housing made using 3D printing technology.

\bigskip

\noindent\textbf{Keywords:} metal detector, pulse induction, digital signal processing, embedded systems

\bigskip

\noindent\textbf{OECD consistent field of science and technology classification:} Engineering and technology, Electrical engineering, electronic engineering, information engineering, Robotics and automatic control
