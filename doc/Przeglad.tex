\chapter{Przegląd literatury}
W niniejszym przeglądzie przyjęto założenia upraszczające szeroko stosowane w literaturze naukowej dotyczącej podstaw detekcji metali, pozwalających jednocześnie na uzyskanie rozwiązań o wysokiej wartości podczas rozważań inżynierskich \cite[p.~308]{griffiths1999introduction}:


\begin{itemize}
    \item Quasi-statyczna aproksymacja,
    \item Badany obiekt jest traktowany jako jednorodna sfera o zadanej konduktywności $\sigma$, \\przenikalności magnetycznej $\mu$ i promieniu $R$,
    \item Badany obiekt nie porusza się ($\mathbf{v}=0$).
\end{itemize}

\section{Elektrodynamika systemów wykrywania metali}

Podstawą funkcjonowania wszystkich indukcyjnych wykrywaczy metali jest prawo indukcji Faradaya.
Zjawisko to wiąże w czasie pole magnetyczne z indukowanym wirowym polem elektrycznym \cite[p.~302]{griffiths1999introduction}.
Zgodnie z równaniem
\begin{equation}
\label{eq:prawo_faradaya}
\nabla \times \mathbf{E} = - \frac{\partial \mathbf{B}}{\partial t}
\end{equation}
gdzie $\mathbf{E}$ oznacza wektor natężenia, a $\mathbf{B}$ oznacza wektor indukcji magnetycznej.

\begin{figure}[htbp]
  \centering
  \includesvg[width=0.3\textwidth]{./img/Alternator_1.svg}
  \caption{Schemat alternatora przedstawiający obracający się magnes (wirnik) i nieruchome uzwojenie drutu (stator) oraz napięcie wytwarzane, gdy obracające się pole magnetyczne indukuje prąd w przewodzie.}
    \hfill\small źródło: wikipedia.org \cite{alternator_svg} 
\end{figure}

W kontekście inżynierii systemów detekcji cewka nadawcza zasilana prądem zmiennym lub impulsowym generuje zmienny strumień magnetyczny. Strumień ten penetruje przestrzeń, w tym potencjalne obiekty przewodzące oraz grunt. Zgodnie z przytoczonym prawem, zmienność strumienia w czasie ($\partial \mathbf{B}/\partial t$) indukuje siłę elektromotoryczną w każdym przewodniku objętym polem.

Kluczowym mechanizmem, na którym opiera się detekcja, jest powstawanie prądów wirowych (ang. eddy currents). Prądy te, płynąc w zamkniętych pętlach wewnątrz materiału przewodzącego (w płaszczyznach prostopadłych do linii pola magnetycznego), generują własne, wtórne pole magnetyczne \cite{kriezis}. Istotę tego zjawiska wyjaśnia reguła Lenza, która stanowi, że kierunek indukowanego prądu jest taki, iż wytworzone przez niego pole magnetyczne przeciwdziała zmianie pola pierwotnego, które je wywołało \cite{kriezis}.

\begin{equation}
    \label{eq:prawo_lenza}
    \mathcal{E} = -\frac{d \Phi_\mathbf{B}}{d t}
\end{equation}

To wzajemne oddziaływanie pól – pierwotnego (wymuszającego) i wtórnego (reakcyjnego) – prowadzi do mierzalnych zmian impedancji wzajemnej układu cewek lub zmian napięcia indukowanego w cewce odbiorczej, co stanowi sygnał użyteczny dla wykrywacza metalu.


\section{Dyfuzja magnetyczna}

Wykrywacze metalu z wbudowaną funkcjonalnością dyskryminacji metalu pozwalają na selektywne ofiltrowanie sygnałów od niepożądanych obiektów.
Działają na podstawie właściwości materiału badanego:
\begin{itemize}
    \item przenikalność magnetyczna
    \item przewodność właściwa
\end{itemize}
Umożliwia to użytkownikowi urządzenia natychmiastowe i nieinwazyjne ustalenie rodzaju wykrytego metalu.
Zwiększa to efektywność poszukiwań, pozwalając na wczesne odrzucenie sygnału w przypadku podejrzenia wykrycia niepożądanego obiektu.

\subsection{Wyprowadzenie wzoru}

Wyprowadzenie wzoru na dyfuzję magnetyczną przy założeniach określonych na wstępie do rozdziału można rozpocząć od wyznaczenia wzoru na gęstość prądu z równania Ampère'a-Maxwella:
\begin{equation}
    \label{eq:ampere}
    \nabla \times \mathbf{B} = \mu \mathbf{J}
\end{equation}
\begin{equation}
    \label{eq:gestosc_pradu}
    \mathbf{J} = \frac{1}{\mu} (\nabla \times \mathbf{B})
\end{equation}
gdzie $\mathbf{B}$ jest polem magnetycznym, $\mu$ jest przenikalnością magnetyczną materiału, $\mathbf{J}$ jest gęstością prądu elektrycznego.
\\Wzór na gęstość prądu wynikający z prawa Ohma:
\begin{equation}    
    \label{eq:ohma}
    \mathbf{J} = \sigma \mathbf{E}
\end{equation}
gdzie $\sigma$ oznacza konduktywność prądu, $\mathbf{E}$ oznacza pole elektryczne

Łącząc wzór \ref{eq:gestosc_pradu} ze wzorem na gęstość prądu wynikającą z prawa Ohma (\ref {eq:ohma}, otrzymano wzór na pole elektryczne wyrażone przez pole magnetyczne:
\begin{equation}
     \mathbf{E} = \frac{1}{\mu \sigma} (\nabla \times \mathbf{B})
     \label{eq:elektryczne_przez_magnetyczne}
\end{equation}
Wiedząc, że gradient pola elektycznego jest równy ujemnej zmianie pola magnetycznego (równanie Maxwella-Faradaya, \ref{eq:prawo_faradaya})
\begin{equation}
\nabla \times \mathbf{E} = - \frac{\partial \mathbf{B}}{\partial t}
      \tag{\ref{eq:prawo_faradaya}}
\end{equation}
podstawiono $\mathbf{E}$ ze wzoru \ref{eq:elektryczne_przez_magnetyczne}:
\begin{equation}
    \nabla \times \left( \frac{1}{\mu \sigma} (\nabla \times \mathbf{B}) \right) = - \frac{\partial \mathbf{B}}{\partial t}
    \label{eq:test}
\end{equation}
Korzystając z tożsamości wektorowej

$$\nabla \times (\nabla \times \mathbf{B}) = \nabla(\nabla \cdot \mathbf{B}) - \nabla^2 \mathbf{B}$$
otrzymano:
\begin{equation}
    - \frac{1}{\mu \sigma} \nabla^2 \mathbf{B} = - \frac{\partial \mathbf{B}}{\partial t}
    \label{eq:diffused_dirty}
\end{equation}
Po uporządkowaniu otrzymano równanie dyfuzji:
\begin{equation}
    \label{eq:diffused}
    \nabla^2 \mathbf{B} = \mu \sigma \frac{\partial \mathbf{B}}{\partial t}
\end{equation}

gdzie $\mu$ to absolutna przenikalność magnetyczna materiału, a $\sigma$ to jego konduktywność (przewodność właściwa). Równanie to wskazuje, że interakcja pola z metalem nie jest natychmiastowa, lecz podlega procesom relaksacyjnym zależnym od właściwości materiałowych obiektu oraz częstotliwości fali pobudzającej. Prowadzi to do zjawiska naskórkowości (ang. skin effect), które ogranicza penetrację prądów wirowych do warstw powierzchniowych przy wyższych częstotliwościach, co ma fundamentalne znaczenie dla identyfikacji obiektów.

\subsection{Zjawisko naskórkowości}

\begin{table}[H]
    \centering
    \caption{Rezystywność wybranych materiałów dla 1 atm w 20$^\circ$C.}
    \begin{tabular}{llll}
        \toprule
        \textbf{Materiał} & \textbf{Rezystywność ($\Omega \cdot m$)} & \textbf{Materiał} & \textbf{Rezystywność ($\Omega \cdot m$)} \\
    \midrule
        \textit{Przewodniki:} & & \textit{Materiały słabo przewodzące:} & \\
        Srebro & $1,59 \times 10^{-8}$ & Woda morska & $0,2$ \\
        Miedź & $1,68 \times 10^{-8}$ & German & $0,46$ \\
        Złoto & $2,21 \times 10^{-8}$ & Diament & $2,7$ \\
        Aluminium & $2,65 \times 10^{-8}$ & Krzem & $2500$ \\
        Żelazo & $9,61 \times 10^{-8}$ & \textit{Izolatory:} & \\
        Rtęć & $9,61 \times 10^{-7}$ & Woda (czysta) & $8,3 \times 10^3$ \\
        Nichrom & $1,08 \times 10^{-6}$ & Szkło & $10^9 - 10^{14}$ \\
        Mangan & $1,44 \times 10^{-6}$ & Guma & $10^{13} - 10^{15}$ \\
        Grafit & $1,6 \times 10^{-5}$ & Teflon & $10^{22} - 10^{24}$ \\
    \bottomrule
    \end{tabular}
    \vspace{0.2cm} \\
    \hfill\small źródło: Handbook of Chemistry and Physics \cite{haynes2010crc} 
\end{table}
\begin{table}[H]
    \centering
    \caption{Maksymalna przenikalność magnetyczna wybranych materiałów dla niskich częstotliwości}
    \begin{tabular}{l l l}
        \toprule
        \textbf{Substancja} & \textbf{Typ grupy} & \textbf{Względna przenikalność } $\mu_r$ \\
        \midrule
        Bizmut & Diamagnetyk & 0,99983 \\
        Srebro & Diamagnetyk & 0,99998 \\
        Ołów & Diamagnetyk & 0,99983 \\
        Miedź & Diamagnetyk & 0,99991 \\
        Woda & Diamagnetyk & 0,99991 \\
        \addlinespace
        Próżnia & Niemagnetyczny & 1 \\
        \addlinespace
        Powietrze & Paramagnetyk & 1,0000004 \\
        Aluminium & Paramagnetyk & 1,00002 \\
        Pallad & Paramagnetyk & 1,0008 \\
        \addlinespace
        Kobalt & Ferromagnetyk & 250 \\
        Nikiel & Ferromagnetyk & 600 \\
        Ferroxcube 3 (proszek ferrytowy Mn-Zn) & Ferromagnetyk & 1 500 \\
        Stal miękka (0,2\% C) & Ferromagnetyk & 2 000 \\
        Żelazo (0,2\% zanieczyszczeń) & Ferromagnetyk & 5 000 \\
        \addlinespace
        Żelazo krzemowe (4\% Si) & Ferromagnetyk & 7 000 \\
        Permalloy 78 (78,5\% Ni) & Ferromagnetyk & 100 000 \\
        Mumetal (75\% Ni, 5\% Cu, 2\% Cr) & Ferromagnetyk & 100 000 \\
        Oczyszczone żelazo (0,05\% zanieczyszczeń) & Ferromagnetyk & 200 000 \\
        Supermaloj (5\% Mo, 79\% Ni) & Ferromagnetyk & 1 000 000 \\
        \bottomrule
    \end{tabular}

    \vspace{0.2cm}
    \hfill\small źródło: Electromagnetics Explained, \cite{schmitt2002electromagnetics}, Electromagnetics with Applications \cite[p.~201]{kraus1999electromagnetics} 
\end{table}


\section{Sondy}

Podstawowy wzór na indukcyjność cewki soleidalnej:

\begin{equation}
    L = \frac{\mu N^2 A}{l}
    \label{eq:indukcyjnosc_cewki}
\end{equation}
gdzie $\mu$ oznacza przenikalność materiału użytego do wykonania cewki, N oznacza ilość wykonanych zwojów, A oznacza pole przekroju poprzecznego, l długość

Wykrywacze metalu z reguły nie korzystają z tego rodzaju sondy.

\begin{figure}[H]
    \begin{center}
        \includegraphics[width=0.95\textwidth]{./img/multiturn_cewka.png}
    \end{center}
    \caption{Przekrój poprzeczny cewki}\label{fig:poprz}
\end{figure}


\begin{equation}
    L = \frac{0,8 (N A^2)}{6a + 9b + 10c}
    \label{eq:indukcyjnosc_dobra}
\end{equation}

\section{Analiza porównawcza architektur detektorów}

\subsection{Problemy różnych rodzajów wykrywaczy}
Różne technologie wykrywaczy metali posiadają swoje mocne, jak i słabe strony.
W niniejszym rozdziale zostanie przeprowadzona analiza porównawcza popularnych architektur detektorów oraz omówione ich własności takie jak:

\begin{itemize}
    \item Balans gruntowy
    \item Głębokość penetracji
    \item Szerokość penetracji
    \item Odporność na mineralizację gruntu
    \item Umiejętność dokładnego wskazywania położenia poszukiwanego obiektu
    \item Umiejętność dyskryminacji metali
\end{itemize}

\subsection{BFO - Beat Frequency Oscillator}

Technologia wykrywaczy metali oparta o generator zdudnieniowy (ang. \textit{Beat Frequency Oscillator}).
Podstawowy system BFO opiera się o dwa oscylatory, będące dostrojone do jak najbliższej wspólnej częstotliwości.
Do generowania żądanej częstotliwości wykorzystywane są przeważnie obwody rezonansowe LC.
Pierwszy z oscylatorów, referencyjny, korzysta z cewki o zadanej indukcyjności.
Oscylator szukający, używa sondy wykrywacza metali jako swoje źródło indukcyjności. \cite{moreland_bfo_theory}

\begin{figure}[H]
    \begin{center}
\begin{tikzpicture}[
    mixer/.style={
        draw, 
        circle, 
        minimum size=1.5cm, 
        align=center, 
        font=\sffamily
    },
    label_text/.style={
        font=\large\itshape
    }
]

    \node[blok] (search_osc) at (0, 2.5) {Oscylator\\szukający};
    \node[blok] (ref_osc) at (0, -2.5) {Oscylator\\referencyjny};
    \node[mixer] (mixer) at (4, 0) {Mieszacz};
    \node[blok] (lpf) at (8, 0) {Filtr\\dolnoprzepustowy};

    \coordinate (output) at (11, 0);


    \draw[thick] ($(search_osc.west) + (-1.2, 0)$) coordinate (coil_center) circle (0.5cm);
    \draw[thick] (coil_center) circle (0.35cm);
    \draw[thick] ($(coil_center) + (0.35, 0.1)$) -- ($(search_osc.west) + (0, 0.1)$);
    \draw[thick] ($(coil_center) + (0.35, -0.1)$) -- ($(search_osc.west) + (0, -0.1)$);
    \draw[thick, decoration={coil, aspect=0.4, segment length=2mm, amplitude=2mm, pre length=1mm, post length=1mm}, decorate] 
        ($(ref_osc.west) + (-1.5, 0.4)$) -- ($(ref_osc.west) + (-1.5, -0.4)$);
    \draw[thick] ($(ref_osc.west) + (-1.5, 0.4)$) -- ($(ref_osc.west) + (0, 0.4)$);
    \draw[thick] ($(ref_osc.west) + (-1.5, -0.4)$) -- ($(ref_osc.west) + (0, -0.4)$);


    \draw[arrow] (search_osc.east) -| (mixer.north) 
        node[midway, left, yshift=10pt, label_text] {$f_{search}$};
    \draw[arrow] (ref_osc.east) -| (mixer.south) 
        node[midway, left, yshift=-10pt, label_text] {$f_{ref}$};
    \draw[arrow] (mixer.east) -- (lpf.west) 
        node[midway, above, label_text] {$f_s+f_r$}
        node[midway, below, label_text] {$f_s-f_r$};
    \draw[arrow] (lpf.east) -- (output) 
        node[right, label_text] {$f_s-f_r$};

\end{tikzpicture}
    \end{center}
    \caption{Schemat podstawowego systemu BFO}\label{fig:bfo}
\end{figure}

Kluczowym procesem zachodzącym w wykrywaczach wykorzystujących BFO jest mieszanie sygnałów. 
Generowane przez układy rezonansowe sygnały wprowadzane są do mieszacza częstotliwości, który wyznacza sumę oraz różnicę obu częstotliwości.

Przy założeniu częstotliwości generowanych przez oscylatory opisanych wzorami Thomsona:
\begin{equation}
    f_{ref} = \frac{1}{2\pi\sqrt{LC}}  
    \label{eq:bfo_f_ref}
\end{equation}
gdzie $L$ oraz $C$ to parametry odpowiednio indukcyjności oraz pojemności oscylatorów

\begin{equation}
    f_{search} = \frac{1}{2\pi\sqrt{(L_0 \pm \Delta L)C}}
    \label{eq:bfo_f_search}
\end{equation}
gdzie $L_0$ to indukcyjność własna cewki (idealnie $L$), $\Delta L$ to jej zmiana wywołana przez obiekt

Przy założeniu dwóch sygnałów sinusoidalnych:
\begin{equation}
    V_s(t) = A_s \cos(2\pi f_{search} t)
    \label{eq:bfo_v_s_t}
\end{equation}
gdzie $V_s(t)$ jest sygnałem sterującym cewką szukającą, $A_s$ amplidudą napięcia.

\begin{equation}
    V_r(t) = A_r \cos(2\pi f_{ref} t)
    \label{eq:bfo_v_r_t}
\end{equation}
gdzie $V_r(t)$ jest sygnałem referencyjnym, $A_r$ amplidudą napięcia.

Wykonywane jest mieszanie sygnałów \ref{eq:bfo_v_s_t} oraz \ref{eq:bfo_v_r_t}, co idealnie oznacza ich mnożenie opisywanym przez tożsamość trygonometryczną:
\begin{equation}
    \cos(\alpha)\cos(\beta) = \frac{1}{2} [\cos(\alpha - \beta) + \cos(\alpha + \beta)]
    \label{eq:mnozenie_cosinusow}
\end{equation}
Produkt sygnałów opisuje równanie:
\begin{equation}
    V_{mix}(t) \propto \frac{A_s A_r}{2} [\cos(2\pi(f_{search} - f_{ref})t) + \cos(2\pi(f_{search} + f_{ref})t)]
    \label{eq:produkt_bfo}
\end{equation}
gdzie $V_{mix}(t)$ to sygnał wychodzący z mieszacza.

Wymieszany sygnał zawiera dwie składowe:

\begin{itemize}
    \item sumę częstotliwości, którą należy odfiltrować (około kilkuset kHz)
    \item różnicę częstotliwości, będącą słyszalnym dudnieniem (około kilku kHz)
\end{itemize}
Wymieszany sygnał jest następnie traktowany filtrem dolnoprzepustowym, usuwającym wysoką składową częstotliwości.


Ze względu na prostotę wykrywacze stosujące technologię BFO rzadko są stosowane w warunkach profesjonalnych.
Urządzenia oparte o generator zdudnieniowy wymagają częstej rekalibracji, szczególnie w wysokozmineralizowanych glebach \cite{moreland_bfo_theory}.


\subsection{IB - Induction Balance}

\begin{figure}[H]
    \begin{center}
    \begin{tikzpicture}
        \node[blok] (nadawczy) at (0,0) {Układ\\nadawczy};
        \node[blok] (odbiorczy) at (7,0) {Układ\\odbiorczy};

        \coordinate (middle) at ($(nadawczy.east)!0.5!(odbiorczy.west)$);
        \draw[connector] ($(middle)+(-0.6,0)$) circle (0.8cm);
        \draw[connector] ($(middle)+(0.6,0)$) circle (0.8cm);

        \draw[connector] (nadawczy.east) -- ($(middle)+(-0.6-0.8, 0)$);
        \draw[connector] (odbiorczy.west) -- ($(middle)+(0.6+0.8, 0)$);
    \end{tikzpicture}
    \end{center}
    \caption{Schemat działania wykrywacza balansu indukcyjnego}\label{fig:ib_schemat}
\end{figure}


Oryginalnie przedstawiona przed doktora Davida Edwarda Hughesa \cite{doi:10.1098/rspl.1879.0012} w 1879 technologia z powodzeniem używana współcześnie.
Sonda zbudowana jest z dwóch obwodów - nadawczego oraz odbiorczego. Cewka nadawcza sterowana jest stabilnym sygnałem sinusoidalnym o niskiej częstotliwości (typowo 5-20kHz) \cite{Mazurek_Wdowiak_2015}.
Z tego powodu technologia ta jest znana również pod nazwą VLF (ang. Very Low Frequency).

Wytwarzane przez cewkę nadawczą zmienne pole elektromagnetyczne odbierane jest przez cewkę odbiorczą.
W stanie neutralnym (brak przewodników w pobliżu) kostrukcja sondy powoduje zminimalizowanie sygnału powrotnego.
Umieszczenie w zasięgu działania sondy obiektu mającego wpływ na generowane pole elektromagnetyczne powoduje zakłócenie równowagi.
Rodzaj przewodnika, a także wielkość oraz odległość od obiektu ma wpływ na amplitudę oraz fazę sygnału odbieranego przez cewkę odbiorczą, co pozwala na przeprowadzenie dyskryminacji rodzaju metalu, z jakiego jest stworzony badany obiekt.

\subsection{PI - Pulse Induction}

W przeciwieństwie do wcześniej opisywanych, technologia indukcji impulsowej (ang. \textit{Pulse Induction}) operuje w dziedzinie czasu. 
Sonda najczęściej zbudowana z pojedynczej cewki typu monoloop pełniąca rolę nadawczej oraz odbiorczej.
Wykrywacz generuje impuls prądu wytwarzający pole magnetyczne penetrujące przestrzeń. Po gwałtownym odcięciu przepływu prądu, pole zanika, co indukuje prądy wirowe w badanym obiekcie \cite{pi_performance}.
Układ odbiorczy mierzy oraz porównuje zmianę prędkości zaniku sygnału powracającego oraz jego kształt \cite{Mazurek_Wdowiak_2015} \cite{moreland2024inside}.

Pasożytnicza pojemność cewki $C_p$ (wraz z pojemnością kabla) tworzy z indukcyjnością $L$ obwód rezonansowy, który ma tendencję do oscylacji. Oscylacje te maskują słabe sygnały pochodzące z badanych obiektów w fazie odbioru. Aby wygasić oscylacje w minimalnym czasie bez przeregulowania, konieczne jest zastosowanie rezystora tłumiącego $R_d$ włączonego równolegle do cewki.

Wartość dla tłumienia krytycznego wynosi:
\begin{equation}
    R_{crit} = \frac{1}{2} \sqrt{\frac{L}{C_p}}
    \label{eq:pi_crit}
\end{equation}

