\documentclass{pginz}

%  Czasami brakuje opisu z nieco wyższego poziomu. 
% Np w opisie konstrukcji przydałby się schemat blokowy, pokazujący źródło zasilania, mikroprocesor, moduł nadawczy, moduł odbiorczy, cewkę i jakie całość jest połączona.
% Później można opisywać poszczególne moduły (jak ma to miejsce teraz) z odniesieniem się do wcześniejszego schematu. Komuś nie zaznajomionego z tematem może być ciężko wyobrazić sobie całość z wysokiego poziomu.
% Podobnie przydałoby się schemat ogólny procedury np: nadanie impulsu, odbiór impulsu, przetwarzanie sygnałów, detekcja obiektu, wyświetlenie informacji.

%%%%%%%%%Miejsce na dodatkowe pakiety%%%%%%%%%%%%%
\usepackage{subcaption}

\begin{document}
\includepdf[pages={1}]{StronaTytulowa_193682.pdf}
%\includepdf[pages={2}]{Oswiadczenie.pdf}
\setcounter{page}{3}

\chapter*{Streszczenie}

Tematem niniejszej pracy dyplomowej jest projekt oraz realizacja prototypu wykrywacza metali działającego w technologii impulsowej (ang. \textit{Pulse Induction}).
Część teoretyczna zawiera rys historyczny rozwoju metod detekcji, począwszy od odkryć w XIX wieku, kończąc na współczesnych rozwiązaniach.
Dokonano analizy zjawisk fizycznych, takich jak prawo indukcji Faradaya, czy prądy wirowe, które stanowią fundament działania urządzenia.
Przeprowadzono także przegląd istniejących popularnych architektur wykrywaczy metali oraz typów sond, uzasadniając wybór technologii.
W części praktycznej szczegółowo opisano proces budowy prototypu opartego na mikrokontrolerze \textit{STM32H523CE} z rdzeniem \textit{ARM Cortex M-33}.
Wytłumaczono przebieg pracy utworzonego oprogramowania realizującego akwizycję danych w czasie rzeczywistym oraz detekcję anomalii sygnałowych poprzez analizę czasu odpowiedzi oraz nachylenia spadku.
Projekt zakłada wykonanie funkcjonalnego urządzenia mobilnego, wraz z wyprodukowaną płytką drukowaną oraz sondą zamknięte w zaprojektowanej obudowanie wykonanej w technologii druku 3D.

\bigskip

\noindent\textbf{Słowa kluczowe:} wykrywacz metali, indukcja impulsowa, cyfrowe przetwarzanie sygnałów, systemy wbudowane

\bigskip

\noindent\textbf{Dziedzina nauki i techniki zgodna z OECD} Nauki inżynieryjne i techniczne, Elektrotechnika, elektronika i inżynieria informatyczna, Robotyka i Automatyka

\chapter*{Abstract}

The subject of this thesis is the design and implementation of a metal detector prototype operating in Pulse Induction technology. The theoretical part contains a historical outline of the development of detection methods, starting from discoveries in the 19th century and ending with modern solutions. An analysis of physical phenomena, such as Faraday's law of induction or eddy currents, which constitute the foundation of the device's operation, was performed. A review of existing popular metal detector architectures and coil types was also conducted, justifying the choice of technology. In the practical part, the process of building a prototype based on the STM32H523CE microcontroller with an ARM Cortex M-33 core is described in detail. The operation of the created software implementing real-time data acquisition and signal anomaly detection through the analysis of response time and decay slope is explained. The project assumes the creation of a functional mobile device, along with a manufactured printed circuit board and a probe enclosed in a designed housing made using 3D printing technology.

\bigskip

\noindent\textbf{Keywords:} metal detector, pulse induction, digital signal processing, embedded systems

\bigskip

\noindent\textbf{OECD consistent field of science and technology classification:} Engineering and technology, Electrical engineering, electronic engineering, information engineering, Robotics and automatic control


\tableofcontents
\addcontentsline{toc}{chapter}{Spis treści}

\chapter*{Lista symboli}

\begin{itemize}[noitemsep,topsep=0pt,parsep=0pt,partopsep=0pt,labelwidth=1cm,align=left,itemindent=0pt]
\item[$R$] - Rezystancja
\item[$L$] - Indukcyjność
\item[$B$] - Wektor indukcji magnetycznej
\item[$E$] - Wektor natężenia pola elektrycznego
\item[$\Phi_{B}$] - Strumień magnetyczny
\item[$\mathcal{E}$] - Siła elektromotoryczna
\item[$\mu$] - Przenikalność magnetyczna
\item[$\sigma$] - Konduktywność (przewodność właściwa)
\item[$J$] - Gęstość prądu elektrycznego
\item[$\delta$] - Głębokość naskórkowości
\item[$f$] - Częstotliwość
\item[$I$] - Natężenie prądu
\item[$C$] - Pojemność elektryczna
\item[$C_{p}$] - Pojemność pasożytnicza cewki
\item[$V_{cc}$] - Napięcie zasilania
\item[$\alpha$] - Współczynnik wygładzania w filtrze EMA
\end{itemize}

\chapter*{Lista skrótów}

\begin{itemize}[noitemsep,topsep=0pt,parsep=0pt,partopsep=0pt,labelwidth=1cm,align=left,itemindent=0pt]
\item[MSPS] - Miliony próbek na sekundę (ang. \textit{Mega samples per second})
\item[DMA] - Bezpośredni dostęp do pamięci (ang. \textit{Direct Memory Access})
\item[IB] - Balans indukcyjny (ang. \textit{Induction Balance})
\item[PI] - Indukcja Impulsowa (ang. \textit{Pulse Induction})
\item[BFO] - Generator zdudnieniowy (ang. \textit{Beat Frequency Oscillator})
\item[VLF] - Bardzo mała częstotliwość (ang. \textit{Very Low Frequency})
\item[CAD] - Projektowanie wspomagane komputerowo (ang. \textit{Computer Aided Design})
\item[SP] - Metoda potencjału własnego (ang. \textit{Self-Potential})
\item[LC] - Obwód rezonansowy cewka-kondensator (ang. \textit{Inductor-Capacitor})
\item[DD] - Cewka typu Double-D
\item[MOSFET] - Tranzystor polowy z izolowaną bramką (ang. \textit{Metal-Oxide Semiconductor Field-Effect Transistor})
\item[PWM] - Modulacja szerokości impulsów (ang. \textit{Pulse Width Modulation})
\item[ADC] - Przetwornik analogowo-cyfrowy (ang. \textit{Analog-to-Digital Converter})
\item[FFT] - Szybka Transformata Fouriera (ang. \textit{Fast Fourier Transform})
\item[EMA] - Wykładnicza średnia krocząca (ang. \textit{Exponential Moving Average})
\item[IIR] - Filtr o nieskończonej odpowiedzi impulsowej (ang. \textit{Infinite Impulse Response})
\item[PCB] - Płytka drukowana (ang. \textit{Printed Circuit Board})
\item[USART] - Uniwersalny synchroniczny asynchroniczny nadajnik-odbiornik (ang. \textit{Universal Synchronous Asynchronous Receiver Transmitter})
\item[SWD] - Szeregowy interfejs debugowania (ang. \textit{Serial Wire Debug})
\item[TTL] - Logika tranzystor-tranzystor (ang. \textit{Transistor-Transistor Logic})
\end{itemize}


\chapter{Wstęp}
% tam można opisać rys historyczny, kto i kiedy jako pierwszy skonstruował tego typu urządzenie, do czego ono jest wykorzystywane i dlaczego jest ważne.
\section{Rys historyczny}
Historia wykrywaczy metali sięga połowy XIX wieku, kiedy w 1830 roku brytyjski geolog i wynalazca Robert Were Fox odkrył, że różne złoża metali wykazują różnicę potencjału elektrycznego. Na podstawie m.in. tej wiedzy skontruował pierwsze udokumentowane urządzenie, które było wykorzystywane w celu wykrywania metali w kopalniach. Eksperyment Foxa, przeprowadzony w kopalniach Kornwalii, opierał się na zasadzie, że żyły rud metalu otoczone wodą słoną i gliną tworzą naturalne ogniwa. W swojej metodzie nazwanej później metodą potencjału własnego (self-potential, SP) R.W. Fox umieszczał miedziane płytki w żyłach, łącząc je miedzianym przewodem z galwanometrem – urządzeniem do pomiaru natężenia prądu elektrycznego. Ruch wskaźnika przyrządu wskazywał na obecność minerałów \cite{fox}.

Przełom w technologii wykrywania metalu nastąpił w roku 1841. Pruski fizyk Heinrich Wilhelm Dove opublikował wtedy swój wynalazek zwany induktorem różnicowym (ang. differential inductor). Urządzenie to składało się z czterech cewek umieszczonych na dwóch szklanych rurkach, przy czym każda rurka owinięta była dwiema miedzianymi cewkami. Naładowane butelki lejdejskie (ówczesna forma kondensatorów wysokiego napięcia) były rozładowywane przez dwie cewki pierwotne, a powstały impuls prądowy indukował napięcie w cewkach wtórnych. Kluczowa dla działania urządzenia była konfiguracja, w której cewki wtórne połączono w opozycji - indukowane napięcia wzajemnie się znosiły, co profesor Dove weryfikował osobiście, trzymając końce cewek wtórnych. Gdy kawałek metalu został umieszczony wewnątrz jednej ze szklanych rurek, profesor Dove otrzymał wstrząs elektryczny. Ta obserwacja stanowiła fundamentalną zasadę działania pierwszego magnetycznego detektora metalu wykorzystującego indukcję magnetyczną, a jednocześnie pierwszego impulsowego detektora metalu w historii \cite{enwiki:1247591717}\cite{Dove1841}.

Niedługo później wykrywacz metalu został wykorzystany w służbie ratowania życia - Alexander Graham Bell wykorzystał technologię bilansu indukcyjnego (ang. Induction Balance) podczas próby zlokalizowania pocisku w rannym prezydencie Stanów Zjednoczonych Jamesie Garfieldzie. Motywacją naukowca było zastąpienie wymaganego dotychczas inwazyjnego nacinania pacjentów w celu poszukiwania pozostałego ciała obcego. Wykorzystanie detektora typu bilansu indukcyjnego nie przyniosło jednak  \cite{Bell1883Upon}
\todo[inline]{dokonczyc}

Na uwagę zasługuje polski wykrywacz min. W 1937 roku Departament Artylerii polskiego Ministerstwa Obrony Narodowej zlecił budowę urządzenia umożliwiającego zlokalizować niewybuchy pozostawione na poligonach. Za projekt odpowiadała AVA Wytwórnia Radiotechniczna \cite{enwiki:1253721450}. W 1939 roku po inwazji Niemiec na Polskę prace zostały wstrzymane. Po ucieczce do Wielkiej Brytanii porucznik Józef Kosacki dokończył swoje urządzenie i podarował technologię Brytyjskim Siłom Zbrojnym, za co Król Jerzy VI podziękował listownie. Wykrywacze "Mine Detector (Polish) Mark I" (pol. wykrywacz min) były wykorzystywane między innymi podczas II bitwy pod El Alamein, gdzie dwukrotnie przyspieszyły proces odminowywania terenów i uratowały tysiące żyć \cite{10824231}.

\begin{figure}[H]
    \centering
    \includegraphics[width = 7.5cm]{img/saperzy.jpg}
    \caption{Saperzy Korpusu Inżynierów Królewskich używający wykrywacza min, 28 sierpień 1942}
    \hfill\small źródło: iwm.org.uk \cite{saperzy_jpg} 
\end{figure}

\section{Zastosowanie wykrywaczy metali we współczesnym świecie}
W dzisiejszych czasach wykrywacze metali posiadają niezwykle wszechstronne zastosowanie, idące daleko poza wymienione wcześniej cele wykopaliskowe, militarne lub medyczne.

W sektorze bezpieczeństwa publicznego wykrywacze stanowią witalny element zabezpieczenia lotnisk oraz innych miejsc masowych zgromadzeń ludzi. Systemy bramek detektujących obecność metalu pozwalają na szybkie i nieinwazyjne zapewnienie bezpieczeństwa, chroniąc przed wnoszeniem niebezpiecznych przedmiotów, takich jak ostre narzędzia, czy broń palna. Ręczne detektory metali, stosowane przez funkcjonariuszy publicznych jak i ochronę pozwalają na przefiltrowanie osób przed wejściem na imprezy masowe.

W przemyśle spożywczym, chemicznym oraz farmaceutycznym wykrywacze metali stanowią podstawowy element systemu kontroli jakości. Pozwalają wykryć potencjalne zanieczyszczenia w produktach oraz półfabrykatach, zapobiegając trafieniu do konsumentów produktów niebezpiecznych dla zdrowia człowieka.

Wykrywacze metalu znajdują również zastosowanie w archeologii oraz badaniach naukowych. Archeolodzy posługują się tymi instrumentami do odkrywania ukrytych w ziemi artefaktów oraz zabytków, takich jak biżuteria, monety, narzędzia, czy broń, bez konieczności przeprowadzania destrukcyjnych prac wykopaliskowych.

\section{Cel pracy}

Praca inżynierska ma na celu zbadanie 
\todo[inline]{dokonczyc}

\chapter{Przegląd literatury}
W niniejszym przeglądzie przyjęto założenia upraszczające szeroko stosowane w literaturze naukowej dotyczącej podstaw detekcji metali, pozwalających jednocześnie na uzyskanie rozwiązań o wysokiej wartości podczas rozważań inżynierskich \cite[p.~308]{griffiths1999introduction}:


\begin{itemize}
    \item Quasi-statyczna aproksymacja,
    \item Badany obiekt jest traktowany jako jednorodna sfera o zadanej konduktywności $\sigma$, \\przenikalności magnetycznej $\mu$ i promieniu $R$,
    \item Badany obiekt nie porusza się ($\mathbf{v}=0$).
\end{itemize}

\section{Elektrodynamika systemów wykrywania metali}

Podstawą funkcjonowania wszystkich indukcyjnych wykrywaczy metali jest prawo indukcji Faradaya.
Zjawisko to wiąże w czasie pole magnetyczne z indukowanym wirowym polem elektrycznym \cite[p.~302]{griffiths1999introduction}.
Zgodnie z równaniem
\begin{equation}
\label{eq:prawo_faradaya}
\nabla \times \mathbf{E} = - \frac{\partial \mathbf{B}}{\partial t}
\end{equation}
gdzie $\mathbf{E}$ oznacza wektor natężenia, a $\mathbf{B}$ oznacza wektor indukcji magnetycznej.

\begin{figure}[htbp]
  \centering
  \includesvg[width=0.3\textwidth]{./img/Alternator_1.svg}
  \caption{Schemat alternatora przedstawiający obracający się magnes (wirnik) i nieruchome uzwojenie drutu (stator) oraz napięcie wytwarzane, gdy obracające się pole magnetyczne indukuje prąd w przewodzie.}
    \hfill\small źródło: wikipedia.org \cite{alternator_svg} 
\end{figure}

W kontekście inżynierii systemów detekcji cewka nadawcza zasilana prądem zmiennym lub impulsowym generuje zmienny strumień magnetyczny. Strumień ten penetruje przestrzeń, w tym potencjalne obiekty przewodzące oraz grunt. Zgodnie z przytoczonym prawem, zmienność strumienia w czasie ($\partial \mathbf{B}/\partial t$) indukuje siłę elektromotoryczną w każdym przewodniku objętym polem.

Kluczowym mechanizmem, na którym opiera się detekcja, jest powstawanie prądów wirowych (ang. eddy currents). Prądy te, płynąc w zamkniętych pętlach wewnątrz materiału przewodzącego (w płaszczyznach prostopadłych do linii pola magnetycznego), generują własne, wtórne pole magnetyczne \cite{kriezis}. Istotę tego zjawiska wyjaśnia reguła Lenza, która stanowi, że kierunek indukowanego prądu jest taki, iż wytworzone przez niego pole magnetyczne przeciwdziała zmianie pola pierwotnego, które je wywołało \cite{kriezis}.

\begin{equation}
    \label{eq:prawo_lenza}
    \mathcal{E} = -\frac{d \Phi_\mathbf{B}}{d t}
\end{equation}

To wzajemne oddziaływanie pól – pierwotnego (wymuszającego) i wtórnego (reakcyjnego) – prowadzi do mierzalnych zmian impedancji wzajemnej układu cewek lub zmian napięcia indukowanego w cewce odbiorczej, co stanowi sygnał użyteczny dla wykrywacza metalu.


\section{Dyfuzja magnetyczna}

Wykrywacze metalu z wbudowaną funkcjonalnością dyskryminacji metalu pozwalają na selektywne ofiltrowanie sygnałów od niepożądanych obiektów.
Działają na podstawie właściwości materiału badanego:
\begin{itemize}
    \item przenikalność magnetyczna
    \item przewodność właściwa
\end{itemize}
Umożliwia to użytkownikowi urządzenia natychmiastowe i nieinwazyjne ustalenie rodzaju wykrytego metalu.
Zwiększa to efektywność poszukiwań, pozwalając na wczesne odrzucenie sygnału w przypadku podejrzenia wykrycia niepożądanego obiektu.

\subsection{Wyprowadzenie wzoru}

Wyprowadzenie wzoru na dyfuzję magnetyczną przy założeniach określonych na wstępie do rozdziału można rozpocząć od wyznaczenia wzoru na gęstość prądu z równania Ampère'a-Maxwella:
\begin{equation}
    \label{eq:ampere}
    \nabla \times \mathbf{B} = \mu \mathbf{J}
\end{equation}
\begin{equation}
    \label{eq:gestosc_pradu}
    \mathbf{J} = \frac{1}{\mu} (\nabla \times \mathbf{B})
\end{equation}
gdzie $\mathbf{B}$ jest polem magnetycznym, $\mu$ jest przenikalnością magnetyczną materiału, $\mathbf{J}$ jest gęstością prądu elektrycznego.
\\Wzór na gęstość prądu wynikający z prawa Ohma:
\begin{equation}    
    \label{eq:ohma}
    \mathbf{J} = \sigma \mathbf{E}
\end{equation}
gdzie $\sigma$ oznacza konduktywność prądu, $\mathbf{E}$ oznacza pole elektryczne

Łącząc wzór \ref{eq:gestosc_pradu} ze wzorem na gęstość prądu wynikającą z prawa Ohma (\ref {eq:ohma}, otrzymano wzór na pole elektryczne wyrażone przez pole magnetyczne:
\begin{equation}
     \mathbf{E} = \frac{1}{\mu \sigma} (\nabla \times \mathbf{B})
     \label{eq:elektryczne_przez_magnetyczne}
\end{equation}
Wiedząc, że gradient pola elektycznego jest równy ujemnej zmianie pola magnetycznego (równanie Maxwella-Faradaya, \ref{eq:prawo_faradaya})
\begin{equation}
\nabla \times \mathbf{E} = - \frac{\partial \mathbf{B}}{\partial t}
      \tag{\ref{eq:prawo_faradaya}}
\end{equation}
podstawiono $\mathbf{E}$ ze wzoru \ref{eq:elektryczne_przez_magnetyczne}:
\begin{equation}
    \nabla \times \left( \frac{1}{\mu \sigma} (\nabla \times \mathbf{B}) \right) = - \frac{\partial \mathbf{B}}{\partial t}
    \label{eq:test}
\end{equation}
Korzystając z tożsamości wektorowej

$$\nabla \times (\nabla \times \mathbf{B}) = \nabla(\nabla \cdot \mathbf{B}) - \nabla^2 \mathbf{B}$$
otrzymano:
\begin{equation}
    - \frac{1}{\mu \sigma} \nabla^2 \mathbf{B} = - \frac{\partial \mathbf{B}}{\partial t}
    \label{eq:diffused_dirty}
\end{equation}
Po uporządkowaniu otrzymano równanie dyfuzji:
\begin{equation}
    \label{eq:diffused}
    \nabla^2 \mathbf{B} = \mu \sigma \frac{\partial \mathbf{B}}{\partial t}
\end{equation}

gdzie $\mu$ to absolutna przenikalność magnetyczna materiału, a $\sigma$ to jego konduktywność (przewodność właściwa). Równanie to wskazuje, że interakcja pola z metalem nie jest natychmiastowa, lecz podlega procesom relaksacyjnym zależnym od właściwości materiałowych obiektu oraz częstotliwości fali pobudzającej. Prowadzi to do zjawiska naskórkowości (ang. skin effect), które ogranicza penetrację prądów wirowych do warstw powierzchniowych przy wyższych częstotliwościach, co ma fundamentalne znaczenie dla identyfikacji obiektów.

\subsection{Zjawisko naskórkowości}

\begin{table}[H]
    \centering
    \caption{Rezystywność wybranych materiałów dla 1 atm w 20$^\circ$C.}
    \begin{tabular}{llll}
        \toprule
        \textbf{Materiał} & \textbf{Rezystywność ($\Omega \cdot m$)} & \textbf{Materiał} & \textbf{Rezystywność ($\Omega \cdot m$)} \\
    \midrule
        \textit{Przewodniki:} & & \textit{Materiały słabo przewodzące:} & \\
        Srebro & $1,59 \times 10^{-8}$ & Woda morska & $0,2$ \\
        Miedź & $1,68 \times 10^{-8}$ & German & $0,46$ \\
        Złoto & $2,21 \times 10^{-8}$ & Diament & $2,7$ \\
        Aluminium & $2,65 \times 10^{-8}$ & Krzem & $2500$ \\
        Żelazo & $9,61 \times 10^{-8}$ & \textit{Izolatory:} & \\
        Rtęć & $9,61 \times 10^{-7}$ & Woda (czysta) & $8,3 \times 10^3$ \\
        Nichrom & $1,08 \times 10^{-6}$ & Szkło & $10^9 - 10^{14}$ \\
        Mangan & $1,44 \times 10^{-6}$ & Guma & $10^{13} - 10^{15}$ \\
        Grafit & $1,6 \times 10^{-5}$ & Teflon & $10^{22} - 10^{24}$ \\
    \bottomrule
    \end{tabular}
    \vspace{0.2cm} \\
    \hfill\small źródło: Handbook of Chemistry and Physics \cite{haynes2010crc} 
\end{table}
\begin{table}[H]
    \centering
    \caption{Maksymalna przenikalność magnetyczna wybranych materiałów dla niskich częstotliwości}
    \begin{tabular}{l l l}
        \toprule
        \textbf{Substancja} & \textbf{Typ grupy} & \textbf{Względna przenikalność } $\mu_r$ \\
        \midrule
        Bizmut & Diamagnetyk & 0,99983 \\
        Srebro & Diamagnetyk & 0,99998 \\
        Ołów & Diamagnetyk & 0,99983 \\
        Miedź & Diamagnetyk & 0,99991 \\
        Woda & Diamagnetyk & 0,99991 \\
        \addlinespace
        Próżnia & Niemagnetyczny & 1 \\
        \addlinespace
        Powietrze & Paramagnetyk & 1,0000004 \\
        Aluminium & Paramagnetyk & 1,00002 \\
        Pallad & Paramagnetyk & 1,0008 \\
        \addlinespace
        Kobalt & Ferromagnetyk & 250 \\
        Nikiel & Ferromagnetyk & 600 \\
        Ferroxcube 3 (proszek ferrytowy Mn-Zn) & Ferromagnetyk & 1 500 \\
        Stal miękka (0,2\% C) & Ferromagnetyk & 2 000 \\
        Żelazo (0,2\% zanieczyszczeń) & Ferromagnetyk & 5 000 \\
        \addlinespace
        Żelazo krzemowe (4\% Si) & Ferromagnetyk & 7 000 \\
        Permalloy 78 (78,5\% Ni) & Ferromagnetyk & 100 000 \\
        Mumetal (75\% Ni, 5\% Cu, 2\% Cr) & Ferromagnetyk & 100 000 \\
        Oczyszczone żelazo (0,05\% zanieczyszczeń) & Ferromagnetyk & 200 000 \\
        Supermaloj (5\% Mo, 79\% Ni) & Ferromagnetyk & 1 000 000 \\
        \bottomrule
    \end{tabular}

    \vspace{0.2cm}
    \hfill\small źródło: Electromagnetics Explained, \cite{schmitt2002electromagnetics}, Electromagnetics with Applications \cite[p.~201]{kraus1999electromagnetics} 
\end{table}


\section{Sondy}

Podstawowy wzór na indukcyjność cewki soleidalnej:

\begin{equation}
    L = \frac{\mu N^2 A}{l}
    \label{eq:indukcyjnosc_cewki}
\end{equation}
gdzie $\mu$ oznacza przenikalność materiału użytego do wykonania cewki, N oznacza ilość wykonanych zwojów, A oznacza pole przekroju poprzecznego, l długość

Wykrywacze metalu z reguły nie korzystają z tego rodzaju sondy.

\begin{figure}[H]
    \begin{center}
        \includegraphics[width=0.95\textwidth]{./img/multiturn_cewka.png}
    \end{center}
    \caption{Przekrój poprzeczny cewki}\label{fig:poprz}
\end{figure}


\begin{equation}
    L = \frac{0,8 (N A^2)}{6a + 9b + 10c}
    \label{eq:indukcyjnosc_dobra}
\end{equation}

\section{Analiza porównawcza architektur detektorów}

\subsection{Problemy różnych rodzajów wykrywaczy}
Różne technologie wykrywaczy metali posiadają swoje mocne, jak i słabe strony.
W niniejszym rozdziale zostanie przeprowadzona analiza porównawcza popularnych architektur detektorów oraz omówione ich własności takie jak:

\begin{itemize}
    \item Balans gruntowy
    \item Głębokość penetracji
    \item Szerokość penetracji
    \item Odporność na mineralizację gruntu
    \item Umiejętność dokładnego wskazywania położenia poszukiwanego obiektu
    \item Umiejętność dyskryminacji metali
\end{itemize}

\subsection{BFO - Beat Frequency Oscillator}

Technologia wykrywaczy metali oparta o generator zdudnieniowy (ang. \textit{Beat Frequency Oscillator}).
Podstawowy system BFO opiera się o dwa oscylatory, będące dostrojone do jak najbliższej wspólnej częstotliwości.
Do generowania żądanej częstotliwości wykorzystywane są przeważnie obwody rezonansowe LC.
Pierwszy z oscylatorów, referencyjny, korzysta z cewki o zadanej indukcyjności.
Oscylator szukający, używa sondy wykrywacza metali jako swoje źródło indukcyjności. \cite{moreland_bfo_theory}

\begin{figure}[H]
    \begin{center}
\begin{tikzpicture}[
    mixer/.style={
        draw, 
        circle, 
        minimum size=1.5cm, 
        align=center, 
        font=\sffamily
    },
    label_text/.style={
        font=\large\itshape
    }
]

    \node[blok] (search_osc) at (0, 2.5) {Oscylator\\szukający};
    \node[blok] (ref_osc) at (0, -2.5) {Oscylator\\referencyjny};
    \node[mixer] (mixer) at (4, 0) {Mieszacz};
    \node[blok] (lpf) at (8, 0) {Filtr\\dolnoprzepustowy};

    \coordinate (output) at (11, 0);


    \draw[thick] ($(search_osc.west) + (-1.2, 0)$) coordinate (coil_center) circle (0.5cm);
    \draw[thick] (coil_center) circle (0.35cm);
    \draw[thick] ($(coil_center) + (0.35, 0.1)$) -- ($(search_osc.west) + (0, 0.1)$);
    \draw[thick] ($(coil_center) + (0.35, -0.1)$) -- ($(search_osc.west) + (0, -0.1)$);
    \draw[thick, decoration={coil, aspect=0.4, segment length=2mm, amplitude=2mm, pre length=1mm, post length=1mm}, decorate] 
        ($(ref_osc.west) + (-1.5, 0.4)$) -- ($(ref_osc.west) + (-1.5, -0.4)$);
    \draw[thick] ($(ref_osc.west) + (-1.5, 0.4)$) -- ($(ref_osc.west) + (0, 0.4)$);
    \draw[thick] ($(ref_osc.west) + (-1.5, -0.4)$) -- ($(ref_osc.west) + (0, -0.4)$);


    \draw[arrow] (search_osc.east) -| (mixer.north) 
        node[midway, left, yshift=10pt, label_text] {$f_{search}$};
    \draw[arrow] (ref_osc.east) -| (mixer.south) 
        node[midway, left, yshift=-10pt, label_text] {$f_{ref}$};
    \draw[arrow] (mixer.east) -- (lpf.west) 
        node[midway, above, label_text] {$f_s+f_r$}
        node[midway, below, label_text] {$f_s-f_r$};
    \draw[arrow] (lpf.east) -- (output) 
        node[right, label_text] {$f_s-f_r$};

\end{tikzpicture}
    \end{center}
    \caption{Schemat podstawowego systemu BFO}\label{fig:bfo}
\end{figure}

Kluczowym procesem zachodzącym w wykrywaczach wykorzystujących BFO jest mieszanie sygnałów. 
Generowane przez układy rezonansowe sygnały wprowadzane są do mieszacza częstotliwości, który wyznacza sumę oraz różnicę obu częstotliwości.

Przy założeniu częstotliwości generowanych przez oscylatory opisanych wzorami Thomsona:
\begin{equation}
    f_{ref} = \frac{1}{2\pi\sqrt{LC}}  
    \label{eq:bfo_f_ref}
\end{equation}
gdzie $L$ oraz $C$ to parametry odpowiednio indukcyjności oraz pojemności oscylatorów

\begin{equation}
    f_{search} = \frac{1}{2\pi\sqrt{(L_0 \pm \Delta L)C}}
    \label{eq:bfo_f_search}
\end{equation}
gdzie $L_0$ to indukcyjność własna cewki (idealnie $L$), $\Delta L$ to jej zmiana wywołana przez obiekt

Przy założeniu dwóch sygnałów sinusoidalnych:
\begin{equation}
    V_s(t) = A_s \cos(2\pi f_{search} t)
    \label{eq:bfo_v_s_t}
\end{equation}
gdzie $V_s(t)$ jest sygnałem sterującym cewką szukającą, $A_s$ amplidudą napięcia.

\begin{equation}
    V_r(t) = A_r \cos(2\pi f_{ref} t)
    \label{eq:bfo_v_r_t}
\end{equation}
gdzie $V_r(t)$ jest sygnałem referencyjnym, $A_r$ amplidudą napięcia.

Wykonywane jest mieszanie sygnałów \ref{eq:bfo_v_s_t} oraz \ref{eq:bfo_v_r_t}, co idealnie oznacza ich mnożenie opisywanym przez tożsamość trygonometryczną:
\begin{equation}
    \cos(\alpha)\cos(\beta) = \frac{1}{2} [\cos(\alpha - \beta) + \cos(\alpha + \beta)]
    \label{eq:mnozenie_cosinusow}
\end{equation}
Produkt sygnałów opisuje równanie:
\begin{equation}
    V_{mix}(t) \propto \frac{A_s A_r}{2} [\cos(2\pi(f_{search} - f_{ref})t) + \cos(2\pi(f_{search} + f_{ref})t)]
    \label{eq:produkt_bfo}
\end{equation}
gdzie $V_{mix}(t)$ to sygnał wychodzący z mieszacza.

Wymieszany sygnał zawiera dwie składowe:

\begin{itemize}
    \item sumę częstotliwości, którą należy odfiltrować (około kilkuset kHz)
    \item różnicę częstotliwości, będącą słyszalnym dudnieniem (około kilku kHz)
\end{itemize}
Wymieszany sygnał jest następnie traktowany filtrem dolnoprzepustowym, usuwającym wysoką składową częstotliwości.


Ze względu na prostotę wykrywacze stosujące technologię BFO rzadko są stosowane w warunkach profesjonalnych.
Urządzenia oparte o generator zdudnieniowy wymagają częstej rekalibracji, szczególnie w wysokozmineralizowanych glebach \cite{moreland_bfo_theory}.


\subsection{IB - Induction Balance}

\begin{figure}[H]
    \begin{center}
    \begin{tikzpicture}
        \node[blok] (nadawczy) at (0,0) {Układ\\nadawczy};
        \node[blok] (odbiorczy) at (7,0) {Układ\\odbiorczy};

        \coordinate (middle) at ($(nadawczy.east)!0.5!(odbiorczy.west)$);
        \draw[connector] ($(middle)+(-0.6,0)$) circle (0.8cm);
        \draw[connector] ($(middle)+(0.6,0)$) circle (0.8cm);

        \draw[connector] (nadawczy.east) -- ($(middle)+(-0.6-0.8, 0)$);
        \draw[connector] (odbiorczy.west) -- ($(middle)+(0.6+0.8, 0)$);
    \end{tikzpicture}
    \end{center}
    \caption{Schemat działania wykrywacza balansu indukcyjnego}\label{fig:ib_schemat}
\end{figure}


Oryginalnie przedstawiona przed doktora Davida Edwarda Hughesa \cite{doi:10.1098/rspl.1879.0012} w 1879 technologia z powodzeniem używana współcześnie.
Sonda zbudowana jest z dwóch obwodów - nadawczego oraz odbiorczego. Cewka nadawcza sterowana jest stabilnym sygnałem sinusoidalnym o niskiej częstotliwości (typowo 5-20kHz) \cite{Mazurek_Wdowiak_2015}.
Z tego powodu technologia ta jest znana również pod nazwą VLF (ang. Very Low Frequency).

Wytwarzane przez cewkę nadawczą zmienne pole elektromagnetyczne odbierane jest przez cewkę odbiorczą.
W stanie neutralnym (brak przewodników w pobliżu) kostrukcja sondy powoduje zminimalizowanie sygnału powrotnego.
Umieszczenie w zasięgu działania sondy obiektu mającego wpływ na generowane pole elektromagnetyczne powoduje zakłócenie równowagi.
Rodzaj przewodnika, a także wielkość oraz odległość od obiektu ma wpływ na amplitudę oraz fazę sygnału odbieranego przez cewkę odbiorczą, co pozwala na przeprowadzenie dyskryminacji rodzaju metalu, z jakiego jest stworzony badany obiekt.

\subsection{PI - Pulse Induction}

W przeciwieństwie do wcześniej opisywanych, technologia indukcji impulsowej (ang. \textit{Pulse Induction}) operuje w dziedzinie czasu. 
Sonda najczęściej zbudowana z pojedynczej cewki typu monoloop pełniąca rolę nadawczej oraz odbiorczej.
Wykrywacz generuje impuls prądu wytwarzający pole magnetyczne penetrujące przestrzeń. Po gwałtownym odcięciu przepływu prądu, pole zanika, co indukuje prądy wirowe w badanym obiekcie \cite{pi_performance}.
Układ odbiorczy mierzy oraz porównuje zmianę prędkości zaniku sygnału powracającego oraz jego kształt \cite{Mazurek_Wdowiak_2015} \cite{moreland2024inside}.

Pasożytnicza pojemność cewki $C_p$ (wraz z pojemnością kabla) tworzy z indukcyjnością $L$ obwód rezonansowy, który ma tendencję do oscylacji. Oscylacje te maskują słabe sygnały pochodzące z badanych obiektów w fazie odbioru. Aby wygasić oscylacje w minimalnym czasie bez przeregulowania, konieczne jest zastosowanie rezystora tłumiącego $R_d$ włączonego równolegle do cewki.

Wartość dla tłumienia krytycznego wynosi:
\begin{equation}
    R_{crit} = \frac{1}{2} \sqrt{\frac{L}{C_p}}
    \label{eq:pi_crit}
\end{equation}


\chapter{Budowa prototypu wykrywacza metalu}
\begin{figure}[H]
\centering
    \begin{tikzpicture}[
        node distance=1.5cm and 2cm,
        probe/.style = {
            base,
            trapezium,
            trapezium left angle=70,
            trapezium right angle=70,
        },
        label_text/.style = {
            font=\sffamily\footnotesize\bfseries,
            fill=white,
            inner sep=2pt
        }
    ]
        \node[blok] (user) {Panel\\użytkownika};
        \node[blok, below=1.5cm of user] (digital) {Sekcja\\cyfrowa};
        \node[blok, below=2.5cm of digital, xshift=-3cm] (rx) {Sekcja\\odbiorcza};
        \node[blok, below=2.5cm of digital, xshift=3cm] (tx) {Sekcja\\nadawcza};
        \node[blok, below=2cm of $(rx.south)!0.5!(tx.south)$] (sonda) {Sonda};

        \draw[arrow] ([xshift=-3mm]user.south) -- ([xshift=-3mm]digital.north);
        \draw[arrow] ([xshift=3mm]digital.north) -- ([xshift=3mm]user.south);
        \draw[arrow] ([xshift=3mm]digital.south) -- node[label_text, pos=0.6] {PULSE\_OUT} (tx.north);
        \draw[arrow] (tx.south) -- ([xshift=3mm]sonda.north); 
        \draw[arrow] ([xshift=-3mm]sonda.north) -- (rx.south);
        \draw[arrow] (rx.north) -- node[label_text, pos=0.4] {PULSE\_IN} ([xshift=-3mm]digital.south);

    \end{tikzpicture}
    \caption{Schemat blokowy działania prototypu wykrywacza}
    \label{sch:budowa}
\end{figure}
Prototyp został wykonany z wykorzystaniem wiedzy zebranej w poprzednich rozdziałach.
Urządzenie zostało zbudowane w oparciu o łatwo dostępne komponenty oraz wydruk 3D.
Architektura wykonanego urządzenia podzielona jest na 5 części ukazanych na rysunku \ref{sch:budowa}.
W niniejszym rozdziale zostaną omówione zastosowane rozwiązania tworzące poszczególne sekcje oraz technikalia ich działania.

\section{Zasilanie}

Zasilanie zbudowanego urządzenia odbywa się przez dostępne z lewej strony gniazdo DC 5,5mm o polaryzacji dodatniej bolca.
Wspierany przez układ zakres napięcia wynosi od $+9V$ to $+12V$ DC. W celu tłumienia zakłóceń spowodowanych przez nagłe skoki przepływu prądu w momencie sterowania sondą detekcyjną zastosowano kondensatory o pojemności $1500\mu F$ i maksymalnemu napięciu $16V$. Do zasilania sekcji cyfrowej zastosowano stabilizator liniowy $LF33CV$ obniżający napięcie do 3,3V bez wprowadzania zakłóceń \cite{lf33cv_ds} o wysokiej częstotliwości. 

\section{Sekcja nadawcza}

\begin{figure}[H]
    \centering
    \includegraphics[width = 12cm]{img/budowa/sch/Sekcja_nadawcza.png}
    \caption{Schemat sekcji sterującej tranzystorem MOSFET przy użyciu sygnału \textit{PULSE\_IN}}
\end{figure}

Wygenerowany przez układ sterujący sekcją cyfrową sygnał prostokątny o napięciu $3,3V$ wzmacniany jest przez dwa tranzystory bipolarne $Q_2$, a następnie $Q_1$ połączone szeregowo pozwalające na uzyskanie wystarczającego napięcia do wysterowania tranzystora MOSFET IRF740 $Q_3$. Zastosowana konfiguracja tranzystorów bipolarnych pozwala na zminimalizowanie natężenia prądu wymaganego do wysterowania sekcji. Jest to szczególnie istotne biorąc pod uwagę fakt, że maksymalne natężenie oferowane przez zastosowany mikrokontroler wynosi $20mA$.
    
\begin{figure}[H]
    \centering
    \includegraphics[width = 7.5cm, height = 7.5cm]{img/output_characteristics.png}
    \caption{Charakterystyka wyjściowa tranzystora IRF740}
\end{figure}

\subsection{Pętla cewki}
\begin{figure}[H]
    \centering
    \includegraphics[width = 12cm]{img/budowa/sch/Pętla_cewki.png}
    \caption{Pętla rozładowująca cewkę}
\end{figure}

Tranzystor $Q_3$ zwiera sondę do masy powodując przepływ prądu przez cewkę podłączoną do \textit{COIL1} i \textit{COIL2}. Po otwarciu tranzystora $Q_3$ na cewce indukowane jest napięcie, którego górnym limitem jest przez napięcie breakdown $V_{brk}$ tranzystora wynoszące minimum $400V$ \cite{irf740}. W celu umożliwienia rozładowania napięcia w cewce zastosowano dwie szybkie diody typu \textit{BAV19} połączone równolegle odwrotnie do siebie pozwalające na przepływ prądu dla napięcia powyżej $0,7V$. Efekt jest opóźniony prez zastosowanie rezystora $R_6$. Dla końcowego etapu rozładowywania cewki, czyli kiedy napięcie jest poniżej progu przewodzenia diody rezystor $R5$ wraz z trymerem rezystorowym $RV1$ umożliwia przepływ prądu. Potencjometr zastosowany jest, aby łatwo dostroić tłumienie cewki w sposób inżynierski (zamiast wyznaczania wartości rezystora tłumiącego przy pomocy równania \ref{eq:pi_crit}). Pożądane jest krytyczne tłumienie odpowiedzi sondy pozwalające na najdokładniejsze pomiary zmian. Na rysunku 2.4 narysowano porównanie odpowiedzi sondy nietłumionej oraz krytycznie tłumionej. Zademonstrowana odpowiedź jest mierzona za kondensatorem $C_1$.

\begin{figure}[H]
    \centering
    \includegraphics[width=0.8\textwidth]{img/cewka_odp.png}
    \caption{Wykres porównawczy odpowiedzi cewki}
\end{figure}


\section{Sonda detekcyjna}
Sonda detekcyjna wykonana została z miedzianego przewodu o przekroju 0.4mm. Nawinięto 31 zwojów o średnicy 15cm.
\todo[inline]{wyznaczyc teorytyczna indukcyjnosc}

\begin{figure}
    \begin{center}
        \includegraphics[width=0.6\textwidth]{img/donut otwarty kat.png}
    \end{center}
    \caption{Komputerowe przedstawienie obudowy sondy}\label{fig:donut_otwarty}
\end{figure}


\section{Sekcja odbiorcza}


\begin{figure}[H]
    \begin{center}
        \includegraphics[width=0.95\textwidth]{img/wzmocnienie.png}
    \end{center}
    \caption{Schemat sekcji przetwarzającej sygnał odebrany przez sondę}
\end{figure}
Uzyskana odpowiedź z cewki po przejściu przez kondensator usuwający offset DC podciągana jest do napięcia $VGND$ za pomocą rezystora $R_7$. Następnie odpowiedź jest mnożona 1000-krotnie w celu uwydatnienia ostatniej części odpowiedzi sygnału najbardziej podatnej na działanie powstałego pola magnetycznego. Uzyskane napięcie może posiadać wartość zbliżoną do napięcia zasilania $V_{cc}$, dlatego w celu dostosowania go do poziomu pozwalającego obsłużyć go przez zastosowany w wykrywaczu MCU zastosowano dzielnik napięcia wykonujący mnożenie razy $0,33$. Tak wyznaczony sygnał nie jest wystarczająco silny, dlatego przechodzi on przez kopiujący wzmacniacz $U_1$. Warto zaznaczyć, że wzmacniacz MCP6281 jest typu rail-to-rail, pozwalający na operowanie na sygnałach bliskich napięciom zasilania $V_+$, jak i $V_-$ wzmacniacza \cite{mcp6281_ds}. Poprzednie sekcje nie wymagały tej specjalnej właściwości. Tak przygotowany sygnał jest gotowy do analizy przez mikrokontroler. Dla dodatkowego bezpieczeństwa zastosowano diodę zenera $3,3V$ gwarantującą, że \textit{SIGNAL\_OUT} należy do zakresu napięć obsługiwanego przez sekcję cyfrową.


\subsection{Wirtualna masa}
\begin{figure}[H]
    \begin{center}
    \includegraphics[width=7.5cm]{img/vgnd.png}
    \end{center}
    \caption{Konfiguracja wzmacniacza operacyjnego generującego napięcie $VGND$}
\end{figure}

W celu uzyskania pełnej odpowiedzi cewki zastosowana jest sztuczna masa wynosząca $0,5$ napięcia zasilania. Dzięki niej wzmacniacze operacyjne w sekcji wzmocnienia odpowiedzi mogą wzmocnić odpowiedź pracując wyłącznie z dodatnim napięciem, co omija konieczność generowania ujemnego napięcia upraszczając finalny schemat wykrywacza metalu. Wykonana jest za pomocą dzielnika napięcia wykonującego działanie

\begin{equation}
    U_{VGND}=Vcc\cdot\frac{R9}{R8+R9}=Vcc\cdot\frac{1}{2}  
    \label{eq:vgnd}
\end{equation}
którego wynikowe napięcie przechodzi przez wzmacniacz operacyjny $U3$ kopiujący napięcie wejściowe wzmacniając sygnał, jednocześnie odciążając zastosowane w dzielniku rezystory.



\section{Sekcja cyfrowa}

\begin{figure}[H]
    \begin{center}
        \includegraphics[width=7.5cm]{img/stm32.png}
    \end{center}
    \caption{Cyfrowe przedstawienie układu STM32F411RE wraz z użytymi pinami}\label{fig:stm32}
\end{figure}


Część cyfrowa wykrywacza metalu oparta jest o moduł STM32H523CECoreBoard \cite{WeActStudio_STM32H523CoreBoard} zawierający układ \textbf{STM32H523CE}. Ten produkowany przez ST Microelectronics mikrokontroler posiada jeden 32-bitowy rdzeń oparty na architekturze ARM\textregistered{} Cortex-M33\textregistered{} \cite{st_stm32h5_web} \cite{st_stm32h533_ds} o maksymalnym taktowaniu 250Mhz oraz m.in wykorzystane w projekcie układy peryferyjne:

\subsection{Liczniki sprzętowe}
\label{subsec:timery}
Wykorzystywane są dwa 16-bitowe liczniki sprzętowe napędzane za pomocą zegara \textit{APB2} skonfigurowanego na taktowanie z częstotliwością $F_{timer}=250Mhz$. Częstotliwość przepełnienia licznika sprzętowego można wyznaczyć za pomocą poniższego równania: 

\begin{equation}
F_{output}=\frac{F_{timer}}{(PSC+1)(ARR+1)}
\end{equation}

gdzie PSC oznacza Prescaler, natomiast ARR oznacza Auto Reload Register


Licznik \textit{TIM10} pełni funkcję maszyny stanów zarządzając aktualnym stanem sygnału \textit{PULSE\_OUT}.
W celu zachowania dokładności przy jednoczesnym jak najmniejszym obciążeniu MCU licznik wywołuje przerwania w częstotliwości $F_{output}=1\mu s$ poprzez konfigurację $PSC=0, ARR=249$.

Podczas pracy układu cyfrowego w funkcji przerwania sprawdzana jest wartość zmiennej inkrementowanej co każde wywołanie przerwania. Jeżeli licznik jest równy 0, wystawiany jest sygnał +3.3V na pin 1 (PULSE\_OUT) układu, powodując otwarcie tranzystora MOSFET. Następnie w pożądanym momencie sygnał wyjściowy jest wygaszony, co pozwala na dynamiczną konfigurację długości impulsu wysyłanego do cewki.

\begin{figure}[H]
    \begin{center}
\begin{tikzpicture}[node distance=1cm and 2.5cm]

    \node[startstop] (start) {Wywołanie przerwania};
    
    \node[decision, below=1cm of start] (dec1) {Czy \texttt{pulseTickCtr == 0}?};
    \node[decision, below=of dec1] (dec2) {Czy \texttt{pulseTickCtr} \\ \texttt{== PULSE\_WIDTH}?};
    \node[decision, below=of dec2] (dec3) {Czy \texttt{pulseTickCtr} \\ \texttt{== PULSE\_WIDTH} \\ \texttt{+ 10us}?};
    \node[decision, below=of dec3] (dec4) {Czy \texttt{pulseTickCtr} \\ \texttt{== TOTAL\_WIDTH}?};
    
    \node[process, below=of dec4] (inc) {Inkrementuj \texttt{pulseTickCtr}};
    \node[startstop, below=of inc] (stop) {Koniec};

    \node[process, right=of dec1] (proc1) {Wystaw sygnał\\ \textbf{PULSE\_OUT}};
    \node[process] (proc2) at (proc1 |- dec2) {Wygaś sygnał\\ \textbf{PULSE\_OUT}};
    \node[process] (proc3) at (proc1 |- dec3) {Zapisz licznik DMA do\\ \textbf{dmaBufferHead}};
    \node[process] (proc4) at (proc1 |- dec4) {Ustaw \\ \texttt{pulseTickCtr = 0}};
    
    \draw[arrow] (start) -- (dec1);
    \draw[arrow] (dec1) -- node[left] {Nie} (dec2);
    \draw[arrow] (dec2) -- node[left] {Nie} (dec3);
    \draw[arrow] (dec3) -- node[left] {Nie} (dec4);
    \draw[arrow] (dec4) -- node[left] {Nie} (inc);
    \draw[arrow] (inc) -- node[left] {} (stop);
    
    \draw[arrow] (dec1) -- node[above] {Tak} (proc1);
    \draw[arrow] (dec2) -- node[above] {Tak} (proc2);
    \draw[arrow] (dec3) -- node[above] {Tak} (proc3);
    \draw[arrow] (dec4) -- node[above] {Tak} (proc4);

    \draw[arrow] (proc1) |- ($(dec1.south)!0.5!(dec2.north)$);
    \draw[arrow] (proc2) |- ($(dec2.south)!0.5!(dec3.north)$);
    \draw[arrow] (proc3) |- ($(dec3.south)!0.5!(dec4.north)$);
    \draw[arrow] (proc4) |- (stop);

\end{tikzpicture}
    \end{center}
    \caption{Procedura przerwania TIM10}
\end{figure}


Drugi licznik \textit{TIM2} wykorzystany jest do generowania sygnału PWM o wypełnieniu 50\% i dynamicznie wyznaczanej częstotliwości do brzęczyka wskazującego o wykryciu anomalii w przetworzonym sygnale \textit{PULSE\_IN}.

\subsection{Konwerter cyfrowo-analogowy}

ADC napędzany jest sygnałem zegarowym \textit{HCLK} o częstotliwości $F_{ADC}=250Mhz$, z preskalerem o wartości $4$, co daje końcową częstotliwość podukładu wynoszącą $62,5Mhz$.
Maksymalna rozdzielczość pomiaru wynosi 12 bitów, co daje dokładność na poziomie $0,8 mV$.

\begin{equation}
R= \frac{Vref}{2^N - 1} =\frac{3.3V}{4095}\approx 0,8mV
\end{equation}

gdzie $V_{ref}$  to napięcie referencyjne (tutaj $V_{in}=3.3V$), N = ilość bitów

W celu osiągnięcia jak największego zasięgu wykrywacza wymagana jest jednocześnie jak największa dokładność układu ADC oraz jego szybkość działania. Dla trybu 12-bitowego każdy pomiar zajmuje $C_{sample}=15$ cykli zegarowych dających teorytyczny limit przy ustalonym wcześniej taktowaniu zegara ADC wynoszący $4,16 MSPS$ (milionów próbek na sekundę ,ang. \textit{Mega Sample Per Second}), co daje jedną próbkę napięcia co $~0,24\mu s$.

\begin{equation}
    SPS = \frac{F_{ADC}}{C_{sample}}=\frac{62,5\cdot10^{6}}{15}\approx{4,16\cdot10^6}
\end{equation}

Dla maksymalnego odciążenia mikrokontrolera wykorzystany został tryb DMA z wyłączonym generowaniem przerwań sprzętowych. Każda nowa próbka zmierzona przez konwerter analogowo-cyfrowy zapisywana jest w do cyklicznego buforu znajdującym się pamięci SRAM wskazanym podczas konfiguracji bez dodatkowego udziału MCU.
Podczas wykonywania algorytmu wykrywania kopiowany jest pożądany indeks próbki z bufora i dokonywany jest proces analizy sygnału.

\section{Algorytm wykrywania anomalii}

Główny algorytm napędzający prototyp wykrywacza metali podzielony jest na X części.
\subsection{Odszukanie obszaru zainteresowania}

Pierwsza część odpowiada za wyszukanie obszaru zainteresowania. Ze względu na 

\begin{figure}[H]
    \begin{center}
\begin{tikzpicture}[node distance=1.5cm and 2cm]

    % Węzły
    \node[startstop] (start) {Start};
    \node[decision, below=of start] (dec1) {Czy wszystkie \\ próbki?};
    \node[decision, below=of dec1] (dec2) {Czy próbka \\ $\ge$ \texttt{THRES}?};
    
    % "Następna próbka" po lewej stronie (zgodnie z rysunkiem)
    \node[process, left=of dec2] (next) {Następna \\ próbka};
    
    % Akcja po znalezieniu progu
    \node[process, below=of dec2] (set_area) {Ustaw \\ \texttt{AREA\_START} \\ na indeks};
    
    \node[startstop, below=of set_area] (stop) {Koniec};

    % Połączenia
    \draw[arrow] (start) -- (dec1);
    
    % Decyzja 1: Czy wszystkie sprawdzone?
    \draw[arrow] (dec1) -- node[right] {Nie} (dec2);
    % Tak -> Koniec (ścieżka bokiem, omijająca resztę)
    \draw[arrow] (dec1.east) -- node[above] {Tak} ++(2,0) |- (stop);

    % Decyzja 2: Czy przekroczono próg?
    \draw[arrow] (dec2) -- node[right] {Tak} (set_area);
    \draw[arrow] (set_area) -- (stop);

    % Pętla "Nie" (powrót do sprawdzania kolejnej próbki)
    \draw[arrow] (dec2) -- node[above] {Nie} (next);
    \draw[arrow] (next) |- (dec1);

\end{tikzpicture}
    \end{center}
    \caption{}\label{fig:}
\end{figure}


\begin{figure}[H]
    \centering
\begin{tikzpicture}[node distance=1cm and 2.5cm]

    % Węzły główne (pionowa oś)
    \node[startstop] (start) {Start};
    
    \node[process, below=of start] (fetch) {Pobierz próbkę \\ (z ostatnich 10us)};
    
    \node[decision, below=of fetch] (dec1) {Czy próbka \\ > \texttt{AREA\_THRESHOLD}?};
    
    \node[decision, below=of dec1] (dec2) {Czy poprzednia \\ > \texttt{MIN\_SAMPLE}?};
    
    \node[process, below=of dec2] (filter) {Dodaj do \\ \texttt{1ST\_STAGE\_SAMPLES} \\ \textit{Oblicz} \texttt{1ST\_STAGE\_AVG}};
    
    \node[decision, below=of filter] (dec3) {Czy próbka \\ > \texttt{1ST\_STAGE\_AVG}?};
    
    \node[startstop, below=of dec3] (stop) {Koniec};

    % Węzły boczne (akcje warunkowe)
    \node[process, right=of dec2] (proc_min) {Przypisz wartość do \\ \texttt{MIN\_SAMPLE}};
    
    \node[process, right=of dec3] (proc_alarm) {Anomalia: \\ Ustal brzęczyk};

    % Połączenia - Główna ścieżka i logika Tak/Nie
    \draw[arrow] (start) -- (fetch);
    \draw[arrow] (fetch) -- (dec1);

    % Decyzja 1: Threshold
    % Nie: Idź do następnej próbki (pętla powrotna do pobierania)
    \draw[arrow] (dec1.west) -- node[above] {Nie} ++(-1,0) |- (fetch.west);
    % Tak: Szukanie najniższej próbki
    \draw[arrow] (dec1) -- node[left] {Tak} (dec2);

    % Decyzja 2: Min Sample
    % Tak: Ignoruj (przejdź od razu do filtracji)
    \draw[arrow] (dec2) -- node[left] {Tak} (filter);
    % Nie: Aktualizuj MIN_SAMPLE
    \draw[arrow] (dec2) -- node[above] {Nie} (proc_min);
    
    % Powrót z aktualizacji MIN_SAMPLE do głównego nurtu (przed filtracją)
    \draw[arrow] (proc_min) |- ($(dec2.south)!0.5!(filter.north)$);

    % Połączenie do Decyzji 3
    \draw[arrow] (filter) -- (dec3);

    % Decyzja 3: Anomalia
    % Nie: Koniec
    \draw[arrow] (dec3) -- node[left] {Nie} (stop);
    % Tak: Anomalia
    \draw[arrow] (dec3) -- node[above] {Tak} (proc_alarm);
    
    % Powrót z obsługi anomalii do Końca
    \draw[arrow] (proc_alarm) |- (stop);

\end{tikzpicture}
    % \begin{tikzpicture}
    %     \node[startstop] (start) {Start};
    %     \node[process, below=of start] (analiza) {Analiza zawartości bufora cyklicznego};
    %     \node[process, below=of analiza] (wyznaczenie) {Wyznaczenie próbki o najmniejszej wartości w obszarze zainteresowania};
    %     \node[blok, below=of wyznaczenie] (suma4) {Suma\\czterech próbek};
    %     \node[blok, below=of suma4] (suma32) {Suma 32\\próbek};
    %     \node[blok, below=of suma32] (tlo) {- background};
    %     \node[blok, below=of tlo] (anomalia) {sprawdzenie\\anomalii};
    %     \node[blok, below=of anomalia] (buzzer) {buzzer};
    %     \node[startstop, below=of buzzer] (koniec) {Koniec};
    %     \draw[arrow] (start) -- (analiza);
    %     \draw[arrow] (analiza) -- (wyznaczenie);
    %     \draw[arrow] (wyznaczenie) -- (suma4);
    %     \draw[arrow] (suma4) -- (suma32);
    %     \draw[arrow] (suma32) -- (tlo);
    %     \draw[arrow] (tlo) -- (anomalia);
    %     \draw[arrow] (anomalia) -- (buzzer);
    %     \draw[arrow] (buzzer) -- (koniec);
    % \end{tikzpicture}
    \caption{Schemat blokowy działania algorytmu}\label{fig:diagram}
\end{figure}

Algorytm wykrywania metalu oparty jest o próbkowanie sygnału \textit{PULSE\_OUT} w dynamicznie określanym odstępie czasowym. Zbierana jest suma czterech następnych próbek, która zapisywana jest do bufora cyklicznego przechowującego 32 pozycje (Stage 1). Średnia obliczona ze wszystkich wartości znajdujących się w buforze etatu 1 umieszczana jest w drugim buforze cyklicznym, mający pojemność 64 próbki. Ten bufor służy do eliminacji tła. Średnia wartości etapu pierwszego porównywana jest ze średnią wartości znajdujących się w buforze tła, każda różnica co najmniej 5 jednostek traktowana jest jako anomalia i oznacza wykrycie metalu.

\begin{figure}[H]
    \begin{center}
        \includegraphics[width=0.95\textwidth]{../meas/pulse_out.png}
    \end{center}
    \caption{Wykres sygnału \textit{PULSE\_OUT} używany w algorytmie}\label{fig:meas}

\end{figure}

\section{Płytka drukowana}
Płytka drukowana (PCB, ang. \textit{Printed Circuit Board}) prototypu została zaprojektowana w otwartoźródłowym programie typu CAD (projektowanie wspomagane komputerowo, ang. \textit{Computer Aided Design}) KiCad\cite{kicad}.
Finalny produkt został wykonany z wykorzystaniem laminatu dwustronnego o wymiarach $80mm$ x $101mm$.
W celu minimalizacji zakłóceń elektromagnetycznych, wolne przestrzenie na obu warstwach PCB zostały wypełnione wylewką miedzi (ang. \textit{copper pour}) podłączoną do potencjału masy.

\begin{figure}[H]
\centering
  \includegraphics[width=0.5\textwidth]{./pcb.pdf}
  \caption{Schemat projektu PCB (z pominiętą płaszczyzną masy)}
\end{figure}

Projekt zawiera enkoder, który umożliwia interakcję z użytkownikiem poprzez przekręcanie oraz wciskanie wystającego wału oraz dwa porty przygotowane pod moduły wyświetlaczy ILI9341 oraz SSD1306.



\begin{figure}
\centering
  \includegraphics[width=0.7\textwidth]{./img/pcb_3drender.png}
  \caption{Trójwymiarowe komputerowe przedstawienie projektu PCB}
\end{figure}

\section{Obudowa}
Obudowa wykrywacza metalu została wykonana w technologii druku 3D z wykorzystaniem plastiku PET-G gwarantującym większą wytrzymałość mechaniczną w porównaniu
do alternatyw.
Głównym założeniem projektowym było stworzenie kompaktowej konstrukcji. Wymiary zewnętrzne obudowy wynoszą odpowiednio $85mm$ x $126mm$ x $28mm$.
Projekt posiada dedykowane otwory na port sondy RJ45 oraz gniazdo zasilające DC.
Od góry wykonane zostało wycięcie na wyświetlacz SSD1306 oraz wał enkodera.
Całość zamykana jest poprzez cztery śruby M3, wkręcane we wkładki metalowe montowane na gorąco.
Obudowa PCB, jak i sondy została zaprojektowana w oprogramowaniu CAD 3D Autodesk Fusion.

\begin{figure}
    \begin{center}
        \includegraphics[width=\textwidth]{img/pudlo otwarte.png}
    \end{center}
    \caption{Trójwymiarowe przedstawienie PCB zamkniętego w zaprojektowanej obudowie}
\end{figure}


\chapter{Testy i analiza skuteczności}


\listoffigures
\addcontentsline{toc}{chapter}{Spis rysunków}
\listoftables
\addcontentsline{toc}{chapter}{Spis tabel}

\printbibliography

\begin{appendices}
    \appendix
    \appendix
\chapter{Warstwy PCB utworzonej płytki drukowanej}
\begin{figure}[htbp]
  \centering
  \includesvg{./img/pcb/GitKop-F_Cu.svg}
  \caption{Warstwa F\_Cu}
\end{figure}

\begin{figure}[htbp]
  \centering
  \includesvg{./img/pcb/GitKop-F_Mask.svg}
  \caption{Warstwa F\_Mask}
\end{figure}

\begin{figure}[htbp]
  \centering
  \includesvg{./img/pcb/GitKop-F_Silkscreen.svg}
  \caption{Warstwa F\_Silkscreen}
\end{figure}

\begin{figure}[htbp]
 \centering
 \includesvg{./img/pcb/GitKop-B_Cu.svg}
 \caption{Warstwa B\_Cu}
\end{figure}

\begin{figure}[htbp]
  \centering
  \includesvg{./img/pcb/GitKop-B_Mask.svg}
  \caption{Warstwa B\_Mask}
\end{figure}

% \begin{figure}[htbp]
% \centering
%   \includesvg{./img/pcb/GitKop-B_Silkscreen.svg}
%   \caption{Warstwa B\_Silkscreen}
% \end{figure}

    \chapter{Schemat gotowego wykrywacza metalu}
    \begin{figure}[H]
        \includegraphics[angle=90, width=1.0\textwidth]{Schematic.pdf}
    \end{figure}
\end{appendices}

\end{document}
