\documentclass[10pt]{beamer}

\usepackage[utf8]{inputenc}
\usepackage[T1]{fontenc}
\RequirePackage[main=polish,english, provide=*]{babel}
\usepackage{graphicx}
\usepackage{booktabs}
\usepackage{svg}
\usepackage{tikz}
\usetikzlibrary{positioning, shapes.geometric, arrows.meta, positioning, decorations.pathmorphing, calc}

\usetheme{default}
\usecolortheme{dolphin}

\title[Wykrywacz Metalu]{Wykrywacz metalu}
\subtitle{Projekt i realizacja prototypu w technologii impulsowej}
\author{Paweł Reich}
\institute[Politechnika Gdańska]{
  Wydział Elektroniki, Telekomunikacji i Informatyki\\
  Kierunek: Automatyka, Cybernetyka i Robotyka\\
  \vspace{0.5cm}
  Opiekun pracy: dr inż. Kamil Stawiarski
}
\date{2025}

\begin{document}

\tikzset{
    base/.style = {
        font=\sffamily\footnotesize,
        align=center,
        thick
    },
    blok/.style={
        draw,
        rectangle,
        thick,
        rounded corners=3pt,
        align=center,
        minimum width=3.5cm,
        minimum height=1.2cm,
        inner sep=2mm
    },
    startstop/.style = {
        base,
        rectangle,
        rounded corners,
        draw=black!80!black,
        minimum width=3cm,
        minimum height=1cm
    },
    decision/.style = {
        base,
        diamond,
        aspect=2.5, % Rozciąga romb w poziomie
        draw=black!80!black,
        inner sep=0pt,
        text width=3.5cm % Zawijanie tekstu wewnątrz
    },
    process/.style = {
        base,
        rectangle,
        draw=black!80!black,
        minimum width=3.5cm,
        minimum height=1cm
    },
    connector/.style = {
        thick,
        draw=gray!80!black
    },
    arrow/.style = {
        connector,
        ->,
        >=stealth,
    },
}
% Slajd tytułowy
\begin{frame}
  \titlepage
\end{frame}

% Spis treści
\begin{frame}{Plan prezentacji}
  \tableofcontents
\end{frame}

% --- SEKCJA 1: WSTĘP ---
\section{Wstęp i cel pracy}

\begin{frame}{Cel pracy}
  \begin{block}{Główny cel}
    Celem pracy było wykonanie porównania popularnych architektur wykrywaczy metali oraz zbudowanie prototypu wybranego rodzaju przy użyciu łatwo dostępnych komponentów.
  \end{block}

  \begin{itemize}
    \item Analiza zjawisk fizycznych (indukcja elektromagnetyczna, prądy wirowe)
    \item Przegląd istniejących technologii (BFO, IB, PI)
    \item Projekt obwodów analogowych i cyfrowych
    \item Implementacja algorytmów na mikrokontrolerze STM32
  \end{itemize}
\end{frame}

% --- SEKCJA 2: TEORIA ---
\section{Podstawy teoretyczne}

\begin{frame}{Elektrodynamika}
  Podstawą działania urządzenia jest \textbf{Prawo Indukcji Faradaya} oraz \textbf{Prądy Wirowe}.

  \begin{columns}
    \column{0.6\textwidth}
    \begin{itemize}
      \item Zmienne pole magnetyczne cewki przenika grunt
      \item W obiekcie metalowym indukują się prądy wirowe
      \item Zgodnie z regułą Lenza, prądy te wytwarzają przeciwne pole magnetyczne
      \item Istotnym czynnikiem jest \textbf{zjawisko naskórkowości} -- ogranicza ono penetrację przy wysokich częstotliwościach
    \end{itemize}
    
    \column{0.4\textwidth}
    \begin{center}
        \begin{figure}
            \begin{center}
                \includegraphics[width=0.6\textwidth]{img/moreland_induction_eddy.png} 
            \end{center}
        \end{figure}
    \end{center}
  \end{columns}
\end{frame}

% --- SEKCJA 3: PRZEGLĄD ROZWIĄZAŃ ---
\section{Analiza architektur}

\begin{frame}{Porównanie technologii detekcji}
  W pracy przeanalizowano trzy główne architektury :
  
  \begin{enumerate}
    \item \textbf{BFO (Beat Frequency Oscillator):}
      \begin{itemize}
        \item Prosta konstrukcja oparta na dudnieniu częstotliwości
        \item Wada: Niska stabilność, wymaga częstej rekalibracji
      \end{itemize}
    \item \textbf{IB (Induction Balance / VLF):}
      \begin{itemize}
        \item Układ cewek nadawczych i odbiorczych w równowadze
        \item Dobra dyskryminacja metali
      \end{itemize}
    \item \textbf{PI (Pulse Induction) -- Wybrana do projektu:}
      \begin{itemize}
        \item Działanie w dziedzinie czasu (impulsy prądowe)
        \item Jedna cewka (Monoloop)
        \item Wysoka odporność na mineralizację gruntu
        \item Głęboka penetracja
      \end{itemize}
  \end{enumerate}
\end{frame}

% --- SEKCJA 4: BUDOWA PROTOTYPU ---
\section{Budowa prototypu}

\begin{frame}{Schemat blokowy urządzenia}
  Architektura systemu została podzielona na 5 głównych sekcji:
  
  \begin{center}

      \resizebox{5cm}{5cm}{%
    \begin{tikzpicture}[
        node distance=1.5cm and 2cm,
        probe/.style = {
            base,
            trapezium,
            trapezium left angle=70,
            trapezium right angle=70,
        },
        label_text/.style = {
            font=\sffamily\footnotesize\bfseries,
            fill=white,
            inner sep=2pt
        }
    ]
        \node[blok] (user) {Panel\\użytkownika};
        \node[blok, below=1.5cm of user] (digital) {Sekcja\\cyfrowa};
        \node[blok, below=2.5cm of digital, xshift=-3cm] (rx) {Sekcja\\odbiorcza};
        \node[blok, below=2.5cm of digital, xshift=3cm] (tx) {Sekcja\\nadawcza};
        \node[blok, below=2cm of $(rx.south)!0.5!(tx.south)$] (sonda) {Sonda};

        \draw[arrow] ([xshift=-3mm]user.south) -- ([xshift=-3mm]digital.north);
        \draw[arrow] ([xshift=3mm]digital.north) -- ([xshift=3mm]user.south);
        \draw[arrow] ([xshift=3mm]digital.south) -- node[label_text, pos=0.6] {PULSE\_OUT} (tx.north);
        \draw[arrow] (tx.south) -- ([xshift=3mm]sonda.north); 
        \draw[arrow] ([xshift=-3mm]sonda.north) -- (rx.south);
        \draw[arrow] (rx.north) -- node[label_text, pos=0.4] {PULSE\_IN} ([xshift=-3mm]digital.south);

    \end{tikzpicture}
}
  \end{center}

  \begin{itemize}
    \item \textbf{Interfejs:} Panel użytkownika
    \item \textbf{Sekcja cyfrowa:} Sterowanie i analiza sygnału
    \item \textbf{Sekcja nadawcza:} Generowanie impulsu dużej mocy
    \item \textbf{Sekcja odbiorcza:} Wzmacnianie słabego sygnału powrotnego
    \item \textbf{Sonda:} Cewka Monoloop
  \end{itemize}
\end{frame}

\begin{frame}{Sekcja Analogowa - Nadawcza i Odbiorcza}
  \begin{columns}
    \column{0.5\textwidth}
    \textbf{Nadajnik :}
    \begin{itemize}
      \item Tranzystor MOSFET IRF740 sterowany układem bipolarnym
      \item Generowanie silnego impulsu elektromagnetycznego
      \item Diodowe układy tłumiące oscylacje pasożytnicze (tłumienie krytyczne)
    \end{itemize}

    \column{0.5\textwidth}
    \textbf{Odbiornik :}
    \begin{itemize}
      \item Wzmacniacze operacyjne TL071 oraz MCP6281 (Rail-to-Rail)
      \item Zastosowanie wirtualnej masy ($V_{GND} = \frac{1}{2} V_{CC}$) do obsługi sygnałów bipolarnych
      \item Wzmocnienie sygnału echa po zaniku impulsu głównego
    \end{itemize}
  \end{columns}
\end{frame}

\begin{frame}{Sekcja Cyfrowa - STM32}
  Sercem układu jest mikrokontroler \textbf{STM32H523CE} (Cortex-M33).
  
  \vspace{0.5cm}
  \textbf{Kluczowe peryferia:}
  \begin{itemize}
    \item \textbf{Liczniki sprzętowe (Timers):} Precyzyjne generowanie impulsów (rozdzielczość $1\mu s$)
    \item \textbf{ADC (Przetwornik A/C):} Próbkowanie sygnału powrotnego z częstotliwością ok. 4,16 MSPS
    \item \textbf{DMA (Direct Memory Access):} Zapis próbek do bufora cyklicznego bez obciążania procesora
  \end{itemize}
\end{frame}

\begin{frame}{Sonda Detekcyjna}
  \begin{columns}
    \column{0.6\textwidth}
    \begin{itemize}
      \item Typ cewki: \textbf{Monoloop} (jedna cewka nadawczo-odbiorcza)
      \item Konstrukcja: Przewód miedziany 0.4mm, 31 zwojów, średnica 15cm
      \item Obudowa cewki zaprojektowana w CAD 3D i wydrukowana
    \end{itemize}
    \column{0.4\textwidth}
    \begin{center}
        \begin{figure}
            \begin{center}
                \includegraphics[width=0.95\textwidth]{img/donut otwarty kat.png}
            \end{center}
        \end{figure}
    \end{center}
  \end{columns}
\end{frame}

\section{Prototyp}

\begin{figure}
    \begin{center}
        \includegraphics[width=0.95\textwidth]{img/pudlo otwarte.png}
    \end{center}
\end{figure}

\section{Algorytm detekcji}

\begin{frame}{Algorytm wykrywania anomalii}
  Proces analizy sygnału realizowany programowo :
  
  \begin{enumerate}
    \item \textbf{Próbkowanie:} Pobranie próbek po wygaszeniu impulsu nadawczego
    \item \textbf{Buforowanie i uśrednianie:} Uśrednianie szumu sondy z wykorzystaniem EMA
    \item \textbf{Eliminacja tła:} Porównanie bieżącej średniej z EMA tła (dłuższa stała czasowa)
    \item \textbf{Detekcja:} Wykrycie różnicy przekraczającej próg (threshold) sygnalizuje obecność metalu
  \end{enumerate}

  \begin{figure}
      \begin{center}
          \includesvg[height=0.5\textheight]{img/testy/filtry_wykryte.svg}
      \end{center}
  \end{figure}
  
\end{frame}

% --- SEKCJA 6: REALIZACJA ---
\section{Realizacja praktyczna}

\begin{frame}{PCB i Obudowa}
  \begin{columns}
    \column{0.5\textwidth}
    \textbf{Płytka PCB :}
    \begin{itemize}
      \item Projekt w KiCad
      \item PCB dwuwarstwowa
      \item Integracja z modułem STM32
    \end{itemize}
    
    \column{0.5\textwidth}
    \textbf{Obudowa :}
    \begin{itemize}
      \item Projekt w Autodesk Fusion
      \item Technologia druku 3D (materiał PET-G)
      \item Ergonomiczne rozmieszczenie portów i wyświetlacza
    \end{itemize}
  \end{columns}

  \begin{center}
    \vspace{0.5cm}
    \begin{figure}
        \begin{center}
            \includegraphics[height=0.5\textheight]{img/pcb_3drender.png}
        \end{center}
    \end{figure}
    
  \end{center}
\end{frame}

    \begin{figure}
        \includegraphics[ width=1.0\textwidth]{Schematic.pdf}
    \end{figure}
% --- SEKCJA 7: PODSUMOWANIE ---
\section{Podsumowanie}

\begin{frame}{Podsumowanie}
  \begin{block}{Osiągnięte rezultaty}
    \begin{itemize}
      \item Zaprojektowano i zbudowano działający prototyp wykrywacza metali typu PI (Pulse Induction)
      \item Opracowano dedykowany układ elektroniczny (analogowy i cyfrowy)
      \item Zaimplementowano oprogramowanie sterujące na platformie STM32 z wykorzystaniem DMA
      \item Wykonano obudowę w technologii druku 3D
    \end{itemize}
  \end{block}

  \vspace{0.5cm}
  Projekt potwierdził możliwość budowy skutecznego wykrywacza impulsowego przy użyciu powszechnie dostępnych komponentów.
\end{frame}

\begin{frame}
  \begin{center}
    \Huge Dziękuję za uwagę
  \end{center}
\end{frame}

\end{document}
