\section{Sekcja odbiorcza}


\begin{figure}[H]
    \begin{center}
        \includegraphics[width=0.95\textwidth]{img/wzmocnienie.png}
    \end{center}
    \caption{Schemat sekcji przetwarzającej sygnał odebrany przez sondę}
\end{figure}
Uzyskana odpowiedź z cewki po przejściu przez kondensator usuwający offset DC podciągana jest do napięcia $VGND$ za pomocą rezystora $R_7$. Następnie odpowiedź jest mnożona 1000-krotnie w celu uwydatnienia ostatniej części odpowiedzi sygnału najbardziej podatnej na działanie powstałego pola magnetycznego. Uzyskane napięcie może posiadać wartość zbliżoną do napięcia zasilania $V_{cc}$, dlatego w celu dostosowania go do poziomu pozwalającego obsłużyć go przez zastosowany w wykrywaczu MCU zastosowano dzielnik napięcia wykonujący mnożenie razy $0,33$. Tak wyznaczony sygnał nie jest wystarczająco silny, dlatego przechodzi on przez kopiujący wzmacniacz $U_1$. Warto zaznaczyć, że wzmacniacz MCP6281 jest typu rail-to-rail, pozwalający na operowanie na sygnałach bliskich napięciom zasilania $V_+$, jak i $V_-$ wzmacniacza \cite{mcp6281_ds}. Poprzednie sekcje nie wymagały tej specjalnej właściwości. Tak przygotowany sygnał jest gotowy do analizy przez mikrokontroler. Dla dodatkowego bezpieczeństwa zastosowano diodę zenera $3,3V$ gwarantującą, że \textit{SIGNAL\_OUT} należy do zakresu napięć obsługiwanego przez sekcję cyfrową.


\subsection{Wirtualna masa}
\begin{figure}[H]
    \begin{center}
    \includegraphics[width=7.5cm]{img/vgnd.png}
    \end{center}
    \caption{Konfiguracja wzmacniacza operacyjnego generującego napięcie $VGND$}
\end{figure}

W celu uzyskania pełnej odpowiedzi cewki zastosowana jest sztuczna masa wynosząca $0,5$ napięcia zasilania. Dzięki niej wzmacniacze operacyjne w sekcji wzmocnienia odpowiedzi mogą wzmocnić odpowiedź pracując wyłącznie z dodatnim napięciem, co omija konieczność generowania ujemnego napięcia upraszczając finalny schemat wykrywacza metalu. Wykonana jest za pomocą dzielnika napięcia wykonującego działanie

\begin{equation}
    U_{VGND}=Vcc\cdot\frac{R9}{R8+R9}=Vcc\cdot\frac{1}{2}  
    \label{eq:vgnd}
\end{equation}
którego wynikowe napięcie przechodzi przez wzmacniacz operacyjny $U3$ kopiujący napięcie wejściowe wzmacniając sygnał, jednocześnie odciążając zastosowane w dzielniku rezystory.


