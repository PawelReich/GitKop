\section{Sekcja nadawcza}

\begin{figure}[H]
    \centering
    \includegraphics[width = 12cm]{img/budowa/sch/Sekcja_nadawcza.png}
    \caption{Schemat sekcji sterującej tranzystorem MOSFET przy użyciu sygnału \textit{PULSE\_IN}}
\end{figure}

Wygenerowany przez układ sterujący sekcją cyfrową sygnał prostokątny o napięciu $3,3V$ wzmacniany jest przez dwa tranzystory bipolarne $Q_2$, a następnie $Q_1$ połączone szeregowo pozwalające na uzyskanie wystarczającego napięcia do wysterowania tranzystora MOSFET IRF740 $Q_3$. Zastosowana konfiguracja tranzystorów bipolarnych pozwala na zminimalizowanie natężenia prądu wymaganego do wysterowania sekcji. Jest to szczególnie istotne biorąc pod uwagę fakt, że maksymalne natężenie oferowane przez zastosowany mikrokontroler wynosi $20mA$.
    
\begin{figure}[H]
    \centering
    \includegraphics[width = 7.5cm, height = 7.5cm]{img/output_characteristics.png}
    \caption{Charakterystyka wyjściowa tranzystora IRF740}
\end{figure}

\subsection{Pętla cewki}
\begin{figure}[H]
    \centering
    \includegraphics[width = 12cm]{img/budowa/sch/Pętla_cewki.png}
    \caption{Pętla rozładowująca cewkę}
\end{figure}

Tranzystor $Q_3$ zwiera sondę do masy powodując przepływ prądu przez cewkę podłączoną do \textit{COIL1} i \textit{COIL2}. Po otwarciu tranzystora $Q_3$ na cewce indukowane jest napięcie, którego górnym limitem jest przez napięcie breakdown $V_{brk}$ tranzystora wynoszące minimum $400V$ \cite{irf740}. W celu umożliwienia rozładowania napięcia w cewce zastosowano dwie szybkie diody typu \textit{BAV19} połączone równolegle odwrotnie do siebie pozwalające na przepływ prądu dla napięcia powyżej $0,7V$. Efekt jest opóźniony prez zastosowanie rezystora $R_6$. Dla końcowego etapu rozładowywania cewki, czyli kiedy napięcie jest poniżej progu przewodzenia diody rezystor $R5$ wraz z trymerem rezystorowym $RV1$ umożliwia przepływ prądu. Potencjometr zastosowany jest, aby łatwo dostroić tłumienie cewki w sposób inżynierski (zamiast wyznaczania wartości rezystora tłumiącego przy pomocy równania \ref{eq:pi_crit}). Pożądane jest krytyczne tłumienie odpowiedzi sondy pozwalające na najdokładniejsze pomiary zmian. Na rysunku 2.4 narysowano porównanie odpowiedzi sondy nietłumionej oraz krytycznie tłumionej. Zademonstrowana odpowiedź jest mierzona za kondensatorem $C_1$.

\begin{figure}[H]
    \centering
    \includegraphics[width=0.8\textwidth]{img/cewka_odp.png}
    \caption{Wykres porównawczy odpowiedzi cewki}
\end{figure}

