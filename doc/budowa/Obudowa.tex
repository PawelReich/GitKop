\section{Obudowa}
Obudowa wykrywacza metalu została wykonana w technologii druku 3D z wykorzystaniem plastiku PET-G.
Projekt posiada wycięcia na port sondy RJ45 oraz gniazdo zasilające DC. Od góry wykonane zostało wycięcie na wyświetlacz SSD1306 oraz wał enkodera.
Całość zamykana jest poprzez cztery śruby M3. Obudowa PCB, jak i sondy została zaprojektowana w oprogramowaniu CAD 3D Autodesk Fusion.
\begin{figure}[H]
    \begin{center}
        \includegraphics[width=0.7\textwidth]{img/pudlo otwarte.png}
    \end{center}
    \caption{Trójwymiarowe przedstawienie PCB zamkniętego w zaprojektowanej obudowie}
\end{figure}
