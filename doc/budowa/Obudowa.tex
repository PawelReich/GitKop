\section{Obudowa}
Obudowa wykrywacza metalu została wykonana w technologii druku 3D z wykorzystaniem plastiku PET-G gwarantującym większą wytrzymałość mechaniczną w porównaniu
do alternatyw.
Głównym założeniem projektowym było stworzenie kompaktowej konstrukcji. Wymiary zewnętrzne obudowy wynoszą odpowiednio $85mm$ x $126mm$ x $28mm$.
Projekt posiada dedykowane otwory na port sondy RJ45 oraz gniazdo zasilające DC.
Od góry wykonane zostało wycięcie na wyświetlacz SSD1306 oraz wał enkodera.
Całość zamykana jest poprzez cztery śruby M3, wkręcane we wkładki metalowe montowane na gorąco.
Obudowa PCB, jak i sondy została zaprojektowana w oprogramowaniu CAD 3D Autodesk Fusion.

\begin{figure}
    \begin{center}
        \includegraphics[width=\textwidth]{img/pudlo otwarte.png}
    \end{center}
    \caption{Trójwymiarowe przedstawienie PCB zamkniętego w zaprojektowanej obudowie}
\end{figure}
