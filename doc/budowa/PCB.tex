\section{Płytka drukowana}
Płytka drukowana (PCB, ang. \textit{Printed Circuit Board}) prototypu została zaprojektowana w otwartoźródłowym programie typu CAD (projektowanie wspomagane komputerowo, ang. \textit{Computer Aided Design}) KiCad\cite{kicad}.
Finalny produkt został wykonany z wykorzystaniem laminatu dwustronnego o wymiarach $80mm$ x $101mm$.
W celu minimalizacji zakłóceń elektromagnetycznych, wolne przestrzenie na obu warstwach PCB zostały wypełnione wylewką miedzi (ang. \textit{copper pour}) podłączoną do potencjału masy.

\begin{figure}[H]
\centering
  \includegraphics[width=0.5\textwidth]{./pcb.pdf}
  \caption{Schemat projektu PCB (z pominiętą płaszczyzną masy)}
\end{figure}

Projekt zawiera enkoder, który umożliwia interakcję z użytkownikiem poprzez przekręcanie oraz wciskanie wystającego wału oraz dwa porty przygotowane pod moduły wyświetlaczy ILI9341 oraz SSD1306.



\begin{figure}
\centering
  \includegraphics[width=0.7\textwidth]{./img/pcb_3drender.png}
  \caption{Trójwymiarowe komputerowe przedstawienie projektu PCB}
\end{figure}
