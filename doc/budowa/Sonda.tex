\section{Sonda detekcyjna}
Sonda detekcyjna wykonana została z miedzianego przewodu o przekroju 0.4mm. Nawinięto 31 zwojów o średnicy 15cm.

Zgodnie z formułą Wheelera (\ref{eq:wheeler}) przedstawioną podczas przeglądu literatury indukcyjność cewki może zostać oszacowana w łatwy sposób. Jednostką miary w omawianym wzorze jest cal, więc wzór należy przekonwertować, aby móc z niego wygodnie skorzystać posiadając wymiary cewki w systemie metrycznym:
\begin{equation}
    L = \frac{0,8\cdot N^2 (\frac{a}{2,54})^2}{\frac{1}{2,54}\cdot(6a + 9b + 10c)}=\frac{0,315 \cdot N^2 a^2}{6a + 9b + 10c}
    \label{eq:cewka_id}
\end{equation}
gdzie a, b, c są wymiarami w centymetrach według rysunku \ref{fig:poprz}, N jest ilością zwojów w cewce.

\begin{equation}
    L=\frac{0,315 \cdot N^2 a^2}{6a + 9b + 10c}=\frac{0,315 \cdot 31^2 \cdot 7,5^2}{6 \cdot 7,5 + 9 \cdot 0,5 + 10 \cdot 0,5} \approx 312,4 \mu H
    \label{eq:cewka_obl}
\end{equation}

\begin{figure}[H]
    \begin{center}
        \includegraphics[width=0.6\textwidth]{img/donut otwarty kat.png}
    \end{center}
    \caption{Komputerowe przedstawienie obudowy sondy}\label{fig:donut_otwarty}
\end{figure}

