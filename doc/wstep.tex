\chapter{Wstęp}
% tam można opisać rys historyczny, kto i kiedy jako pierwszy skonstruował tego typu urządzenie, do czego ono jest wykorzystywane i dlaczego jest ważne.
\section{Rys historyczny}
Historia wykrywaczy metali sięga połowy XIX wieku, kiedy w 1830 roku brytyjski geolog i wynalazca Robert Were Fox odkrył, że różne złoża metali wykazują różnicę potencjału elektrycznego. Na podstawie m.in. tej wiedzy skontruował pierwsze udokumentowane urządzenie, które było wykorzystywane w celu wykrywania metali w kopalniach. Eksperyment Foxa, przeprowadzony w kopalniach Kornwalii, opierał się na zasadzie, że żyły rud metalu otoczone wodą słoną i gliną tworzą naturalne ogniwa. W swojej metodzie nazwanej później metodą potencjału własnego (self-potential, SP) R.W. Fox umieszczał miedziane płytki w żyłach, łącząc je miedzianym przewodem z galwanometrem – urządzeniem do pomiaru natężenia prądu elektrycznego. Ruch wskaźnika przyrządu wskazywał na obecność minerałów \cite{fox}.

Przełom w technologii wykrywania metalu nastąpił w roku 1841. Pruski fizyk Heinrich Wilhelm Dove opublikował wtedy swój wynalazek zwany induktorem różnicowym (ang. differential inductor). Urządzenie to składało się z czterech cewek umieszczonych na dwóch szklanych rurkach, przy czym każda rurka owinięta była dwiema miedzianymi cewkami. Naładowane butelki lejdejskie (ówczesna forma kondensatorów wysokiego napięcia) były rozładowywane przez dwie cewki pierwotne, a powstały impuls prądowy indukował napięcie w cewkach wtórnych. Kluczowa dla działania urządzenia była konfiguracja, w której cewki wtórne połączono w opozycji - indukowane napięcia wzajemnie się znosiły, co profesor Dove weryfikował osobiście, trzymając końce cewek wtórnych. Gdy kawałek metalu został umieszczony wewnątrz jednej ze szklanych rurek, profesor Dove otrzymał wstrząs elektryczny. Ta obserwacja stanowiła fundamentalną zasadę działania pierwszego magnetycznego detektora metalu wykorzystującego indukcję magnetyczną, a jednocześnie pierwszego impulsowego detektora metalu w historii \cite{enwiki:1247591717}\cite{Dove1841}.

Niedługo później wykrywacz metalu został wykorzystany w służbie ratowania życia - Alexander Graham Bell wykorzystał technologię bilansu indukcyjnego (ang. Induction Balance) podczas próby zlokalizowania pocisku w rannym prezydencie Stanów Zjednoczonych Jamesie Garfieldzie. Motywacją naukowca było zastąpienie wymaganego dotychczas inwazyjnego nacinania pacjentów w celu poszukiwania pozostałego ciała obcego. Wykorzystanie detektora typu bilansu indukcyjnego nie przyniosło jednak  \cite{Bell1883Upon}
\todo[inline]{dokonczyc}

Na uwagę zasługuje polski wykrywacz min. W 1937 roku Departament Artylerii polskiego Ministerstwa Obrony Narodowej zlecił budowę urządzenia umożliwiającego zlokalizować niewybuchy pozostawione na poligonach. Za projekt odpowiadała AVA Wytwórnia Radiotechniczna \cite{enwiki:1253721450}. W 1939 roku po inwazji Niemiec na Polskę prace zostały wstrzymane. Po ucieczce do Wielkiej Brytanii porucznik Józef Kosacki dokończył swoje urządzenie i podarował technologię Brytyjskim Siłom Zbrojnym, za co Król Jerzy VI podziękował listownie. Wykrywacze "Mine Detector (Polish) Mark I" (pol. wykrywacz min) były wykorzystywane między innymi podczas II bitwy pod El Alamein, gdzie dwukrotnie przyspieszyły proces odminowywania terenów i uratowały tysiące żyć \cite{10824231}.

\begin{figure}[H]
    \centering
    \includegraphics[width = 7.5cm]{img/saperzy.jpg}
    \caption{Saperzy Korpusu Inżynierów Królewskich używający wykrywacza min, 28 sierpień 1942}
    \hfill\small źródło: iwm.org.uk \cite{saperzy_jpg} 
\end{figure}

\section{Zastosowanie wykrywaczy metali we współczesnym świecie}
W dzisiejszych czasach wykrywacze metali posiadają niezwykle wszechstronne zastosowanie, idące daleko poza wymienione wcześniej cele wykopaliskowe, militarne lub medyczne.

W sektorze bezpieczeństwa publicznego wykrywacze stanowią witalny element zabezpieczenia lotnisk oraz innych miejsc masowych zgromadzeń ludzi. Systemy bramek detektujących obecność metalu pozwalają na szybkie i nieinwazyjne zapewnienie bezpieczeństwa, chroniąc przed wnoszeniem niebezpiecznych przedmiotów, takich jak ostre narzędzia, czy broń palna. Ręczne detektory metali, stosowane przez funkcjonariuszy publicznych jak i ochronę pozwalają na przefiltrowanie osób przed wejściem na imprezy masowe.

W przemyśle spożywczym, chemicznym oraz farmaceutycznym wykrywacze metali stanowią podstawowy element systemu kontroli jakości. Pozwalają wykryć potencjalne zanieczyszczenia w produktach oraz półfabrykatach, zapobiegając trafieniu do konsumentów produktów niebezpiecznych dla zdrowia człowieka.

Wykrywacze metalu znajdują również zastosowanie w archeologii oraz badaniach naukowych. Archeolodzy posługują się tymi instrumentami do odkrywania ukrytych w ziemi artefaktów oraz zabytków, takich jak biżuteria, monety, narzędzia, czy broń, bez konieczności przeprowadzania destrukcyjnych prac wykopaliskowych.

\section{Cel pracy}

Praca inżynierska ma na celu zbadanie 
\todo[inline]{dokonczyc}
