\chapter{Przegląd literatury}
W niniejszym przeglądzie przyjęto założenia upraszczające szeroko stosowane w literaturze naukowej dotyczącej podstaw detekcji metali, pozwalających jednocześnie na uzyskanie rozwiązań o wysokiej wartości podczas rozważań inżynierskich:

\begin{itemize}
    \item Quasi-statyczna aproksymacja,
    \item Badany obiekt jest traktowany jako jednorodna sfera o zadanej konduktywności $\sigma$, przenikalności magnetycznej $\mu$ i promieniu $R$,
    \item Badany obiekt nie porusza się ($\mathbf{v}=0$).
\end{itemize}

\section{Elektrodynamika systemów wykrywania metali}
Podstawą funkcjonowania wszystkich indukcyjnych wykrywaczy metali jest prawo indukcji Faradaya.
Zjawisko to wiąże w czasie pole magnetyczne z indukowanym wirowym polem elektrycznym \cite[p.~302]{griffiths1999introduction}.
Zgodnie z równaniem
\begin{equation}
\label{eq:prawo_faradaya}
\nabla \times \mathbf{E} = - \frac{\partial \mathbf{B}}{\partial t}
\end{equation}
gdzie $\mathbf{E}$ oznacza wektor natężenia, a $\mathbf{B}$ oznacza wektor indukcji magnetycznej.

\begin{figure}[htbp]
  \centering
  \includesvg[width=0.3\textwidth]{./img/Alternator_1.svg}
  \caption{Schemat alternatora przedstawiający obracający się magnes (wirnik) i nieruchome uzwojenie drutu (stator) oraz napięcie wytwarzane, gdy obracające się pole magnetyczne indukuje prąd w przewodzie.}
    \hfill\small źródło: wikipedia.org \cite{alternator_svg} 
\end{figure}

W kontekście inżynierii systemów detekcji cewka nadawcza zasilana prądem zmiennym lub impulsowym generuje zmienny strumień magnetyczny. Strumień ten penetruje przestrzeń, w tym potencjalne obiekty przewodzące oraz grunt. Zgodnie z przytoczonym prawem, zmienność strumienia w czasie ($\partial \mathbf{B}/\partial t$) indukuje siłę elektromotoryczną w każdym przewodniku objętym polem.

%%przeredagowac
Kluczowym mechanizmem, na którym opiera się detekcja, jest powstawanie prądów wirowych (ang. eddy currents). Prądy te, płynąc w zamkniętych pętlach wewnątrz materiału przewodzącego (w płaszczyznach prostopadłych do linii pola magnetycznego), generują własne, wtórne pole magnetyczne \cite{kriezis}. Istotę tego zjawiska wyjaśnia reguła Lenza, która stanowi, że kierunek indukowanego prądu jest taki, iż wytworzone przez niego pole magnetyczne przeciwdziała zmianie pola pierwotnego, które je wywołało \cite{kriezis}.

\begin{equation}
    \label{eq:prawo_lenza}
    \mathcal{E} = -\frac{d \Phi_\mathbf{B}}{d t}
\end{equation}

To wzajemne oddziaływanie pól – pierwotnego (wymuszającego) i wtórnego (reakcyjnego) – prowadzi do mierzalnych zmian impedancji wzajemnej układu cewek lub zmian napięcia indukowanego w cewce odbiorczej, co stanowi sygnał użyteczny dla wykrywacza metalu.

\section{Dyfuzja magnetyczna}

\todo[inline]{wprowadzenie do dyfuzji magnetycznej}

\subsection{Wyprowadzenie wzoru}

Wyprowadzenie wzoru na dyfuzję magnetyczną przy założeniach określonych na wstępie do rozdziału można rozpocząć od wyznaczenia wzoru na gęstość prądu z prawa Ampère’a:
\todo[inline]{opisać składowe równań}
\begin{equation}
    \label{eq:ampere}
    \nabla \times \mathbf{B} = \mu \mathbf{J}
\end{equation}
\begin{equation}
    \label{eq:gestosc_pradu}
    \mathbf{J} = \frac{1}{\mu} (\nabla \times \mathbf{B})
\end{equation}
2. Prawo Ohma:
\begin{equation}    
    \label{eq:ohma}
    \mathbf{J} = \sigma \mathbf{E}
\end{equation}
Łącząc wzór \ref{eq:gestosc_pradu} ze wzorem na gęstość prądu wynikającą z prawa Ohma, otrzymano wzór na pole elektryczne wyrażone przez pole magnetyczne
\begin{equation}
     \mathbf{E} = \frac{1}{\mu \sigma} (\nabla \times \mathbf{B})
     \label{eq:elektryczne_przez_magnetyczne}
\end{equation}
Wiedząc, że gradient pola elektycznego jest równy ujemnej zmianie pola magnetycznego (równanie Maxwella-Faradaya, \ref{eq:prawo_faradaya})
\begin{equation}
\nabla \times \mathbf{E} = - \frac{\partial \mathbf{B}}{\partial t}
      \tag{\ref{eq:prawo_faradaya}}
\end{equation}
podstawiono $\mathbf{E}$ ze wzoru \ref{eq:elektryczne_przez_magnetyczne}:
\begin{equation}
    \nabla \times \left( \frac{1}{\mu \sigma} (\nabla \times \mathbf{B}) \right) = - \frac{\partial \mathbf{B}}{\partial t}
    \label{eq:test}
\end{equation}
Korzystając z tożsamości wektorowej

$$\nabla \times (\nabla \times \mathbf{B}) = \nabla(\nabla \cdot \mathbf{B}) - \nabla^2 \mathbf{B}$$
otrzymano:
\begin{equation}
    - \frac{1}{\mu \sigma} \nabla^2 \mathbf{B} = - \frac{\partial \mathbf{B}}{\partial t}
    \label{eq:diffused_dirty}
\end{equation}
Po uporządkowaniu otrzymano równanie dyfuzji:
\begin{equation}
    \label{eq:diffused}
    \nabla^2 \mathbf{B} = \mu \sigma \frac{\partial \mathbf{B}}{\partial t}
\end{equation}

gdzie $\mu$ to absolutna przenikalność magnetyczna materiału, a $\sigma$ to jego konduktywność (przewodność właściwa). Równanie to wskazuje, że interakcja pola z metalem nie jest natychmiastowa, lecz podlega procesom relaksacyjnym zależnym od właściwości materiałowych obiektu oraz częstotliwości fali pobudzającej. Prowadzi to do zjawiska naskórkowości (ang. skin effect), które ogranicza penetrację prądów wirowych do warstw powierzchniowych przy wyższych częstotliwościach, co ma fundamentalne znaczenie dla identyfikacji obiektów.

\subsection{Zjawisko naskórkowości}

\begin{table}[H]
    \centering
    \caption{Rezystywność wybranych materiałów dla 1 atm w 20$^\circ$C.}
    \begin{tabular}{llll}
        \toprule
        \textbf{Materiał} & \textbf{Rezystywność ($\Omega \cdot m$)} & \textbf{Materiał} & \textbf{Rezystywność ($\Omega \cdot m$)} \\
    \midrule
        \textit{Przewodniki:} & & \textit{Półprzewodniki:} & \\
        Srebro & $1,59 \times 10^{-8}$ & Woda morska & $0,2$ \\
        Miedź & $1,68 \times 10^{-8}$ & German & $0,46$ \\
        Złoto & $2,21 \times 10^{-8}$ & Diament & $2,7$ \\
        Aluminium & $2,65 \times 10^{-8}$ & Krzem & $2500$ \\
        Żelazo & $9,61 \times 10^{-8}$ & \textit{Izolatory:} & \\
        Rtęć & $9,61 \times 10^{-7}$ & Woda (czysta) & $8,3 \times 10^3$ \\
        Nichrom & $1,08 \times 10^{-6}$ & Szkło & $10^9 - 10^{14}$ \\
        Mangan & $1,44 \times 10^{-6}$ & Guma & $10^{13} - 10^{15}$ \\
        Grafit & $1,6 \times 10^{-5}$ & Teflon & $10^{22} - 10^{24}$ \\
    \bottomrule
    \end{tabular}
    \vspace{0.2cm} \\
    \hfill\small źródło: Handbook of Chemistry and Physics \cite{haynes2010crc} 
\end{table}
\begin{table}[H]
    \centering
    \caption{Maksymalna przenikalność magnetyczna wybranych materiałów dla niskich częstotliwości}
    \begin{tabular}{l l l}
        \toprule
        \textbf{Substancja} & \textbf{Typ grupy} & \textbf{Względna przenikalność } $\mu_r$ \\
        \midrule
        Bizmut & Diamagnetyk & 0,99983 \\
        Srebro & Diamagnetyk & 0,99998 \\
        Ołów & Diamagnetyk & 0,99983 \\
        Miedź & Diamagnetyk & 0,99991 \\
        Woda & Diamagnetyk & 0,99991 \\
        \addlinespace
        Próżnia & Niemagnetyczny & 1 \\
        \addlinespace
        Powietrze & Paramagnetyk & 1,0000004 \\
        Aluminium & Paramagnetyk & 1,00002 \\
        Pallad & Paramagnetyk & 1,0008 \\
        \addlinespace
        Kobalt & Ferromagnetyk & 250 \\
        Nikiel & Ferromagnetyk & 600 \\
        Ferroxcube 3 (proszek ferrytowy Mn-Zn) & Ferromagnetyk & 1 500 \\
        Stal miękka (0,2\% C) & Ferromagnetyk & 2 000 \\
        Żelazo (0,2\% zanieczyszczeń) & Ferromagnetyk & 5 000 \\
        \addlinespace
        Żelazo krzemowe (4\% Si) & Ferromagnetyk & 7 000 \\
        Permalloy 78 (78,5\% Ni) & Ferromagnetyk & 100 000 \\
        Mumetal (75\% Ni, 5\% Cu, 2\% Cr) & Ferromagnetyk & 100 000 \\
        Oczyszczone żelazo (0,05\% zanieczyszczeń) & Ferromagnetyk & 200 000 \\
        Supermaloj (5\% Mo, 79\% Ni) & Ferromagnetyk & 1 000 000 \\
        \bottomrule
    \end{tabular}

    \vspace{0.2cm}
    \hfill\small źródło: Electromagnetics Explained, \cite{schmitt2002electromagnetics}, Electromagnetics with Applications \cite[p.~201]{kraus1999electromagnetics} 
\end{table}

\section{Sonda}

Podstawowy wzór na indukcyjność cewki soleidalnej:

\begin{equation}
    L = \frac{\mu N^2 A}{l}
    \label{eq:indukcyjnosc_cewki}
\end{equation}
gdzie $\mu$ oznacza przenikalność materiału użytego do wykonania cewki, N oznacza ilość wykonanych zwojów, A oznacza pole przekroju poprzecznego, l długość

Wykrywacze metalu z reguły nie korzystają z tego rodzaju sondy.

\begin{figure}[H]
    \begin{center}
        \includegraphics[width=0.95\textwidth]{./img/multiturn_cewka.png}
    \end{center}
    \caption{Przekrój poprzeczny cewki}\label{fig:poprz}
\end{figure}


\begin{equation}
    L = \frac{0,8 (N A^2)}{6a + 9b + 10c}
    \label{eq:indukcyjnosc_dobra}
\end{equation}



\section{Analiza porównawcza architektur detektorów}


\subsection{BFO - Beat Frequency Oscillator}


Technologia wykrywaczy metali oparta o generator zdudnieniowy (ang. \textit{Beat Frequency Oscillator}).
Podstawowy system BFO opiera się o dwa oscylatory, będące dostrojone do jak najbliższej wspólnej częstotliwości.
Do generowania żądanej częstotliwości wykorzystywane są przeważnie obwody rezonansowe LC.
Pierwszy z oscylatorów, referencyjny, korzysta z cewki o zadanej indukcyjności.
Oscylator szukający, używa sondy wykrywacza metali jako swoje źródło indukcyjności. \cite{moreland_bfo_theory}

\begin{figure}[H]
    \begin{center}
        \includegraphics[width=0.7\textwidth]{./img/bfo_uklad.png}
    \end{center}
    \caption{Schemat podstawowego systemu BFO}\label{fig:bfo}
\end{figure}
\todo[inline]{przerysowac i przetlumaczyc na polski}

Generowane przez układy rezonansowe sygnały wprowadzane są do mieszacza częstotliwości, który wyznacza sumę oraz różnicę obu częstotliwości.
Wymieszany sygnał jest następnie traktowany filtrem dolnoprzepustowym, usuwającym wysokie składowe częstotliwości.
Wyjście filtru

\subsection{IB}

Przedstawiona przed doktora  \cite{doi:10.1098/rspl.1879.0012}.

\subsection{PI - Pulse Induction}
\cite{moreland2024inside}   

