\chapter{Budowa wykrywacza}

\section{Zasilanie układu}
Wykrywacz jest zasilany za pomocą napięcia +9V, otrzymywanego za pomocą urządzenia Powerbank posiadającego port USB-C wspierający protokół \textit{Power Delivery}. 


\subsection{Zarządzanie stanem sygnału wyjściowego}

\section{Generowanie sygnału}

Wykrywacz generuje sygnał za pomocą układu STM32F411RE, który za pomocą sprzętowego licznika wyzwala sygnał trwający $300 \mu s$  co $10 ms$, patrz \cref{subsec:timery}

\section{Sekcja wzmocnienia sygnału cyfrowego i załączenia cewki}

\begin{figure}[H]
    \centering
    \includegraphics[width = 12cm]{tranzystory2.png}
    \caption{Schemat sekcji}
\end{figure}

Generowany sygnał pulsu (\textit{PULSE\_IN}) jest wzmacniany przez dwa tranzystory bipolarne połączone szeregowo pozwalające na uzyskanie wystarczającego napięcia do wysterowania tranzystora MOSFET IRF740 $Q_3$.
    
\begin{figure}[H]
    \centering
    \includegraphics[width = 7.5cm, height = 7.5cm]{output_characteristics.png}
    \caption{Charakterystyka wyjściowa tranzystora IRF740}
\end{figure}

\section{Układ cewki}
\begin{figure}[H]
    \centering
    \includegraphics[width = 7.5cm]{cewka_sch.png}
    \caption{sch cewki}
\end{figure}

Cewka sterowana jest 

\section{Wirtualna masa}

W celu uzyskania pełnej odpowiedzi cewki zastosowana jest sztuczna masa wynosząca $\frac{1}{2}$ napięcia zasilania. Dzięki niej wzmacniacze operacyjne w sekcji wzmocnienia odpowiedzi mogą wzmocnić odpowiedź pracując wyłącznie z dodatnim napięciem, co omija konieczność generowania ujemnego napięcia upraszczając finalny schemat wykrywacza metalu.

\begin{figure}[H]
    \centering
    \includegraphics[width = 7.5cm]{vgnd.png}
    \caption{Konfiguracja wzmacniacza operacyjnego generującego napięcie $VGND$}
\end{figure}
-
\section{Sekcja wzmocnienia sygnału odpowiedzi}

Uzyskana odpowiedź z cewki po przejściu przez kondensator usuwający DC offset podciągana jest do napięcia $VGND$ za pomocą rezystora. Następnie odpowiedź jest mnożona x1000 w celu powiększenia ostatniej 


\section{Część cyfrowa}

\begin{figure}[H]
    \begin{center}
        \includegraphics[width=7.5cm]{stm32.png}
    \end{center}
    \caption{Cyfrowe przedstawienie układu STM32F411RE wraz z użytymi pinami}\label{fig:}
\end{figure}


Część cyfrowa wykrywacza metalu oparta jest o układ \textbf{STM32F411RE}. Ten produkowany przez ST Microelectronics mikrokontroler posiada jeden rdzeń oparty na architekturze ARM\textregistered{} Cortex-M4\textregistered{} oraz m.in wykorzystane w projekcie układy peryferyjne:

\subsection{Liczniki sprzętowe}
\label{subsec:timery}
Wykorzystywane są dwa 16-bitowe liczniki sprz1ętowe napędzane za pomocą zegara \textit{APB2} skonfigurowanego na taktowanie z częstotliwością $F_{timer}=72Mhz$. Częstotliwość przepełnienia licznika sprzętowego można wyznaczyć za pomocą poniższego równania: 

\begin{equation}
F_{output}=\frac{F_{timer}}{(PSC+1)(ARR+1)}
\end{equation}

gdzie PSC oznacza Prescaler, natomiast ARR oznacza Auto Reload Register


Licznik \textit{TIM10} pełni funkcję maszyny stanów zarządzając aktualnym stanem sygnału \textit{PULSE\_OUT}.
W celu zachowania dokładności przy jednoczesnym jak najmniejszym obciążeniu MCU licznik wywołuje przerwania w częstotliwości $F_{output}=1\mu s$ poprzez konfigurację $PSC=0, ARR=71$.

Podczas pracy układu cyfrowego w funkcji przerwania sprawdzana jest wartość zmiennej inkrementowanej co każde wywołanie przerwania. Jeżeli licznik jest równy 0, wystawiany jest sygnał +3.3V na pin 1 (PULSE\_OUT) układu. Następnie w pożądanym momencie sygnał wyjściowy jest wygaszony, co pozwala na dynamiczną konfigurację długości impulsu wysyłanego do cewki.

\begin{lstlisting}[language=c++, caption={Funkcja zarządzająca stanem sygnału wyjściowego}]
void PulseOut_Handler() {
 
    if (pulseTickCtr == 0)
    {
        HAL_GPIO_WritePin(PULSE_OUT_GPIO_Port, PULSE_OUT_Pin, 1);
    }

    if (pulseTickCtr == 251)
    {
        HAL_GPIO_WritePin(PULSE_OUT_GPIO_Port, PULSE_OUT_Pin, 0);
    }

    pulseTickCtr++;

    /* ... */

    if (pulseTickCtr == 2500)
    {
        pulseTickCtr = 0;
    }3
}
\end{lstlisting}

Drugi licznik \textit{TIM2} wykorzystany jest do generowania sygnału PWM o wypełnieniu 50\% i dynamicznie wyznaczanej częstotliwości do brzęczyka wskazującego o wykryciu anomalii w pomierzonym sygnale \textit{PULSE\_IN}. Preskaler jest skonfigurowany na wartość 71, co daje przerwanie co mikrosekundę.

\begin{lstlisting}[language=c++,caption={Funkcja generująca dzwięk o zadanej częstotliwości}]
void Buzz(uint16_t freq) {
    if (freq == 0)
    {
        __HAL_TIM_SET_COMPARE(&htim2, TIM_CHANNEL_1, 0);
        return;
    }

    uint16_t autoreload = (1000000UL / freq) - 1;

    __HAL_TIM_SET_AUTORELOAD(&htim2, autoreload);
    __HAL_TIM_SET_COMPARE(&htim2, TIM_CHANNEL_1, autoreload / 2);
}
\end{lstlisting}

\subsection{Konwerter cyfrowo-analogowy}

ADC napędzany jest sygnałem zegarowym \textit{PCLK2} o częstotliwości $72Mhz$, z preskalerem o wartości $2$, co daje końcową częstotliwość podukładu wynoszącą $36Mhz$.
Maksymalna rozdzielczość pomiaru wynosi 12 bitów, co daje dokładność na poziomie 8 mV.

\todo[inline]{tutaj potencjalnie porownanie rozdzielczosc vs dokladnosc wykrywania metalu, 
w gre wchodzi tez fakt, ze ADC dla 12-bit zajmuje 15 cykli, 10-bit 13, 8-bit 11, 6-bit 9)}

\begin{equation}
R= \frac{Vref}{2^N - 1} =\frac{3.3V}{4096}\approx 8mV
\end{equation}

gdzie $V_{ref}$  to napięcie referencyjne (tutaj $V_{in}=3.3V$), N = ilość bitów


W celu osiągnięcia jak największego zasięgu wykrywacza wymagana jest jednocześnie jak największa dokładność układu ADC oraz jego szybkość działania. Dla trybu 12-bitowego każdy pomiar zajmuje 15 cykli zegarowych dających teorytyczny limit $2.4 MSPS$, co daje jedną próbkę napięcia co $~0.41\mu s$.

Dla maksymalnego odciążenia mikrokontrolera wykorzystany został tryb DMA z wyłączonym generowaniem przerwań sprzętowych. Każda nowa próbka zmierzona przez konwerter zapisywana jest w docelowym miejscu pamięci skazanym podczas konfiguracji bez dodatkowego udziału MCU.
Podczas wykonywania kodu maszyny stanów w pożądanym momencie kopiowana jest aktualna wartość próbki i dokonywany jest proces analizy sygnału.


