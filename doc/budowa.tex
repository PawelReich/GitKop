\chapter{Budowa wykrywacza}

\section{Zasilanie układu}
Wykrywacz jest zasilany za pomocą napięcia +9V, otrzymywanego za pomocą urządzenia Powerbank posiadającego port USB-C wspierający protokół \textit{Power Delivery}. 

\section{Generowanie sygnału}

Wykrywacz generuje sygnał za pomocą układu ESP32, który za pomocą sprzętowego licznika wyzwala sygnał trwający $300 \mu s$  co $10 ms$.

\subsection{Konfiguracja licznika sprzętowego}

Licznik sprzętowy ESP32 skonfigurowany jest do generowania przerwania w częstotliwości $20 MHz$, co pozwala uzyskać dokładność tworzenia sygnału do $50\mu s$.

\begin{lstlisting}[language=c++, caption={Fragment kodu odpowiedzialny za konfigurację licznika sprzętowego}]
hw_timer* pulseTimer = NULL;
void timer_setup() {
    pulseTimer = timerBegin(2000);
    timerAlarm(pulseTimer, 1, true, 0);
    timerAttachInterrupt(pulseTimer, &pulseTimerFired);
    timerStart(pulseTimer);
}
\end{lstlisting}

\subsection{Zarządzanie stanem sygnału wyjściowego}
Podczas pracy układu cyfrowego w funkcji przerwania sprawdzana jest wartość zmiennej inkrementowanej co każde wywołanie przerwania. Jeżeli licznik jest równy 0, wystawiany jest sygnał +3.3V na pin 1 (PULSE\_OUT) układu. Następnie w pożądanym momencie sygnał wyjściowy jest wygaszony, co pozwala na dynamiczną konfigurację długości impulsu wysyłanego do cewki.

\begin{lstlisting}[language=c++, caption={Funkcja zarządzająca stanem sygnału wyjściowego}]
void IRAM_ATTR pulseTimerFired()
{
    if (pulseTickCtr == 0)
    {
        digitalWrite(PULSE_OUT, HIGH);
    }
    else if (pulseTickCtr == 10)
    {
        digitalWrite(PULSE_OUT, LOW);
    }

    pulseTickCtr++;

    if (pulseTickCtr == 2 * 100)
    {
        pulseTickCtr = 0;
    }
}
\end{lstlisting}

\section{Sekcja wzmocnienia sygnału cyfrowego i załączenia cewki}

\begin{figure}[H]
    \centering
    \includegraphics[width = 12cm]{tranzystory2.png}
    \caption{Schemat sekcji}
\end{figure}

Generowany sygnał pulsu (\textit{PULSE\_IN}) jest wzmacniany przez dwa tranzystory bipolarne połączone szeregowo pozwalające na uzyskanie wystarczającego napięcia do wysterowania tranzystora MOSFET IRF740 $Q_3$.
    
\begin{figure}[H]
    \centering
    \includegraphics[width = 7.5cm, height = 7.5cm]{output_characteristics.png}
    \caption{Charakterystyka wyjściowa tranzystora IRF740}
\end{figure}

\section{Układ cewki}
\begin{figure}[H]
    \centering
    \includegraphics[width = 7.5cm]{cewka_sch.png}
    \caption{sch cewki}
\end{figure}

Cewka sterowana jest 

\section{Wirtualna masa}

W celu uzyskania pełnej odpowiedzi cewki zastosowana jest sztuczna masa wynosząca $\frac{1}{2}$ napięcia zasilania. Dzięki niej wzmacniacze operacyjne w sekcji wzmocnienia odpowiedzi mogą wzmocnić odpowiedź pracując wyłącznie z dodatnim napięciem, co omija konieczność generowania ujemnego napięcia upraszczając finalny schemat wykrywacza metalu.

\begin{figure}[H]
    \centering
    \includegraphics[width = 7.5cm]{vgnd.png}
    \caption{Konfiguracja wzmacniacza operacyjnego generującego napięcie $VGND$}
\end{figure}

\section{Sekcja wzmocnienia sygnału odpowiedzi}

Uzyskana odpowiedź z cewki po przejściu przez kondensator usuwający DC offset podciągana jest do napięcia $VGND$ za pomocą rezystora. Następnie odpowiedź jest mnożona x1000 w celu powiększenia ostatniej 

\section{Część cyfrowa}


