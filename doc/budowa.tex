\chapter{Budowa prototypu wykrywacza metalu}
\begin{figure}[H]
\centering
    \begin{tikzpicture}[
        node distance=1.5cm and 2cm,
        probe/.style = {
            base,
            trapezium,
            trapezium left angle=70,
            trapezium right angle=70,
        },
        label_text/.style = {
            font=\sffamily\footnotesize\bfseries,
            fill=white,
            inner sep=2pt
        }
    ]
        \node[blok] (user) {Panel\\użytkownika};
        \node[blok, below=1.5cm of user] (digital) {Sekcja\\cyfrowa};
        \node[blok, below=2.5cm of digital, xshift=-3cm] (rx) {Sekcja\\odbiorcza};
        \node[blok, below=2.5cm of digital, xshift=3cm] (tx) {Sekcja\\nadawcza};
        \node[blok, below=2cm of $(rx.south)!0.5!(tx.south)$] (sonda) {Sonda};

        \draw[arrow] ([xshift=-3mm]user.south) -- ([xshift=-3mm]digital.north);
        \draw[arrow] ([xshift=3mm]digital.north) -- ([xshift=3mm]user.south);
        \draw[arrow] ([xshift=3mm]digital.south) -- node[label_text, pos=0.6] {PULSE\_OUT} (tx.north);
        \draw[arrow] (tx.south) -- ([xshift=3mm]sonda.north); 
        \draw[arrow] ([xshift=-3mm]sonda.north) -- (rx.south);
        \draw[arrow] (rx.north) -- node[label_text, pos=0.4] {PULSE\_IN} ([xshift=-3mm]digital.south);

    \end{tikzpicture}
    \caption{Schemat blokowy działania prototypu wykrywacza}
    \label{sch:budowa}
\end{figure}
Prototyp został wykonany z wykorzystaniem wiedzy zebranej w poprzednich rozdziałach.
Urządzenie zostało zbudowane w oparciu o łatwo dostępne komponenty oraz wydruk 3D.
Architektura wykonanego urządzenia podzielona jest na 5 części ukazanych na rysunku \ref{sch:budowa}.
W niniejszym rozdziale zostaną omówione zastosowane rozwiązania tworzące poszczególne sekcje oraz technikalia ich działania.

\section{Zasilanie}

Zasilanie zbudowanego urządzenia odbywa się przez dostępne z lewej strony gniazdo DC 5,5mm o polaryzacji dodatniej bolca.
Wspierany przez układ zakres napięcia wynosi od $+9V$ to $+12V$ DC. W celu tłumienia zakłóceń spowodowanych przez nagłe skoki przepływu prądu w momencie sterowania sondą detekcyjną zastosowano kondensatory o pojemności $1500\mu F$ i maksymalnemu napięciu $16V$. Do zasilania sekcji cyfrowej zastosowano stabilizator liniowy $LF33CV$ obniżający napięcie do 3,3V bez wprowadzania zakłóceń \cite{lf33cv_ds} o wysokiej częstotliwości. 

\section{Sekcja nadawcza}

\begin{figure}[H]
    \centering
    \includegraphics[width = 12cm]{img/budowa/sch/Sekcja_nadawcza.png}
    \caption{Schemat sekcji sterującej tranzystorem MOSFET przy użyciu sygnału \textit{PULSE\_IN}}
\end{figure}

Wygenerowany przez układ sterujący sekcją cyfrową sygnał prostokątny o napięciu $3,3V$ wzmacniany jest przez dwa tranzystory bipolarne $Q_2$, a następnie $Q_1$ połączone szeregowo pozwalające na uzyskanie wystarczającego napięcia do wysterowania tranzystora MOSFET IRF740 $Q_3$. Zastosowana konfiguracja tranzystorów bipolarnych pozwala na zminimalizowanie natężenia prądu wymaganego do wysterowania sekcji. Jest to szczególnie istotne biorąc pod uwagę fakt, że maksymalne natężenie oferowane przez zastosowany mikrokontroler wynosi $20mA$.
    
\begin{figure}[H]
    \centering
    \includegraphics[width = 7.5cm, height = 7.5cm]{img/output_characteristics.png}
    \caption{Charakterystyka wyjściowa tranzystora IRF740}
\end{figure}

\subsection{Pętla cewki}
\begin{figure}[H]
    \centering
    \includegraphics[width = 12cm]{img/budowa/sch/Pętla_cewki.png}
    \caption{Pętla rozładowująca cewkę}
\end{figure}

Tranzystor $Q_3$ zwiera sondę do masy powodując przepływ prądu przez cewkę podłączoną do \textit{COIL1} i \textit{COIL2}. Po otwarciu tranzystora $Q_3$ na cewce indukowane jest napięcie, którego górnym limitem jest przez napięcie breakdown $V_{brk}$ tranzystora wynoszące minimum $400V$ \cite{irf740}. W celu umożliwienia rozładowania napięcia w cewce zastosowano dwie szybkie diody typu \textit{BAV19} połączone równolegle odwrotnie do siebie pozwalające na przepływ prądu dla napięcia powyżej $0,7V$. Efekt jest opóźniony prez zastosowanie rezystora $R_6$. Dla końcowego etapu rozładowywania cewki, czyli kiedy napięcie jest poniżej progu przewodzenia diody rezystor $R5$ wraz z trymerem rezystorowym $RV1$ umożliwia przepływ prądu. Potencjometr zastosowany jest, aby łatwo dostroić tłumienie cewki w sposób inżynierski (zamiast wyznaczania wartości rezystora tłumiącego przy pomocy równania \ref{eq:pi_crit}). Pożądane jest krytyczne tłumienie odpowiedzi sondy pozwalające na najdokładniejsze pomiary zmian. Na rysunku 2.4 narysowano porównanie odpowiedzi sondy nietłumionej oraz krytycznie tłumionej. Zademonstrowana odpowiedź jest mierzona za kondensatorem $C_1$.

\begin{figure}[H]
    \centering
    \includegraphics[width=0.8\textwidth]{img/cewka_odp.png}
    \caption{Wykres porównawczy odpowiedzi cewki}
\end{figure}


\section{Sonda detekcyjna}
Sonda detekcyjna wykonana została z miedzianego przewodu o przekroju 0.4mm. Nawinięto 31 zwojów o średnicy 15cm.
\todo[inline]{wyznaczyc teorytyczna indukcyjnosc}

\begin{figure}
    \begin{center}
        \includegraphics[width=0.6\textwidth]{img/donut otwarty kat.png}
    \end{center}
    \caption{Komputerowe przedstawienie obudowy sondy}\label{fig:donut_otwarty}
\end{figure}


\section{Sekcja odbiorcza}


\begin{figure}[H]
    \begin{center}
        \includegraphics[width=0.95\textwidth]{img/wzmocnienie.png}
    \end{center}
    \caption{Schemat sekcji przetwarzającej sygnał odebrany przez sondę}
\end{figure}
Uzyskana odpowiedź z cewki po przejściu przez kondensator usuwający offset DC podciągana jest do napięcia $VGND$ za pomocą rezystora $R_7$. Następnie odpowiedź jest mnożona 1000-krotnie w celu uwydatnienia ostatniej części odpowiedzi sygnału najbardziej podatnej na działanie powstałego pola magnetycznego. Uzyskane napięcie może posiadać wartość zbliżoną do napięcia zasilania $V_{cc}$, dlatego w celu dostosowania go do poziomu pozwalającego obsłużyć go przez zastosowany w wykrywaczu MCU zastosowano dzielnik napięcia wykonujący mnożenie razy $0,33$. Tak wyznaczony sygnał nie jest wystarczająco silny, dlatego przechodzi on przez kopiujący wzmacniacz $U_1$. Warto zaznaczyć, że wzmacniacz MCP6281 jest typu rail-to-rail, pozwalający na operowanie na sygnałach bliskich napięciom zasilania $V_+$, jak i $V_-$ wzmacniacza \cite{mcp6281_ds}. Poprzednie sekcje nie wymagały tej specjalnej właściwości. Tak przygotowany sygnał jest gotowy do analizy przez mikrokontroler. Dla dodatkowego bezpieczeństwa zastosowano diodę zenera $3,3V$ gwarantującą, że \textit{SIGNAL\_OUT} należy do zakresu napięć obsługiwanego przez sekcję cyfrową.


\subsection{Wirtualna masa}
\begin{figure}[H]
    \begin{center}
    \includegraphics[width=7.5cm]{img/vgnd.png}
    \end{center}
    \caption{Konfiguracja wzmacniacza operacyjnego generującego napięcie $VGND$}
\end{figure}

W celu uzyskania pełnej odpowiedzi cewki zastosowana jest sztuczna masa wynosząca $0,5$ napięcia zasilania. Dzięki niej wzmacniacze operacyjne w sekcji wzmocnienia odpowiedzi mogą wzmocnić odpowiedź pracując wyłącznie z dodatnim napięciem, co omija konieczność generowania ujemnego napięcia upraszczając finalny schemat wykrywacza metalu. Wykonana jest za pomocą dzielnika napięcia wykonującego działanie

\begin{equation}
    U_{VGND}=Vcc\cdot\frac{R9}{R8+R9}=Vcc\cdot\frac{1}{2}  
    \label{eq:vgnd}
\end{equation}
którego wynikowe napięcie przechodzi przez wzmacniacz operacyjny $U3$ kopiujący napięcie wejściowe wzmacniając sygnał, jednocześnie odciążając zastosowane w dzielniku rezystory.



\section{Sekcja cyfrowa}

\begin{figure}[H]
    \begin{center}
        \includegraphics[width=7.5cm]{img/stm32.png}
    \end{center}
    \caption{Cyfrowe przedstawienie układu STM32F411RE wraz z użytymi pinami}\label{fig:stm32}
\end{figure}


Część cyfrowa wykrywacza metalu oparta jest o moduł STM32H523CECoreBoard \cite{WeActStudio_STM32H523CoreBoard} zawierający układ \textbf{STM32H523CE}. Ten produkowany przez ST Microelectronics mikrokontroler posiada jeden 32-bitowy rdzeń oparty na architekturze ARM\textregistered{} Cortex-M33\textregistered{} \cite{st_stm32h5_web} \cite{st_stm32h533_ds} o maksymalnym taktowaniu 250Mhz oraz m.in wykorzystane w projekcie układy peryferyjne:

\subsection{Liczniki sprzętowe}
\label{subsec:timery}
Wykorzystywane są dwa 16-bitowe liczniki sprzętowe napędzane za pomocą zegara \textit{APB2} skonfigurowanego na taktowanie z częstotliwością $F_{timer}=250Mhz$. Częstotliwość przepełnienia licznika sprzętowego można wyznaczyć za pomocą poniższego równania: 

\begin{equation}
F_{output}=\frac{F_{timer}}{(PSC+1)(ARR+1)}
\end{equation}

gdzie PSC oznacza Prescaler, natomiast ARR oznacza Auto Reload Register


Licznik \textit{TIM10} pełni funkcję maszyny stanów zarządzając aktualnym stanem sygnału \textit{PULSE\_OUT}.
W celu zachowania dokładności przy jednoczesnym jak najmniejszym obciążeniu MCU licznik wywołuje przerwania w częstotliwości $F_{output}=1\mu s$ poprzez konfigurację $PSC=0, ARR=249$.

Podczas pracy układu cyfrowego w funkcji przerwania sprawdzana jest wartość zmiennej inkrementowanej co każde wywołanie przerwania. Jeżeli licznik jest równy 0, wystawiany jest sygnał +3.3V na pin 1 (PULSE\_OUT) układu, powodując otwarcie tranzystora MOSFET. Następnie w pożądanym momencie sygnał wyjściowy jest wygaszony, co pozwala na dynamiczną konfigurację długości impulsu wysyłanego do cewki.

\begin{figure}[H]
    \begin{center}
\begin{tikzpicture}[node distance=1cm and 2.5cm]

    \node[startstop] (start) {Wywołanie przerwania};
    
    \node[decision, below=1cm of start] (dec1) {Czy \texttt{pulseTickCtr == 0}?};
    \node[decision, below=of dec1] (dec2) {Czy \texttt{pulseTickCtr} \\ \texttt{== PULSE\_WIDTH}?};
    \node[decision, below=of dec2] (dec3) {Czy \texttt{pulseTickCtr} \\ \texttt{== PULSE\_WIDTH} \\ \texttt{+ 10us}?};
    \node[decision, below=of dec3] (dec4) {Czy \texttt{pulseTickCtr} \\ \texttt{== TOTAL\_WIDTH}?};
    
    \node[process, below=of dec4] (inc) {Inkrementuj \texttt{pulseTickCtr}};
    \node[startstop, below=of inc] (stop) {Koniec};

    \node[process, right=of dec1] (proc1) {Wystaw sygnał\\ \textbf{PULSE\_OUT}};
    \node[process] (proc2) at (proc1 |- dec2) {Wygaś sygnał\\ \textbf{PULSE\_OUT}};
    \node[process] (proc3) at (proc1 |- dec3) {Zapisz licznik DMA do\\ \textbf{dmaBufferHead}};
    \node[process] (proc4) at (proc1 |- dec4) {Ustaw \\ \texttt{pulseTickCtr = 0}};
    
    \draw[arrow] (start) -- (dec1);
    \draw[arrow] (dec1) -- node[left] {Nie} (dec2);
    \draw[arrow] (dec2) -- node[left] {Nie} (dec3);
    \draw[arrow] (dec3) -- node[left] {Nie} (dec4);
    \draw[arrow] (dec4) -- node[left] {Nie} (inc);
    \draw[arrow] (inc) -- node[left] {} (stop);
    
    \draw[arrow] (dec1) -- node[above] {Tak} (proc1);
    \draw[arrow] (dec2) -- node[above] {Tak} (proc2);
    \draw[arrow] (dec3) -- node[above] {Tak} (proc3);
    \draw[arrow] (dec4) -- node[above] {Tak} (proc4);

    \draw[arrow] (proc1) |- ($(dec1.south)!0.5!(dec2.north)$);
    \draw[arrow] (proc2) |- ($(dec2.south)!0.5!(dec3.north)$);
    \draw[arrow] (proc3) |- ($(dec3.south)!0.5!(dec4.north)$);
    \draw[arrow] (proc4) |- (stop);

\end{tikzpicture}
    \end{center}
    \caption{Procedura przerwania TIM10}
\end{figure}


Drugi licznik \textit{TIM2} wykorzystany jest do generowania sygnału PWM o wypełnieniu 50\% i dynamicznie wyznaczanej częstotliwości do brzęczyka wskazującego o wykryciu anomalii w przetworzonym sygnale \textit{PULSE\_IN}.

\subsection{Konwerter cyfrowo-analogowy}

ADC napędzany jest sygnałem zegarowym \textit{HCLK} o częstotliwości $F_{ADC}=250Mhz$, z preskalerem o wartości $4$, co daje końcową częstotliwość podukładu wynoszącą $62,5Mhz$.
Maksymalna rozdzielczość pomiaru wynosi 12 bitów, co daje dokładność na poziomie $0,8 mV$.

\begin{equation}
R= \frac{Vref}{2^N - 1} =\frac{3.3V}{4095}\approx 0,8mV
\end{equation}

gdzie $V_{ref}$  to napięcie referencyjne (tutaj $V_{in}=3.3V$), N = ilość bitów

W celu osiągnięcia jak największego zasięgu wykrywacza wymagana jest jednocześnie jak największa dokładność układu ADC oraz jego szybkość działania. Dla trybu 12-bitowego każdy pomiar zajmuje $C_{sample}=15$ cykli zegarowych dających teorytyczny limit przy ustalonym wcześniej taktowaniu zegara ADC wynoszący $4,16 MSPS$ (milionów próbek na sekundę ,ang. \textit{Mega Sample Per Second}), co daje jedną próbkę napięcia co $~0,24\mu s$.

\begin{equation}
    SPS = \frac{F_{ADC}}{C_{sample}}=\frac{62,5\cdot10^{6}}{15}\approx{4,16\cdot10^6}
\end{equation}

Dla maksymalnego odciążenia mikrokontrolera wykorzystany został tryb DMA z wyłączonym generowaniem przerwań sprzętowych. Każda nowa próbka zmierzona przez konwerter analogowo-cyfrowy zapisywana jest w do cyklicznego buforu znajdującym się pamięci SRAM wskazanym podczas konfiguracji bez dodatkowego udziału MCU.
Podczas wykonywania algorytmu wykrywania kopiowany jest pożądany indeks próbki z bufora i dokonywany jest proces analizy sygnału.

\section{Algorytm wykrywania anomalii}

Główny algorytm napędzający prototyp wykrywacza metali podzielony jest na X części.
\subsection{Odszukanie obszaru zainteresowania}

Pierwsza część odpowiada za wyszukanie obszaru zainteresowania. Ze względu na 

\begin{figure}[H]
    \begin{center}
\begin{tikzpicture}[node distance=1.5cm and 2cm]

    % Węzły
    \node[startstop] (start) {Start};
    \node[decision, below=of start] (dec1) {Czy wszystkie \\ próbki?};
    \node[decision, below=of dec1] (dec2) {Czy próbka \\ $\ge$ \texttt{THRES}?};
    
    % "Następna próbka" po lewej stronie (zgodnie z rysunkiem)
    \node[process, left=of dec2] (next) {Następna \\ próbka};
    
    % Akcja po znalezieniu progu
    \node[process, below=of dec2] (set_area) {Ustaw \\ \texttt{AREA\_START} \\ na indeks};
    
    \node[startstop, below=of set_area] (stop) {Koniec};

    % Połączenia
    \draw[arrow] (start) -- (dec1);
    
    % Decyzja 1: Czy wszystkie sprawdzone?
    \draw[arrow] (dec1) -- node[right] {Nie} (dec2);
    % Tak -> Koniec (ścieżka bokiem, omijająca resztę)
    \draw[arrow] (dec1.east) -- node[above] {Tak} ++(2,0) |- (stop);

    % Decyzja 2: Czy przekroczono próg?
    \draw[arrow] (dec2) -- node[right] {Tak} (set_area);
    \draw[arrow] (set_area) -- (stop);

    % Pętla "Nie" (powrót do sprawdzania kolejnej próbki)
    \draw[arrow] (dec2) -- node[above] {Nie} (next);
    \draw[arrow] (next) |- (dec1);

\end{tikzpicture}
    \end{center}
    \caption{}\label{fig:}
\end{figure}


\begin{figure}[H]
    \centering
\begin{tikzpicture}[node distance=1cm and 2.5cm]

    % Węzły główne (pionowa oś)
    \node[startstop] (start) {Start};
    
    \node[process, below=of start] (fetch) {Pobierz próbkę \\ (z ostatnich 10us)};
    
    \node[decision, below=of fetch] (dec1) {Czy próbka \\ > \texttt{AREA\_THRESHOLD}?};
    
    \node[decision, below=of dec1] (dec2) {Czy poprzednia \\ > \texttt{MIN\_SAMPLE}?};
    
    \node[process, below=of dec2] (filter) {Dodaj do \\ \texttt{1ST\_STAGE\_SAMPLES} \\ \textit{Oblicz} \texttt{1ST\_STAGE\_AVG}};
    
    \node[decision, below=of filter] (dec3) {Czy próbka \\ > \texttt{1ST\_STAGE\_AVG}?};
    
    \node[startstop, below=of dec3] (stop) {Koniec};

    % Węzły boczne (akcje warunkowe)
    \node[process, right=of dec2] (proc_min) {Przypisz wartość do \\ \texttt{MIN\_SAMPLE}};
    
    \node[process, right=of dec3] (proc_alarm) {Anomalia: \\ Ustal brzęczyk};

    % Połączenia - Główna ścieżka i logika Tak/Nie
    \draw[arrow] (start) -- (fetch);
    \draw[arrow] (fetch) -- (dec1);

    % Decyzja 1: Threshold
    % Nie: Idź do następnej próbki (pętla powrotna do pobierania)
    \draw[arrow] (dec1.west) -- node[above] {Nie} ++(-1,0) |- (fetch.west);
    % Tak: Szukanie najniższej próbki
    \draw[arrow] (dec1) -- node[left] {Tak} (dec2);

    % Decyzja 2: Min Sample
    % Tak: Ignoruj (przejdź od razu do filtracji)
    \draw[arrow] (dec2) -- node[left] {Tak} (filter);
    % Nie: Aktualizuj MIN_SAMPLE
    \draw[arrow] (dec2) -- node[above] {Nie} (proc_min);
    
    % Powrót z aktualizacji MIN_SAMPLE do głównego nurtu (przed filtracją)
    \draw[arrow] (proc_min) |- ($(dec2.south)!0.5!(filter.north)$);

    % Połączenie do Decyzji 3
    \draw[arrow] (filter) -- (dec3);

    % Decyzja 3: Anomalia
    % Nie: Koniec
    \draw[arrow] (dec3) -- node[left] {Nie} (stop);
    % Tak: Anomalia
    \draw[arrow] (dec3) -- node[above] {Tak} (proc_alarm);
    
    % Powrót z obsługi anomalii do Końca
    \draw[arrow] (proc_alarm) |- (stop);

\end{tikzpicture}
    % \begin{tikzpicture}
    %     \node[startstop] (start) {Start};
    %     \node[process, below=of start] (analiza) {Analiza zawartości bufora cyklicznego};
    %     \node[process, below=of analiza] (wyznaczenie) {Wyznaczenie próbki o najmniejszej wartości w obszarze zainteresowania};
    %     \node[blok, below=of wyznaczenie] (suma4) {Suma\\czterech próbek};
    %     \node[blok, below=of suma4] (suma32) {Suma 32\\próbek};
    %     \node[blok, below=of suma32] (tlo) {- background};
    %     \node[blok, below=of tlo] (anomalia) {sprawdzenie\\anomalii};
    %     \node[blok, below=of anomalia] (buzzer) {buzzer};
    %     \node[startstop, below=of buzzer] (koniec) {Koniec};
    %     \draw[arrow] (start) -- (analiza);
    %     \draw[arrow] (analiza) -- (wyznaczenie);
    %     \draw[arrow] (wyznaczenie) -- (suma4);
    %     \draw[arrow] (suma4) -- (suma32);
    %     \draw[arrow] (suma32) -- (tlo);
    %     \draw[arrow] (tlo) -- (anomalia);
    %     \draw[arrow] (anomalia) -- (buzzer);
    %     \draw[arrow] (buzzer) -- (koniec);
    % \end{tikzpicture}
    \caption{Schemat blokowy działania algorytmu}\label{fig:diagram}
\end{figure}

Algorytm wykrywania metalu oparty jest o próbkowanie sygnału \textit{PULSE\_OUT} w dynamicznie określanym odstępie czasowym. Zbierana jest suma czterech następnych próbek, która zapisywana jest do bufora cyklicznego przechowującego 32 pozycje (Stage 1). Średnia obliczona ze wszystkich wartości znajdujących się w buforze etatu 1 umieszczana jest w drugim buforze cyklicznym, mający pojemność 64 próbki. Ten bufor służy do eliminacji tła. Średnia wartości etapu pierwszego porównywana jest ze średnią wartości znajdujących się w buforze tła, każda różnica co najmniej 5 jednostek traktowana jest jako anomalia i oznacza wykrycie metalu.

\begin{figure}[H]
    \begin{center}
        \includegraphics[width=0.95\textwidth]{../meas/pulse_out.png}
    \end{center}
    \caption{Wykres sygnału \textit{PULSE\_OUT} używany w algorytmie}\label{fig:meas}

\end{figure}

\section{Płytka drukowana}
Płytka drukowana (PCB, ang. \textit{Printed Circuit Board}) prototypu została zaprojektowana w otwartoźródłowym programie typu CAD (projektowanie wspomagane komputerowo, ang. \textit{Computer Aided Design}) KiCad\cite{kicad}.
Finalny produkt został wykonany z wykorzystaniem laminatu dwustronnego o wymiarach $80mm$ x $101mm$.
W celu minimalizacji zakłóceń elektromagnetycznych, wolne przestrzenie na obu warstwach PCB zostały wypełnione wylewką miedzi (ang. \textit{copper pour}) podłączoną do potencjału masy.

\begin{figure}[H]
\centering
  \includegraphics[width=0.5\textwidth]{./pcb.pdf}
  \caption{Schemat projektu PCB (z pominiętą płaszczyzną masy)}
\end{figure}

Projekt zawiera enkoder, który umożliwia interakcję z użytkownikiem poprzez przekręcanie oraz wciskanie wystającego wału oraz dwa porty przygotowane pod moduły wyświetlaczy ILI9341 oraz SSD1306.



\begin{figure}
\centering
  \includegraphics[width=0.7\textwidth]{./img/pcb_3drender.png}
  \caption{Trójwymiarowe komputerowe przedstawienie projektu PCB}
\end{figure}

\section{Obudowa}
Obudowa wykrywacza metalu została wykonana w technologii druku 3D z wykorzystaniem plastiku PET-G gwarantującym większą wytrzymałość mechaniczną w porównaniu
do alternatyw.
Głównym założeniem projektowym było stworzenie kompaktowej konstrukcji. Wymiary zewnętrzne obudowy wynoszą odpowiednio $85mm$ x $126mm$ x $28mm$.
Projekt posiada dedykowane otwory na port sondy RJ45 oraz gniazdo zasilające DC.
Od góry wykonane zostało wycięcie na wyświetlacz SSD1306 oraz wał enkodera.
Całość zamykana jest poprzez cztery śruby M3, wkręcane we wkładki metalowe montowane na gorąco.
Obudowa PCB, jak i sondy została zaprojektowana w oprogramowaniu CAD 3D Autodesk Fusion.

\begin{figure}
    \begin{center}
        \includegraphics[width=\textwidth]{img/pudlo otwarte.png}
    \end{center}
    \caption{Trójwymiarowe przedstawienie PCB zamkniętego w zaprojektowanej obudowie}
\end{figure}

